We will now begin our exposition of the proof of the consistency of tangled type theory.
This section is based on the \texttt{BaseType} and \texttt{Structural} directories in the formalisation.

\subsection{Model parameters}
\label{ss:structure:params}

As described in \cref{ss:theories:ttt}, the types of a given model of tangled type theory are indexed by a limit ordinal \( \lambda \).
\begin{definition}
    \llabel{Params}
    A collection of \emph{model parameters} consists of types \( \lambda, \kappa, \mu \) satisfying the following requirements.
    \begin{enumerate}
        \item \( \lambda, \kappa, \mu \) have well-orderings.
        The order type of \( \lambda \) is a limit ordinal.
        The order types of \( \kappa \) and \( \mu \) are initial ordinals.\footnote{Set theorists would simply require that \( \kappa, \mu \) be cardinals. In our type theory, a cardinal is an equivalence class of types in a given universe that biject.}
        \item The cardinalities of \( \lambda, \kappa, \mu \) satisfy
        \[ \aleph_0 \leq \abs{\lambda} < \abs{\kappa} < \abs{\mu} \]
        \item \( \abs{\kappa} \) is a regular cardinal.
        \( \abs{\mu} \) is a strong limit cardinal, and \( \mu \) has cofinality at least \( \abs{\kappa} \).
    \end{enumerate}
\end{definition}
\begin{definition}
    \llabel{Small}
    Sets smaller than size \( \kappa \) are called \emph{small}.
\end{definition}
\begin{remarks}\mbox\negthinspace
    \begin{enumerate}
        \item \llabel{κ-addMonoid} \( \kappa \) has an additive monoid structure given from its well-ordered structure as follows.
        % TODO: Probably just remove the addMonoid structure from Params and just define it outside
        Write \( f : \kappa \to \mathrm{Ord} \) for the function that maps each inhabitant \( i \) of \( \kappa \) to the order type of \( \{ j \mid j < i \} \).
        Then \( f \) is an injection with image equal to the order type of \( \kappa \).
        We may now define
        \[ i + j = f^{-1}(f(i) + f(j)) \]
        We cannot define this operation constructively, so we directly include it in the model parameters in the Lean formalisation and prove its existence separately.
        \item \llabel{minimalParams}
        We remark that these constraints are satisfiable; \( \lambda = \aleph_0, \kappa = \aleph_1, \mu = \beth_{\omega_1} \) suffice.
    \end{enumerate}
\end{remarks}

\subsection{Atoms, litters, and near-litters}

As described in \cref{s:overview}, our model has an additional level of objects below type zero.
To index the levels of the model, together with this new level, we make the following definition.
\begin{definition}
    \llabel{TypeIndex}
    A \emph{type index} is an element of \( \lambda \) or a distinguished symbol \( \bot \).
    We impose an order on type indices by setting \( \bot < \alpha \) for all \( \alpha \in \lambda \).
    The collection of type indices is denoted \( \lambda^\bot \).
\end{definition}
Elements of \( \lambda \) may be called \emph{proper type indices}.

Our base type is a set of \emph{atoms}, organised into \emph{litters}.
The litters index as large amorphous sets of atoms that can be used as `junk' data.
\begin{definition}
    \llabel{Litter}
    A \emph{litter} is a triple \( L = (\nu, \beta, \gamma) \) where \( \nu \in \mu \), \( \beta \) is a type index, and \( \gamma \neq \beta \) is a proper type index.\footnote{Readers of \cite{con-nf} will note that Holmes defines a litter as a particular set of atoms. We instead define litters to be names for such sets, which will be called \emph{litter sets}.}
\end{definition}
This somewhat arcane definition will be used later when defining the fuzz map.
A litter \( L = (\nu, \beta, \gamma) \) encodes data coming from type \( \beta \) and going into type \( \gamma \).
Note that \( \beta \) may be \( \bot \), but \( \gamma \) may not; this corresponds to the fact that we never construct data in type \( \bot \) from data at higher levels.
The first component \( \nu \) is an index allowing us to have \( \mu \) distinct litters with the same source and target types.
\begin{remark}
    \llabel{mk-litter}
    There are precisely \( \abs{\mu} \) litters.
\end{remark}
\begin{definition}
    \llabel{Atom}
    An \emph{atom} is a pair \( a = (L, i) \) where \( L \) is a litter and \( i \in \kappa \).
    The \emph{associated litter} of an atom is its first projection \( \pr_1(a) \), written \( a^\circ \) for brevity.
    The \emph{litter set} \( \LS(L) \) of a given litter \( L \) is the set of atoms whose associated litter is \( L \); that is, \( \LS(L) = \{ (L, i) \mid i \in \kappa \} \).
    The litter sets partition the type of atoms into \( \abs{\mu} \) sets of \( \kappa \) atoms, and there are \( \abs{\mu} \) atoms in total.
\end{definition}
\begin{remark}
    Many of our constructions for symmetry arguments rely on having only a small set of constraints.
    If our constraints take the form of atoms, the smallness assumption guarantees that most of the atoms in a given litter set are unconstrained.
    Motivated by smallness concerns, we make the following definition.
\end{remark}
\begin{definition}
    \llabel{NearLitter}
    A \emph{near-litter} is a pair \( N = (L, s) \) where \( L \) is a litter and \( s \) is a set of atoms with small symmetric difference to the litter set of \( L \).
    We say that the \emph{associated litter} of \( N \) is \( N^\circ = \pr_1(N) \), or that \( N \) is \emph{near} \( L \).
    For brevity, we will frequently identify a near-litter with its underlying set.
\end{definition}
\begin{remarks}\mbox{\negthinspace}
    \label{rk:mk_near_litter}
    \begin{enumerate}
        \item \llabel{NearLitter.isNearLitter} A set of atoms can be near at most one litter.
        \item \llabel{Litter.toNearLitter} The litter set of any litter \( L \) can be made into a near-litter: \( \NL(L) = (L, \LS(L)) \).
        \item \llabel{mk-nearLitter''} Each (second projection of a) near-litter has size exactly \( \kappa \).
    \end{enumerate}
\end{remarks}
\begin{lemma}
    \llabel{mk-nearLitter}
    There are precisely \( \abs{\mu} \) near-litters.
\end{lemma}
\begin{proof}
    It suffices to show that if \( \abs{\alpha} \) is a strong limit cardinal and \( \alpha \) is well-ordered with initial order type, then \( \alpha \) has precisely \( \abs{\alpha} \) bounded subsets.
    Indeed, this would imply that there are only \( \abs{\mu} \) small sets of atoms, since the cofinality of \( \mu \) is at least \( \abs{\kappa} \), from which it follows that there are precisely \( \abs{\mu} \) near-litters near a given litter, and so precisely \( \abs{\mu} \) in total.
    Note that
    \[ \abs{\{ s \subseteq \alpha \mid s \text{ bounded} \}} = \abs{\bigcup_{i : \alpha} \mathcal P(\{ j \mid j < i \})} \leq \abs{\alpha} \cdot \sup_{i : \alpha} \abs{\mathcal P(\{ j \mid j < i \})} \]
    As \( \alpha \) has initial order type, \( \{ j \mid j < i \} \) has cardinality strictly less than \( \abs{\alpha} \).
    As \( \abs{\alpha} \) is a strong limit, its power set also has cardinality strictly smaller than \( \abs{\alpha} \), so the supremum is bounded above by \( \abs{\alpha} \).
\end{proof}
We can now define the allowable permutations of type \( \bot \), although we will give them a different name for now; they will be precisely those permutations of atoms that respect the structure of near-litters.
\begin{definition}
    \llabel{BasePerm}
    A \emph{base permutation} \( \pi \) consists of a permutation of atoms \( \pi^A \) and a permutation of litters \( \pi^L \) such that if \( N = (L, s) \) is a near-litter, then
    \[ (\pi^L(L), \pi^A[s]) \]
    is also a near-litter.
    The type of base permutations will occasionally be denoted \( \Base \).
\end{definition}
We will often simply use \( \pi \) to denote the permutations \( \pi^A, \pi^L \) of atoms and litters.
Such a base permutation \( \pi \) induces a permutation of near-litters \( \pi^N \), given by
\[ \pi^N(L, s) = (\pi^L(L), \pi^A[s]) \]
We will also call this permutation \( \pi \).
\begin{remarks}\mbox\negthinspace
    \begin{enumerate}
        \item \llabel{BasePerm.ext}
        A base permutation is determined by its action on atoms.
        \item \llabel{instGroupBasePerm}
        Base permutations form a group.
    \end{enumerate}
\end{remarks}
\begin{details}\mbox\negthinspace
    \begin{enumerate}
        \item Lean's type theory does not allow us to directly write \( \pi \) for the two different functions \( \pi^A, \pi^L \).
        We instead use the syntax \( \pi \bullet a, \pi \bullet L \), which is the \texttt{mathlib} notation for group actions.
        In these expressions, the head symbol is \( \bullet \), not \( \pi \) itself.
        This allows Lean to trigger typeclass resolution and dynamically choose the interpretation of \( \bullet \) based on the type of the right-hand side.
        \item We use an alternative formalisation of the pointwise image \( \pi^A[s] \) under a permutation; instead, we write \( ((\pi^A)^{-1})^{-1}(s) \).
        This has the advantage that the bi-implication
        \[ a \in ((\pi^A)^{-1})^{-1}(s) \leftrightarrow (\pi^A)^{-1}(a) \in s \]
        is a definitional equality.
    \end{enumerate}
\end{details}

\subsection{Higher type structure}

A type-\( \alpha \) object has elements of any type \( \beta < \alpha \), which have elements of any type \( \gamma < \beta \), and so on; we must eventually reach \( \bot \) in a finite number of steps by well-foundedness.
We will now make a definition to deal with sequences of type indices obtained in this way.
\begin{definition}
    \llabel{instQuiverTypeIndex}
    A \emph{path of type indices} \( \alpha \rightsquigarrow \varepsilon \) is a finite sequence of type indices
    \[ \alpha > \beta > \gamma > \dots > \varepsilon \]
    If \( \alpha \) is a type index, an \emph{\( \alpha \)-extended type index} is a path \( \alpha \rightsquigarrow \bot \).
    If \( A \) is a path \( \alpha \rightsquigarrow \beta \) and \( B \) is a path \( \beta \rightsquigarrow \gamma \), their composition \( A \pcomp B \) is a path \( \alpha \rightsquigarrow \gamma \) obtained by concatenation of the sequences but removing the duplicated index \( \beta \).
\end{definition}
\begin{detail}
    We define a quiver structure on \( \lambda^\bot \) by setting the type of morphisms \( \alpha \to \beta \) to be the type \( \alpha > \beta \); that is, there is a single morphism \( \alpha \to \beta \) whenever \( \alpha > \beta \).
    Paths of type indices are then defined as paths in this quiver.
    In \texttt{mathlib}, paths in quivers are defined as an inductive data type, with a \texttt{cons} operation on the end.
    That is, it is easy to reason about paths of the form \( \alpha \rightsquigarrow \beta > \gamma \), but difficult to reason about paths of the form \( \alpha > \beta \rightsquigarrow \gamma \).
    In our formalisation, we frequently append segments to the end of paths, but infrequently prepend to the start of paths.
    This gives rise to technical hurdles whenever analysing the start of paths is required.
    An alternative implementation is to encode paths as finite sets, as is done in \cite{con-nf}, but this representation has worse definitional equalities.
\end{detail}
\begin{remark}
    \llabel{mk-extendedIndex}
    For any \( \alpha \in \lambda^\bot \), the set of \( \alpha \)-extended type indices has size at most \( \abs{\lambda} \).
\end{remark}
In our model, the iterated extensions of objects of type \( \alpha \) are indexed by the \( \alpha \)-extended type indices.
We frequently want to analyse or modify an object `along paths': to easily package such data, we make the following definition.
\begin{definition}
    \llabel{Tree}
    Let \( \tau \) be a type and \( \alpha \) be a type index.
    Then the type of \emph{\( \alpha \)-trees of \( \tau \)} is
    \[ \Tree(\tau)_\alpha = (\alpha \rightsquigarrow \bot) \to \tau \]
    That is, an \( \alpha \)-tree of \( \tau \) assigns an inhabitant of \( \tau \) to each \( \alpha \)-extended type index.
\end{definition}
Trees made out of base permutations are called structural permutations.
\begin{definition}
    \llabel{StructPerm}
    An \emph{\( \alpha \)-structural permutation} is an \( \alpha \)-tree of base permutations.
    We write
    \[ \Str_\alpha = \Tree(\Base)_\alpha \]
    An \( \alpha \)-structural permutation \( \pi \) assigns a base permutation to each \( \alpha \)-extended index.
\end{definition}
\begin{remarks}\mbox\negthinspace
    \begin{enumerate}
        \item \llabel{Tree.group} Trees on \( \tau \) are given the group structure of \( \tau \) by acting `along paths': for \( \alpha \)-trees \( a, b \), we define \( (ab)_A = a_A b_A \).
        \item \llabel{Tree.comp} We can extend the notation \( a_A \) to paths \( A : \alpha \rightsquigarrow \beta \) where \( \beta \) is arbitrary.
        Given such a path \( A : \alpha \rightsquigarrow \beta \), we obtain a \emph{derivative} group homomorphism
        \[ \Tree(\tau)_\alpha \to \Tree(\tau)_\beta \]
        written \( a \mapsto a_A \) and given by
        \[ (a_A)_B = a_{A \pcomp B} \]
        where \( A \pcomp B \) is the concatenation of the paths \( A, B \).
        This operates similarly to a restriction map.
        \item \llabel{Tree.toBotIso} There is a canonical group isomorphism \( \Tree(\tau)_\bot \cong \tau \), as there is precisely one path \( \bot \rightsquigarrow \bot \).
        In particular, we may identify \( \bot \)-structural permutations with base permutations.
    \end{enumerate}
\end{remarks}

At proper type indices \( \alpha \), we will define the set of \( \alpha \)-allowable permutations to be a certain subgroup of the group of \( \alpha \)-structural permutations.
These permutations will be chosen in such a way that gives an action on the set of model elements at level \( \alpha \).
Not every structural permutation will have such an action.

