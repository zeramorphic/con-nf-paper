\documentclass{article}

\usepackage[a4paper]{geometry}
\usepackage{parskip}
\usepackage{fontspec}
\usepackage{unicode-math}
\newcommand{\diagup}{\char"27CB}
\setmainfont[Path=fonts/,
	UprightFont=*-Regular,
	BoldFont=*-Bold,
	ItalicFont=*-Italic,
	BoldItalicFont=*-BoldItalic,
	]{STIXTwoText}
\setmathfont[Path=fonts/]{STIXTwoMath-Regular}
\setmonofont[Path=fonts/, Scale=MatchLowercase]{FiraCode-Regular}

\usepackage[backend=biber]{biblatex}
\addbibresource{refs.bib}

\usepackage{enumerate}
\usepackage{amsthm}
\usepackage{faktor}
\usepackage{hyperref}
\usepackage{relsize}
\usepackage{quiver}

\usepackage{cleveref}
% https://tex.stackexchange.com/a/81645
\crefformat{section}{\S#2#1#3}
\crefmultiformat{section}{\S\S#2#1#3}{ and~#2#1#3}{, #2#1#3}{, and~#2#1#3}

\usepackage[shortlabels]{enumitem}
\setlist[enumerate,1]{label={(\roman*)}}
\setlist[enumerate,2]{label={(\alph*)}}

\hypersetup{
	colorlinks=true,
	linkcolor=red!50!black,
	citecolor=green!50!black,
	urlcolor=magenta!70!black,
}

\newcommand{\ttype}{\texttt{type}}
\newcommand{\mquote}[1]{\ensuremath{\text{`}#1\text{'}}}
\newcommand{\symmdiff}{\mathrel{\raisebox{1pt}{\( \mathsmaller\triangle \)}}}
\newcommand{\dom}{\operatorname{dom}}
\newcommand{\ran}{\operatorname{ran}}
\newcommand{\im}{\operatorname{im}}
\newcommand{\pr}{\symsf{pr}}
\newcommand{\LS}{\symsf{LS}}

\theoremstyle{definition}
\newtheorem{theorem}{Theorem}[section]
\newtheorem{definition}[theorem]{Definition}
\newtheorem{lemma}[theorem]{Lemma}
\newtheorem{proposition}[theorem]{Proposition}
\newtheorem{corollary}{Corollary}[theorem]

\theoremstyle{remark}
\newtheorem{remark}[theorem]{Remark}
\newtheorem{remarks}[theorem]{Remarks}

\title{New Foundations is consistent}
\author{Sky Wilshaw}
\date{July 2023}

\begin{document}

\maketitle

\begin{abstract}
	We give a self-contained account of a version of Holmes' proof \cite{holmes2023nf} that Quine's set theory \emph{New Foundations} \cite{quine-nf} is consistent relative to the metatheory ZFC.
	This is a `deformalisation' of the formal proof written in Lean at \cite{leanprover-community-con-nf}.
\end{abstract}

\tableofcontents

\section*{Overview}

In \cref{sec:theories}, we outline the context for the proof we will present.
The mathematical background expected in subsequent sections will be limited to basic familiarity with cardinals and ordinals.
We will then give an outline of the proof in \cref{sec:outline}.
In \cref{sec:structure}, we introduce some basic preliminaries, and explicitly describe the structure within which our model will reside.

All proofs given in \cref{sec:structure} are verified by the theorem prover Lean.

\emph{
	Note: At the present time, the formal proof \cite{leanprover-community-con-nf} is incomplete, and this paper reflects the unfinished state of that proof.
	We aim to keep this paper proof in line with the formal proof, although as the project is ongoing, some variance is to be expected.
	The current version of the paper is available at \url{https://zeramorphic.github.io/con-nf-paper/main.pdf}.
}

\section{The theories at issue}
\label{sec:theories}
In 1937, Quine introduced \emph{New Foundations} (\( \mathsf{NF} \)) \cite{quine-nf}, a set theory with a very small collection of axioms.
To give a proper exposition of the theory that we intend to prove consistent, we will first make a digression to introduce the related theory \( \mathsf{TST} \), as explained by Holmes in \cite{holmes2023nf}.
We will then describe the theory \( \mathsf{TTT} \), which we will use to prove our theorem.

\subsection{The simple theory of types}

The \emph{simple theory of types} (known as \emph{théorie simple des types} or \( \mathsf{TST} \)) is a first order set theory with several sorts, indexed by the nonnegative integers.
Each sort, called a \emph{type}, is comprised of \emph{sets} of that type; each variable \( x \) has a nonnegative integer \( \ttype(\mquote x) \) which denotes the type it belongs to.
For convenience, we may write \( x^n \) to denote a variable \( x \) with type \( n \).

The primitive predicates of this theory are equality and membership.
An equality \( \mquote{x = y} \) is a well-formed formula precisely when \( \ttype(\mquote{x}) = \ttype(\mquote{y}) \), and similarly a membership formula \( \mquote{x \in y} \) is well-formed precisely when \( \ttype(\mquote{x}) + 1 = \ttype(\mquote{y}) \).

The axioms of this theory are extensionality
\[ \forall x^{n + 1}.\, \forall y^{n + 1}.\, (\forall z^n.\, z^n \in x^{n+1} \leftrightarrow z^n \in y^{n+1}) \to x^{n+1} = y^{n+1} \]
and comprehension
\[ \exists x^{n + 1}.\, \forall y^n.\, (y^n \in x^{n+1} \leftrightarrow \varphi(y^n)) \]
where \( \varphi \) is any well-formed formula, possibly with parameters.

\begin{remarks}\mbox{\negthinspace}
	\begin{enumerate}
		\item These are both axiom schemes, instantiated for all type levels \( n \), and (in the latter case) for all well-formed formulae \( \varphi \).
		\item The inhabitants of type 0, called \emph{individuals}, cannot be examined using these axioms.
		\item By comprehension, there is a set at each nonzero type that contains all sets of the previous type.
		Russell-style paradoxes are avoided as formulae of the form \( x^n \in x^n \) are ill-formed.
		% \item A theory has \emph{atoms}, or \emph{urelements}, if there are objects that have no members yet are not equal.
		% The axiom of extensionality prohibits atoms from existing in any positive type, since any two such objects would be equal.
		% Note that there is a different empty set in each type, but they are not atoms; the formula \( \varnothing^n \neq \varnothing^m \) is ill-formed for \( n \neq m \), so they cannot be called `not equal'.
	\end{enumerate}
\end{remarks}
% Replacing the extensionality axiom with
% \[ \forall x^{n + 1}.\, \forall y^{n + 1}.\, \forall w^n.\, w \in x \to ((\forall z^n.\, z^n \in x^{n+1} \leftrightarrow z^n \in y^{n+1}) \to x^{n+1} = y^{n+1}) \]
% yields a theory in which atoms in positive types are permitted.
% We will name the resulting theory \( \mathsf{TST} \)U, or \( \mathsf{TST} \) with urelements.

\subsection{New Foundations}

New Foundations is a one-sorted first-order theory based on \( \mathsf{TST} \).
Its primitive propositions are equality and membership.
There are no well-formedness constraints on these primitive propositions.

Its axioms are precisely the axioms of \( \mathsf{TST} \) with all type annotations erased.
That is, it has an axiom of extensionality
\[ \forall x.\, \forall y.\, (\forall z.\, z \in x \leftrightarrow z \in y) \to x = y \]
and an axiom scheme of comprehension
\[ \exists x.\, \forall y.\, (y \in x \leftrightarrow \varphi(y)) \]
the latter of which is defined for those formulae \( \varphi \) that can be obtained by erasing the type annotations of a well-formed formula of \( \mathsf{TST} \).
Such formulae are called \emph{stratified}.
To avoid the explicit dependence on \( \mathsf{TST} \), we can equivalently characterise the stratified formulae as follows.
A formula \( \varphi \) is said to be stratified when there is a function \( \sigma \) from the set of variables to the nonnegative integers, in such a way that for each subformula \( \mquote{x = y} \) of \( \varphi \) we have \( \sigma(\mquote{x}) = \sigma(\mquote{y}) \), and for each subformula \( \mquote{x \in y} \) we have \( \sigma(\mquote{x}) + 1 = \sigma(\mquote{y}) \).

\begin{remarks}\mbox{\negthinspace}
	\begin{enumerate}
		\item It is important to emphasise that while the axioms come from a many-sorted theory, \( \mathsf{NF} \) is not one; it is well-formed to ask if any set is a member of, or equal to, any other.
		\item Russell's paradox is avoided because the set \( \{ x \mid x \notin x \} \) cannot be formed; indeed, \( x \notin x \) is an unstratified formula.
		Note, however, that the set \( \{ x \mid x = x \} \) is well-formed, and so we have a universe set.
		% (TODO: add reference for Burali--Forti and Cantor)
		% \item The infinite set of stratified comprehension axioms can be captured by a finite subset; this is a result of Hailperin \cite{hailperin-finite-axiomatisation}.
		\item Specker showed in \cite{specker-choice-nf} that \( \mathsf{NF} \) disproves the Axiom of Choice.
	\end{enumerate}
\end{remarks}

While our main result is that New Foundations is consistent, we attack the problem by means of an indirection through a third theory.

\subsection{Tangled type theory}
\label{ss:theories:ttt}

Introduced by Holmes in \cite{holmes-ttt}, \emph{tangled type theory} (\( \mathsf{TTT} \)) is a multi-sorted first order theory based on \( \mathsf{TST} \).
This theory is parametrised by a limit ordinal \( \lambda \), the elements of which will index the sorts; \( \omega \) works, but we prefer generality.
As in \( \mathsf{TST} \), each variable \( x \) has a type that it belongs to, denoted \( \ttype(\mquote x) \).
However, in \( \mathsf{TTT} \), this is not a positive integer, but an element of \( \lambda \).

The primitive predicates of this theory are equality and membership.
An equality \( \mquote{x = y} \) is a well-formed formula when \( \ttype(\mquote{x}) = \ttype(\mquote{y}) \).
A membership formula \( \mquote{x \in y} \) is well-formed when \( \ttype(\mquote{x}) < \ttype(\mquote{y}) \).

The axioms of \( \mathsf{TTT} \) are obtained by taking the axioms of \( \mathsf{TST} \) and replacing all type indices in a consistent way with elements of \( \lambda \).
More precisely, for any order-embedding \( s : \omega \to \lambda \), we can convert a well-formed formula \( \varphi \) of \( \mathsf{TST} \) into a well-formed formula \( \varphi^s \) of \( \mathsf{TTT} \) by replacing each type variable \( \alpha \) with \( s(\alpha) \).

\begin{remarks}\mbox{\negthinspace}
	\begin{enumerate}
		\item Membership across types in \( \mathsf{TTT} \) behaves in some quite bizarre ways.
		Let \( \alpha \in \lambda \), and let \( x \) be a set of type \( \alpha \).
		For any \( \beta < \alpha \), the extensionality axiom implies that \( x \) is uniquely determined by its type-\( \beta \) elements.
		However, it is simultaneously determined by its type-\( \gamma \) elements for any \( \gamma < \alpha \).
		In this way, one extension of a set controls all of the other extensions.
		\item The comprehension axiom allows a set to be built which has a specified extension in a single type.
		The elements not of this type may be considered `controlled junk'.
	\end{enumerate}
\end{remarks}

We now present the following striking theorem.

\begin{theorem}[Holmes]
	\( \mathsf{NF} \) is consistent if and only if \( \mathsf{TTT} \) is consistent. \cite{holmes-ttt}
\end{theorem}

We will actually prove something slightly stronger.

\begin{theorem}
    \label{thm:nf_ttt}
    Let \( T \) be a theory in the language of \( \mathsf{TST} \).
    Let \( T_{\mathsf{NF}} \) be the theory in the language of \( \mathsf{NF} \) given by erasing the type annotations of \( T \).
    Let \( T_{\mathsf{TTT}} \) be the theory in the language of \( \mathsf{TTT} \) given by instantiating the sentences of \( T \) at all possible combinations of type levels.
    Then \( T_{\mathsf{NF}} \) is consistent if and only if \( T_{\mathsf{TTT}} \) is consistent.
\end{theorem}
\begin{proof}
    Suppose that \( T_{\mathsf{NF}} \) has a model \( M \).
    Let \( N \) be the structure in the language of \( \mathsf{TTT} \) where each type \( \alpha \) is interpreted as \( M \), and where the membership relation is given by that on \( M \).
    It is easy to see by induction that all sentences in \( T_{\mathsf{TTT}} \) hold in \( N \), as required.

    Now suppose that \( T_{\mathsf{TTT}} \) has some model \( M \).
    This proof that \( T_{\mathsf{NF}} \) is consistent proceeds in two stages.
    In the first stage, we show that \( T + \mathsf{Amb} \) is consistent, where \( \mathsf{Amb} \) is the \emph{ambiguity scheme}
    \[ \mathsf{Amb} \equiv \{ \varphi \leftrightarrow \varphi^+ \mid \varphi \text{ is a sentence in the language of } \mathsf{TST} \} \]
    % TODO: Write about the (-)^+ operation
    This result is due to Holmes in \cite{holmes-ttt}.
    We will then use this to show that \( T_{\mathsf{NF}} \) is consistent, using a result due to Specker in \cite{typical-ambiguity}.

    Suppose that \( T + \mathsf{Amb} \) is not consistent.
    By compactness, there is some finite set of sentences \( \Sigma \) in the language of \( \mathsf{TST} \) such that \( T + \mathsf{Amb}_\Sigma \) is inconsistent, where
    \[ \mathsf{Amb}_\Sigma \equiv \{ \varphi \leftrightarrow \varphi^+ \mid \varphi \in \Sigma \} \]
    Suppose that \( \Sigma \) uses only type indices \( 0, \dots, n - 1 \).
    Let \( [\lambda]^n \) be the collection of \( n \)-element subsets of \( \lambda \), and define a function \( \sigma : [\lambda]^n \to \mathcal P(\Sigma) \) as follows.
    If
    \[ A = \{\alpha_0, \dots, \alpha_{n-1}\} \text{ with } \alpha_0 < \dots < \alpha_{n-1} \]
    then \( \varphi \in \sigma(A) \) if and only if the interpretation of \( \varphi \) in \( M \) at levels \( \alpha_0, \dots, \alpha_{n-1} \) is true.
    % TODO: Write about interpreting TST formulas in a TTT structure
    This defines a partition of \( [\lambda]^n \) into finitely many subsets.
    By Ramsey's theorem, there is an infinite homogeneous set \( H \subseteq \lambda \) for this partition, that is, if \( A, B \in [H]^n \), then \( \sigma(A) = \sigma(B) \).
    Let \( \alpha_0, \alpha_1, \dots \) be an increasing sequence in \( H \), and define a structure \( N \) in the language of \( \mathsf{TST} \) by interpreting type \( i \) as \( M_{\alpha_i} \).
    Then, \( N \) models \( T + \mathsf{Amb}_\Sigma \) as required.

    Now, we show that the consistency of \( T + \mathsf{Amb} \) implies that of \( T_{\mathsf{NF}} \).
    This relies on a lemma of Specker in \cite{typical-ambiguity}.
    An \emph{endomorphism} of a one-sorted language is an operation \( (-)^\ast \) on the function and relation symbols, mapping them to terms (respectively formulas) with the same free variables.
    This extends in a natural way to formulas in the language.

    We can reformalise \( T \) into a theory \( T' \) over a one-sorted language by adding a unary relation symbol \( T_n \) for each type index \( n \), and recursively replacing each instance of \( \exists x^n.\, \varphi \) with \( \exists x.\, T_n(x) \wedge \varphi \).
    This language has an endomorphism \( (-)^+ \) which maps \( T_n \) to \( T_{n+1} \).

    Specker's lemma can be phrased in the following way.
    \begin{lemma}
        Let \( U \) be a complete theory in a one-sorted language \( L \) with endomorphism \( (-)^\ast \).
        Then if
        \[ U + \{ \varphi \leftrightarrow \varphi^\star \mid \varphi \text{ is an \( L \)-sentence} \} \]
        is consistent, then there is a model \( M \) of \( U \) that admits a function \( f : M \to M \) such that for every relation symbol \( R \) of \( L \),
        \[ M \vDash R(x_1, \dots, x_m) \text{ if and only if } M \vDash R(f(x_1), \dots, f(x_m)) \]
    \end{lemma}
    In our case, \( T + \mathsf{Amb} \) is consistent, so the corresponding one-sorted theory as required for the lemma is consistent (and has a complete extension).
    This requires choosing an interpretation of the membership relation for pairs of type indices that do not differ by one, but this does not interfere with anything that we need (for instance, the relation can always be interpreted as false).
    This yields a model of \( T' \) with a type-raising function \( f \).
    This naturally gives rise to a model of \( T \) in the language of \( \mathsf{TST} \) in which all type levels are isomorphic.
    Therefore, the carrier set of each type level of this model provides a model of \( T_{\mathsf{NF}} \) as required.
\end{proof}

Thus, our task of proving \( \mathsf{NF} \) consistent is reduced to the task of proving \( \mathsf{TTT} \) consistent.
We will do this by exhibiting an explicit model (albeit one that requires a great deal of Choice to construct).
As \( \mathsf{TTT} \) has types indexed by a limit ordinal, and sets can only contain sets of lower type, we can construct a model by recursion over \( \lambda \).
In particular, a model of \( \mathsf{TTT} \) is a well-founded structure.
This was not an option with \( \mathsf{NF} \) directly, as the universe set \( \{ x \mid x = x \} \) would necessarily be constructed before many of its elements.

\subsection{Finitely axiomatising tangled type theory}

Hailperin showed in \cite{hailperin-finite-axiomatisation} that the comprehension scheme of \( \mathsf{NF} \) is equivalent to a finite conjunction of its instances.
These axioms are all stratified (as is extensionality), so \( \mathsf{NF} \) is equivalent to a theory of the form \( T_{\mathsf{NF}} \) where \( T \) is a particular finite theory in the language of \( \mathsf{TST} \).
Then, by \cref{thm:nf_ttt}, the consistency of \( \mathsf{NF} \) can be established by witnessing a model of \( T_{\mathsf{TTT}} \).
The same theorem shows that any model of \( T_{\mathsf{TTT}} \) is a model of \( \mathsf{TTT} \), by executing Hailperin's proof in the language of \( \mathsf{NF} \) and transporting the result back to the language of \( \mathsf{TTT} \).

We will exhibit one such theory \( T \) here, with a list of eleven axioms.
Our choice of axioms for the comprehension scheme are inspired by those used in the Metamath implementation of Hailperin's algorithm in \cite{metamath-nf}.
In the following table, the notation \( \langle a, b \rangle \) denotes the Kuratowski pair \( \{ \{ a \}, \{ a, b \} \} \).
The first column is Hailperin's name for the axiom.
\begin{center}
    \begin{tabular}{lll}
        \( - \) & extensionality & \( \forall x^1.\, \forall y^1.\, (\forall z^0.\, z \in x \leftrightarrow z \in y) \to x = y \) \\
        P1 & anti-intersection & \( \forall x^1 y^1.\, \exists z^1.\, \forall w^0.\, w \in z \leftrightarrow \neg(w \in x \wedge w \in y) \) \\
        P2 & singleton image & \( \forall x^3.\, \exists y^4.\, \forall z^0 w^0.\, \langle \{ z \}, \{ w \} \rangle \in y \leftrightarrow \langle z, w \rangle \in x \) \\
        \( - \) & singleton & \( \forall x^0.\, \exists y^1.\, \forall z^0.\, z \in y \leftrightarrow z = x \) \\
        P3 & insertion two & \( \forall x^3.\, \exists y^5.\, \forall z^0 w^0 t^0.\, \langle \{ \{ z \} \}, \langle w, t \rangle \rangle \in y \leftrightarrow \langle z, t \rangle \in x \) \\
        P4 & insertion three & \( \forall x^3.\, \exists y^5.\, \forall z^0 w^0 t^0.\, \langle \{ \{ z \} \}, \langle w, t \rangle \rangle \in y \leftrightarrow \langle z, w \rangle \in x \) \\
        P5 & cross product & \( \forall x^1.\, \exists y^3.\, \forall z^2.\, z \in y \leftrightarrow \exists w^0 t^0.\, z = \langle w, t \rangle \wedge t \in x \) \\
        P6 & type lowering & \( \forall x^4.\, \exists y^1.\, \forall z^0.\, z \in y \leftrightarrow \forall w^1.\, \langle w, \{ z \} \rangle \in x \) \\
        P7 & converse & \( \forall x^2.\, \exists y^2.\, \forall z^0 w^0.\, \langle z, w \rangle \in y \leftrightarrow \langle w, z \rangle \in x \) \\
        P8 & cardinal one & \( \exists x^2.\, \forall y^1.\, y \in x \leftrightarrow \exists z^0.\, \forall w.\, w \in y \leftrightarrow w = z \) \\
        P9 & subset & \( \exists x^3.\, \forall y^1 z^1.\, \langle y, z \rangle \in x \leftrightarrow \forall w^0.\, w \in y \to w \in z \)
    \end{tabular}
\end{center}


\section{Outline}
\label{sec:outline}
To construct a model of tangled type theory, we build each type individually, and then prove that the resulting structure satisfies the required axioms.
The process for building each type is complicated, and depends on some knowledge about the construction of the previous types.
In the following subsections, we outline the construction the types, as well as the precise facts we need to carry through the inductive hypothesis at each stage.

\subsection{Model parameters}
\label{ss:outline:params}

As described in \cref{ss:theories:ttt}, the types of a given model of tangled type theory are indexed by a limit ordinal \( \lambda \).
Our model will also have two more cardinal parameters, denoted \( \kappa \) and \( \mu \), satisfying \( \lambda < \kappa < \mu \).

Sets smaller than size \( \kappa \) will be called \emph{small}.
We require that \( \kappa \) is a regular cardinal; this ensures that small-indexed unions of small sets are small.
Note that \( \aleph_0 \) is small.

Each type in our model will have size \( \mu \).
We require \( \mu \) to be a strong limit cardinal; power sets of sets smaller than \( \mu \) must also be smaller than \( \mu \).
We stipulate that the cofinality of \( \mu \) is at least \( \kappa \).
This assumption will become important whenever we consider objects indexed by small ordinals.

We remark that these constraints are satisfiable; \( \lambda = \aleph_0, \kappa = \aleph_1, \mu = \beth_{\omega_1} \) suffice.

\subsection{Atoms and permutations}
\label{ss:outline:atoms}

To aid our construction, we will add an additional level of objects below type zero.
These will not be a part of the final model we construct.
This base type will be comprised of objects called \emph{atoms} (although they are not atoms in the traditional model-theoretic sense).

Alongside the construction of the types of our model, we will also construct a collection of permutations of each type, called the \emph{allowable permutations}.
Such permutations will preserve the structure of the model in a strong sense; for instance, they preserve membership.

\subsection{Construction of each type}
\label{ss:outline:construction}

Objects in our model are defined by their elements at all lower type indices.
However, not all collections of extensions may become model elements; for example, they may fail to satisfy extensionality at all levels simultaneously.
We impose two restrictions on what kind of extensions an object may have.

The first restriction is that one of the extensions of a given object must be `preferred', and every other extension must be easily derivable from that particular extension.
This will help us to establish extensionality, as model elements will be the same if and only if their preferred extensions are the same.
The system to compute other extensions uses a construction called the \emph{fuzz} map.
This map turns information about one extension into `ordered junk' in another extension, in such a way that the model cannot learn anything useful about the non-preferred extensions.
Our allowable permutations will be defined as a set of permutations that respect the fuzz map.

The second restriction is that the object must have a small \emph{support} comprised of \emph{addresses}.
That is, the behaviour of the object under the action of allowable permutations must be fully characterisable by the behaviour of a small set of addresses under allowable permutations.
This will ensure that the objects of our model are not too complex.
Because the cofinality of \( \mu \) is at least \( \kappa \), there are only \( \mu \) small sets of elements taken from a collection of size \( \mu \); this observation will play a key role in establishing the sizes of our types.

\subsection{Constraining the size of each type}
\label{ss:outline:size}

The construction of a given type can only be done under the assumption that each smaller type was of size exactly \( \mu \).
This means that we need to prove that each type has size \( \mu \) in the inductive step.
In order to do this, we will need to show that there are a lot of allowable permutations.
The main theorem establishing this, called the \emph{freedom of action theorem}, roughly states that under certain assumptions, a permutation defined on a small set of addresses can be extended to an allowable permutation.
The majority of this paper will be allocated to proving the freedom of action theorem, and it will be outlined in more detail when we are in a position to prove it.
One this is established, we can prove that the size of each type is precisely \( \mu \) by carefully counting the possible ways to describe a model element.

\subsection{Finishing the induction}
\label{ss:outline:finishing}

We can then finish the inductive step and build the entire model.
It remains to show that this is a model of TTT as desired.
This part of the proof is quite direct, and also uses the freedom of action theorem.


\section{The underlying structure}
\label{sec:structure}
\subsection{Atoms, litters, and near-litters}

As described in \cref{ss:outline:atoms}, we have an additional level of objects below type zero.
To index the levels of the model, together with this new level, we make the following definition.
\begin{definition}
    A \emph{type index} is an element of \( \lambda \) or a distinguished symbol \( \bot \).
    We impose an order on type indices by setting \( \bot < \alpha \) for all \( \alpha \in \lambda \).
    The set of type indices is denoted \( \lambda^\bot \).
\end{definition}
Elements of \( \lambda \) may be called \emph{proper type indices}.

Our base type is a set of \emph{atoms}, organised into \emph{litters}.
\begin{definition}
    A \emph{litter} is a triple \( L = (\nu, \beta, \gamma) \) where \( \nu \in \mu \), \( \beta \) is a type index, and \( \gamma \neq \beta \) is a proper type index.
\end{definition}
This somewhat arcane definition will be used to great effect later when defining the fuzz map.
A litter \( L = (\nu, \beta, \gamma) \) encodes data coming from type \( \beta \) and going into type \( \gamma \).
Note that \( \beta \) may be \( \bot \), but \( \gamma \) may not; this corresponds to the fact that we never construct data in type \( \bot \) from data at higher levels.
The first component \( \nu \) is an index allowing us to have \( \mu \) distinct litters with the same source and target types.
\begin{remark}
    There are precisely \( \mu \) litters.
\end{remark}
\begin{definition}
    An \emph{atom} is a pair \( a = (L, i) \) where \( L \) is a litter and \( i \in \kappa \).
    The \emph{associated litter} of an atom is its first projection \( \pr_1(a) \), written \( a^\circ \) for brevity.
    The \emph{litter set} \( \LS(L) \) of a given litter \( L \) is the set of atoms whose associated litter is \( L \); that is, \( \LS(L) = \{ (L, i) \mid i \in \kappa \} \).
    The litter sets partition the set of atoms into \( \mu \) sets of \( \kappa \) atoms, and there are \( \mu \) atoms in total.
\end{definition}
\begin{remark}
    Many of our constructions rely on having only a small set of constraints.
    If our constraints take the form of atoms, the smallness assumption guarantees that most of the atoms in a given litter are unconstrained.
    Motivated by smallness concerns, we make the following definition.
\end{remark}
\begin{definition}
    A \emph{near-litter} is a pair \( N = (L, s) \) where \( L \) is a litter and \( s \) is a set of atoms with small symmetric difference to the litter set of \( L \).
    We say that the \emph{associated litter} of \( N \) is \( N^\circ = \pr_1(N) \), or that \( N \) is \emph{near} \( L \).
\end{definition}
\begin{remarks}\mbox{\negthinspace}
    \label{rk:mk_near_litter}
    \begin{enumerate}
        \item A set of atoms can be near at most one litter.
        For brevity, we will frequently identify a near-litter with its underlying set.
        \item The litter set of any litter \( L \) can be made into a near-litter: \( \NL(L) = (L, \LS(L)) \).
        \item Each near-litter has size exactly \( \kappa \), and there are \( \mu \) near-litters in total; the latter follows from the fact that the cofinality of \( \mu \) is at least \( \kappa \).
    \end{enumerate}
\end{remarks}
We can now define the allowable permutations of type \( \bot \), although we will give them a different name for now; they will be precisely those permutations of atoms that respect the structure of near-litters.
\begin{definition}
    A \emph{near-litter permutation} \( \pi \) is a permutation of atoms that sends near-litters to near-litters.
\end{definition}
\begin{remarks}\mbox{\negthinspace}
    \begin{enumerate}
        \item A near-litter permutation \( \pi \) induces a permutation of litters, which we will also call \( \pi \).
        This is defined by mapping \( L \) to the associated litter of \( \pi '' \LS(L) \), where the double apostrophe denotes pointwise function application (\( f '' s \) denotes the set \( \{ f(x) \mid x \in s \} \)).
        Thus, a near-litter permutation is simultaneously a permutation of atoms, litters, and near-litters.
        \item The set of near-litter permutations forms a group under composition.
    \end{enumerate}
\end{remarks}

\subsection{Higher type structure}

A type-\( \alpha \) object has elements of any type \( \beta < \alpha \), which have elements of any type \( \gamma < \beta \), and so on; we must eventually reach \( \bot \) in a finite number of steps by well-foundedness.
We will now make a definition to deal with sequences of type indices obtained in this way.
\begin{definition}
    A \emph{path of type indices} \( \alpha \rightsquigarrow \varepsilon \) is a finite sequence of type indices
    \[ \alpha > \beta > \gamma > \dots > \varepsilon \]
    If \( \alpha \) is a type index, an \emph{\( \alpha \)-extended type index} is a path \( \alpha \rightsquigarrow \bot \).
    If \( A \) is a path \( \alpha \rightsquigarrow \beta \) and \( B \) is a path \( \beta \rightsquigarrow \gamma \), their composition \( A \gg B \) is a path \( \alpha \rightsquigarrow \gamma \) obtained by concatenation of the sequences but removing the duplicated index \( \beta \).
\end{definition}
\begin{remark}
    \label{rk:mk_extended_index}
    For any \( \alpha \in \lambda^\bot \), the set of \( \alpha \)-extended type indices has size at most \( \lambda \).
\end{remark}
In our model, the iterated extensions of objects of type \( \alpha \) are indexed by the \( \alpha \)-extended type indices.
We can apply operations to an object along each path \( \alpha \rightsquigarrow \bot \).
One important notion defined in this way will be \emph{structural permutations}.
\begin{definition}
    \label{def:struct_perm}
    An \emph{\( \alpha \)-structural permutation} \( \pi \) assigns a near-litter permutation to each \( \alpha \)-extended index.
    If \( A \) is an \( \alpha \)-extended index, the assigned near-litter permutation is called the \emph{derivative} of \( \pi \) along \( A \), and is denoted \( \pi_A \).
    We can extend this notion to arbitrary paths \( \alpha \rightsquigarrow \beta \); in this case, the derivative \( \pi_A \) is a \( \beta \)-structural permutation.
    The group of \( \alpha \)-structural permutations is denoted \( S_\alpha \).
\end{definition}
\begin{remark}
    Near-litter permutations may be identified with \( \bot \)-structural permutations.
\end{remark}
To apply an \( \alpha \)-structural permutation \( \pi \) to some model element \( x \) of type \( \alpha \), we first consider its elements of some type \( \beta < \alpha \).
By recursion, we apply \( \pi_{(\alpha > \beta)} \) to each such element, where \( \pi_{(\alpha > \beta)} \) is the derivative of \( \pi \) along the one-step path from \( \alpha \) to \( \beta \).
These images are then assembled to form a new object of type \( \alpha \); note however that this new object need not in general satisfy the constraints outlined in \cref{ss:outline:construction}, so structural permutations need not map elements of the model to other elements of the model.

\subsection{Addresses and supports}

We are interested in objects that can be characterised by a small amount of data; this data will take the form of \emph{addresses}.
\begin{definition}
    Let \( \alpha \) be a type index.
    An \emph{\( \alpha \)-address} is a pair \( (A, x) \) where \( A \) is an \( \alpha \)-extended type index and \( x \) is either an atom or a near-litter.
\end{definition}
An \( \alpha \)-address encodes a set of atoms, and a path to get there from a type-\( \alpha \) object by descending through iterated extensions.
\begin{note}
    In \cite{holmes2023nf}, addresses are called \emph{support conditions}.
\end{note}
\begin{remark}
    \label{rk:mk_address}
    The set of \( \alpha \)-addresses has cardinality \( \mu \).
    This follows from \cref{rk:mk_near_litter,rk:mk_extended_index}.
\end{remark}

The group of \( \alpha \)-structural permutations acts on the set of \( \alpha \)-addresses by
\[ \pi(A, x) = (A, \pi_A(x)) \]
We would like to say that a \emph{support} is a small set of \( \alpha \)-addresses; for technical reasons that will be discussed later it will actually be better to make the following definition instead.
\begin{definition}
    An \( \alpha \)-support is a function \( S \) whose domain is a small ordinal and whose values are \( \alpha \)-addresses, such that whenever \( N_1, N_2 \) are near-litters in the range of \( S \), the atoms in their symmetric difference \( N_1 \symmdiff N_2 \) are also in the range of \( S \).
    The \emph{underlying set} of a support is its range.
\end{definition}
\begin{remarks}\mbox{\negthinspace}
    \begin{enumerate}
        \item An address may occur multiple times in the range of a support, but all we usually care about is the presence or absence of an address.
        The fact that these addresses have indices in a given support will be useful for some constructions later.
        The symmetric difference condition is also a technical condition that we will use later, it avoids a problem where initial conditions for a construction could be overspecified.
        \item For any small set of addresses \( s \), there is a support \( S \) whose range is \( s \) together with the small set of atom addresses required to satisfy the symmetric difference constraint.
    \end{enumerate}
\end{remarks}
As with many parts of our construction, we will need to count precisely how many supports there are.
We will first need to establish the following lemma.
\begin{lemma}
    \label{lem:konig_converse}
    Let \( \mu \) be a strong limit cardinal.
    Then for any \( \kappa > 0 \) strictly smaller than the cofinality of \( \mu \), we have
    \[ \mu^\kappa = \mu \]
\end{lemma}
Recall that \emph{König's theorem} states that for an infinite cardinal \( \mu \), we have \( \mu^{\operatorname{cf}(\kappa)} > \mu \); this lemma is a partial converse to this theorem in the case where \( \mu \) is a strong limit.
\begin{proof}
    Let \( A \) be a set of size \( \kappa \), and let \( B \) be a set of size \( \mu \).
    We assume that \( B \) has a well-ordering, perhaps obtained from the order on \( \mu \).
    Given any function \( f : A \to B \), we define a relation \( \prec_f \) on \( A \) by the inverse image of the ordering on \( B \) under \( f \):
    \[ a \prec_f b \iff f(a) < f(b) \]
    One can show by induction on \( B \) that any for two functions \( f, g : A \to B \), if \( \im f = \im g \) and \( \prec_f = \prec_g \), then \( f = g \).
    The amount of pairs \( (\im f, \prec_f) \) is bounded by \( \mu \).
    Indeed, the amount of possible images of a function \( f : A \to B \) is bounded by \( \mu \) as \( \kappa \) is less than the cofinality of \( \mu \) and \( \mu \) is a strong limit, and the amount of relations \( \prec_f \) is bounded by \( (2^\kappa)^\kappa \), which is bounded by \( \mu \) as it is a strong limit.
    Thus, \( \mu^\kappa \leq \mu \), and the converse is clear.
\end{proof}
We can now prove that there are exactly \( \mu \) supports for any type index \( \alpha \).
\begin{proposition}
    \label{prop:mk_support}
    Let \( \alpha \) be a type index.
    The set of \( \alpha \)-supports has cardinality \( \mu \).
\end{proposition}
\begin{proof}
    A support is a function from a small ordinal to the set of \( \alpha \)-addresses, which has cardinality \( \mu \) by \cref{rk:mk_address}.
    Thus, the amount of supports is bounded by
    \[ \sum_{i < \kappa} \mu^i \]
    By \cref{lem:konig_converse}, each summand is precisely \( \mu \), and so the sum is precisely \( \kappa \cdot \mu = \mu \).
\end{proof}
The \( \alpha \)-structural permutations act on \( \alpha \)-supports pointwise:
\[ \pi(S)(i) = \pi(S(i)) \]
We will need to concatenate supports together.
The naïve concatenation of supports \( S_1, S_2 \) given by
\[ S(i) = \begin{cases}
    S_1(i) & \text{if } i \in \dom S_1 \\
    S_2(i - \dom S_1) & \text{otherwise}
\end{cases} \]
need not satisfy the symmetric difference condition, so \( S \) may not be a support.
There is not a unique way to expand this concatenation into a support, so we will instead make the following definition.
\begin{definition}
    Let \( f \) be a function from a small ordinal whose values are \( \alpha \)-support conditions.
    We say that a support \( S \) is a \emph{completion} of \( f \) if \( S|_{\dom f} = f \) and every address in the extension \( S|_{\dom S \setminus \dom f} \) is an atom address \( (A, x) \) where \( x \in N_1 \symmdiff N_2 \) and \( (A, N_1), (A, N_2) \in \ran f \).
    Every such function has a completion, as the amount of atom addresses in question is small.
    A support \( S \) is a \emph{sum} of supports \( S_1, S_2 \) if it is a completion of the concatenation of \( S_1 \) and \( S_2 \).
    Every pair of supports has a sum.
\end{definition}


\printbibliography

\end{document}
