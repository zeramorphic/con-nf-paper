\documentclass{article}

\usepackage[a4paper]{geometry}
\usepackage{parskip}
\usepackage{fontspec}
\usepackage{unicode-math}
\newcommand{\diagup}{\char"27CB}
\setmainfont[Path=fonts/,
	UprightFont=*-Regular,
	BoldFont=*-Bold,
	ItalicFont=*-Italic,
	BoldItalicFont=*-BoldItalic,
	]{STIXTwoText}
\setmathfont[Path=fonts/]{STIXTwoMath-Regular}
\setmonofont[Path=fonts/, Scale=MatchLowercase]{FiraCode-Regular}

\usepackage[backend=biber]{biblatex}
\addbibresource{refs.bib}

\usepackage{enumerate}
\usepackage{amsthm}
\usepackage{faktor}
\usepackage{physics}
\usepackage{hyperref}
\usepackage{cleveref}
\usepackage{relsize}
\usepackage{quiver}

\usepackage[shortlabels]{enumitem}
\setlist[enumerate,1]{label={(\roman*)}}
\setlist[enumerate,2]{label={(\alph*)}}

\hypersetup{
	colorlinks=true,
	% linkcolor=green!40!black,
}

\newcommand{\ttype}{\texttt{type}}
\newcommand{\mquote}[1]{\ensuremath{\text{`}#1\text{'}}}
\newcommand{\symmdiff}{\mathrel{\raisebox{1pt}{\( \mathsmaller\triangle \)}}}
\newcommand{\dom}{\operatorname{dom}}
\newcommand{\ran}{\operatorname{ran}}
\newcommand{\pr}{\symsf{pr}}

\theoremstyle{definition}
\newtheorem{theorem}{Theorem}[section]
\newtheorem{definition}[theorem]{Definition}
\newtheorem{lemma}[theorem]{Lemma}
\newtheorem{corollary}{Corollary}[theorem]

\theoremstyle{remark}
\newtheorem*{remark}{Remark}
\newtheorem*{remarks}{Remarks}

\title{New Foundations is consistent}
\author{Sky Wilshaw}
\date{July 2023}

\begin{document}

\maketitle

\begin{abstract}
	We give a self-contained account of a version of Holmes' proof \cite{holmes2023nf} that Quine's set theory \emph{New Foundations} \cite{quine-nf} is consistent.
	This is a `deformalisation' of the formal proof written in Lean at \cite{leanprover-community-con-nf}.
\end{abstract}

\emph{
	Note: At the present time, the formal proof \cite{leanprover-community-con-nf} is incomplete, and this paper reflects the unfinished state of that proof.
	We aim to keep this paper proof in line with the formal proof, although as the project is ongoing, some variance is to be expected.
}

\tableofcontents

\section{The theories at issue}

\section{Outline}

\printbibliography

\end{document}
