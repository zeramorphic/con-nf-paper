\subsection{Atoms, litters, and near-litters}

As described in \cref{ss:outline:atoms}, we have an additional level of objects below type zero.
To index the levels of the model, together with this new level, we make the following definition.
\begin{definition}
    A \emph{type index} is an element of \( \lambda \) or a distinguished symbol \( \bot \).
    We impose an order on type indices by setting \( \bot < \alpha \) for all \( \alpha \in \lambda \).
    The set of type indices is denoted \( \lambda^\bot \).
\end{definition}
Elements of \( \lambda \) may be called \emph{proper type indices}.

Our base type is a set of \emph{atoms}, organised into \emph{litters}.
\begin{definition}
    A \emph{litter} is a triple \( L = (\nu, \beta, \gamma) \) where \( \nu \in \mu \), \( \beta \) is a type index, and \( \gamma \neq \beta \) is a proper type index.
\end{definition}
This somewhat arcane definition will be used to great effect later when defining the fuzz map.
A litter \( L = (\nu, \beta, \gamma) \) encodes data coming from type \( \beta \) and going into type \( \gamma \).
Note that \( \beta \) may be \( \bot \), but \( \gamma \) may not; this corresponds to the fact that we never construct data in type \( \bot \) from data at higher levels.
The first component \( \nu \) is an index allowing us to have \( \mu \) distinct litters with the same source and target types.
\begin{remark}
    There are precisely \( \mu \) litters.
\end{remark}
\begin{definition}
    An \emph{atom} is a pair \( a = (L, i) \) where \( L \) is a litter and \( i \in \kappa \).
    The \emph{associated litter} of an atom is its first projection \( \pr_1(a) \), written \( a^\circ \) for brevity.
    The \emph{litter set} \( \LS(L) \) of a given litter \( L \) is the set of atoms whose associated litter is \( L \); that is, \( \LS(L) = \{ (L, i) \mid i \in \kappa \} \).
    The litter sets partition the set of atoms into \( \mu \) sets of \( \kappa \) atoms, and there are \( \mu \) atoms in total.
\end{definition}
\begin{remark}
    Many of our constructions rely on having only a small set of constraints.
    If our constraints take the form of atoms, the smallness assumption guarantees that most of the atoms in a given litter are unconstrained.
    Motivated by smallness concerns, we make the following definition.
\end{remark}
\begin{definition}
    A \emph{near-litter} is a pair \( N = (L, s) \) where \( L \) is a litter and \( s \) is a set of atoms with small symmetric difference to the litter set of \( L \).
    We say that the \emph{associated litter} of \( N \) is \( N^\circ = \pr_1(N) \), or that \( N \) is \emph{near} \( L \).
    For brevity, we will frequently identify a near-litter with its underlying set.
\end{definition}
\begin{remarks}\mbox{\negthinspace}
    \label{rk:mk_near_litter}
    \begin{enumerate}
        \item A set of atoms can be near at most one litter.
        \item The litter set of any litter \( L \) can be made into a near-litter: \( \NL(L) = (L, \LS(L)) \).
        \item Each (second projection of a) near-litter has size exactly \( \kappa \), and there are \( \mu \) near-litters in total; the latter follows from the fact that the cofinality of \( \mu \) is at least \( \kappa \).
    \end{enumerate}
\end{remarks}
We can now define the allowable permutations of type \( \bot \), although we will give them a different name for now; they will be precisely those permutations of atoms that respect the structure of near-litters.
\begin{definition}
    A \emph{near-litter permutation} \( \pi \) is a permutation of atoms such that for every near-litter \( N = (L, s) \), there is some (necessarily unique) litter \( L' \) such that \( (L', \pi '' s) \) is a near-litter.
\end{definition}
A near-litter permutation \( \pi \) induces a permutation of near-litters, which we will also call \( \pi \).
Moreover, a near-litter permutation \( \pi \) induces a permutation of litters (which we will again call \( \pi \)), defined by mapping \( L \) to the associated litter of \( \pi '' \LS(L) \).
Thus, a near-litter permutation is simultaneously a permutation of atoms, litters, and near-litters.

\subsection{Higher type structure}

A type-\( \alpha \) object has elements of any type \( \beta < \alpha \), which have elements of any type \( \gamma < \beta \), and so on; we must eventually reach \( \bot \) in a finite number of steps by well-foundedness.
We will now make a definition to deal with sequences of type indices obtained in this way.
\begin{definition}
    A \emph{path of type indices} \( \alpha \rightsquigarrow \varepsilon \) is a finite sequence of type indices
    \[ \alpha > \beta > \gamma > \dots > \varepsilon \]
    If \( \alpha \) is a type index, an \emph{\( \alpha \)-extended type index} is a path \( \alpha \rightsquigarrow \bot \).
    If \( A \) is a path \( \alpha \rightsquigarrow \beta \) and \( B \) is a path \( \beta \rightsquigarrow \gamma \), their composition \( A \pcomp B \) is a path \( \alpha \rightsquigarrow \gamma \) obtained by concatenation of the sequences but removing the duplicated index \( \beta \).
\end{definition}
\begin{remark}
    \label{rk:mk_extended_index}
    For any \( \alpha \in \lambda^\bot \), the set of \( \alpha \)-extended type indices has size at most \( \lambda \).
\end{remark}
In our model, the iterated extensions of objects of type \( \alpha \) are indexed by the \( \alpha \)-extended type indices.
We would like to apply operations to an object along each path \( \alpha \rightsquigarrow \bot \); this motivates the definition of \emph{structural permutations}.
\begin{definition}
    \label{def:struct_perm}
    An \emph{\( \alpha \)-structural permutation} \( \pi \) assigns a near-litter permutation to each \( \alpha \)-extended index.
    If \( A \) is an \( \alpha \)-extended index, the assigned near-litter permutation is called the \emph{derivative} of \( \pi \) along \( A \), and is denoted \( \pi_A \).
    We can extend this notion to arbitrary paths \( \alpha \rightsquigarrow \beta \); in this case, the derivative \( \pi_A \) is a \( \beta \)-structural permutation.
    The group of \( \alpha \)-structural permutations is denoted \( S_\alpha \).
\end{definition}
\begin{remark}
    Near-litter permutations may be identified with \( \bot \)-structural permutations.
\end{remark}

At proper type indices \( \alpha \), we will define the set of \( \alpha \)-allowable permutations to be a certain subgroup of the group of \( \alpha \)-structural permutations.
These permutations will be chosen in such a way that gives an action on the set of model elements at level \( \alpha \).
Not every structural permutation will have such an action.

\subsection{Addresses and supports}

We are interested in objects that can be characterised by a small amount of data; this data will take the form of \emph{addresses}.
\begin{definition}
    Let \( \alpha \) be a type index.
    An \emph{\( \alpha \)-address} is a pair \( (A, x) \) where \( A \) is an \( \alpha \)-extended type index and \( x \) is either an atom or a near-litter.
\end{definition}
An \( \alpha \)-address encodes a set of atoms (possibly a singleton), and a path to get there from a type-\( \alpha \) object by descending through iterated extensions.
\begin{note}
    In \cite{holmes2023nf}, addresses are called \emph{support conditions}, and the elements are written in the reverse order.
\end{note}
\begin{remark}
    \label{rk:mk_address}
    The set of \( \alpha \)-addresses has cardinality \( \mu \).
    This follows from \cref{rk:mk_near_litter,rk:mk_extended_index}.
\end{remark}

The group of \( \alpha \)-structural permutations acts on the set of \( \alpha \)-addresses by
\[ \pi(A, x) = (A, \pi_A(x)) \]
We would like to say that a \emph{support} is a small set of \( \alpha \)-addresses; for technical reasons that will be required for \cref{thm:foa_behaviour}, it will actually be better to make the following definition instead.
\begin{definition}
    An \( \alpha \)-support is a function \( S \) whose domain is a small ordinal and whose values are \( \alpha \)-addresses, such that for all near-litters \( N_1, N_2 \) in the range of \( S \),
    \begin{enumerate}
        \item if \( N_1^\circ = N_2^\circ \), then the atoms in their symmetric difference \( N_1 \symmdiff N_2 \) are in the range of \( S \); and
        \item if \( N_1^\circ \neq N_2^\circ \), then the atoms in their intersection \( N_1 \cap N_2 \) are in the range of \( S \).
    \end{enumerate}
    The \emph{underlying set} of a support is its range.
\end{definition}
\begin{remarks}\mbox{\negthinspace}
    \begin{enumerate}
        \item An address may occur multiple times in the range of a support, but all we usually care about is the presence or absence of an address.
        The fact that these addresses have indices in a given support will be useful for some constructions later.
        \item For any small set of addresses \( s \), there is a support \( S \) whose range is \( s \) together with the small set of atom addresses required to satisfy the symmetric difference and intersection constraints.
    \end{enumerate}
\end{remarks}
As with many parts of our construction, we will need to count precisely how many supports there are.
We will first need to establish the following lemma.
\begin{lemma}
    \label{lem:konig_converse}
    Let \( \mu \) be a strong limit cardinal.
    Then for any \( \kappa > 0 \) strictly smaller than the cofinality of \( \mu \), we have
    \[ \mu^\kappa = \mu \]
\end{lemma}
Recall that \emph{König's theorem} states that for an infinite cardinal \( \mu \), we have \( \mu^{\operatorname{cf}(\kappa)} > \mu \); this lemma is a partial converse to this theorem in the case where \( \mu \) is a strong limit.
\begin{proof}
    Let \( A \) be a set of size \( \kappa \), and let \( B \) be a set of size \( \mu \).
    We assume that \( B \) has a well-ordering, perhaps obtained from the order on \( \mu \).
    Given any function \( f : A \to B \), we define a relation \( \prec_f \) on \( A \) by the inverse image of the ordering on \( B \) under \( f \):
    \[ a \prec_f b \iff f(a) < f(b) \]
    One can show by induction on \( B \) that any for two functions \( f, g : A \to B \), if \( \im f = \im g \) and \( \prec_f = \prec_g \), then \( f = g \).
    The amount of pairs \( (\im f, \prec_f) \) is bounded by \( \mu \).
    Indeed, the amount of possible images of a function \( f : A \to B \) is bounded by \( \mu \) as \( \kappa \) is less than the cofinality of \( \mu \) and \( \mu \) is a strong limit, and the amount of relations \( \prec_f \) is bounded by \( (2^\kappa)^\kappa \), which is bounded by \( \mu \) as it is a strong limit.
    Thus, \( \mu^\kappa \leq \mu \), and the converse is clear.
\end{proof}
We can now prove that there are exactly \( \mu \) supports for any type index \( \alpha \).
\begin{proposition}
    \label{prop:mk_support}
    Let \( \alpha \) be a type index.
    The set of \( \alpha \)-supports has cardinality \( \mu \).
\end{proposition}
\begin{proof}
    A support is a function from a small ordinal to the set of \( \alpha \)-addresses, which has cardinality \( \mu \) by \cref{rk:mk_address}.
    Thus, the amount of supports is bounded by
    \[ \sum_{i < \kappa} \mu^i \]
    By \cref{lem:konig_converse}, each summand is precisely \( \mu \), and so the sum is precisely \( \kappa \cdot \mu = \mu \).
\end{proof}
The \( \alpha \)-structural permutations act on \( \alpha \)-supports pointwise:
\[ \pi(S)(i) = \pi(S(i)) \]
We will need to concatenate supports together.
The naïve concatenation of supports \( S_1, S_2 \) given by
\[ S(i) = \begin{cases}
    S_1(i) & \text{if } i \in \dom S_1 \\
    S_2(i - \dom S_1) & \text{otherwise}
\end{cases} \]
need not satisfy the extra atom conditions, so \( S \) may not be a support.
There is not a unique way to expand this concatenation into a support, so we will instead make the following definition.
\begin{definition}
    Let \( f \) be a function from a small ordinal whose values are \( \alpha \)-support conditions.
    We say that a support \( S \) is a \emph{completion} of \( f \) if \( S|_{\dom f} = f \) and every address in the extension \( S|_{\dom S \setminus \dom f} \) is an atom address \( (A, x) \) that is forced to be in \( \ran S \) by the definition of a support.
    Every such function has a completion, as the amount of atom addresses in question is small.
    A support \( S \) is a \emph{sum} of supports \( S_1, S_2 \) if it is a completion of the concatenation of \( S_1 \) and \( S_2 \).
    Every pair of supports has a sum.
\end{definition}
