\documentclass{article}
\setcounter{tocdepth}{2}

\usepackage[a4paper]{geometry}
\usepackage{parskip}
\usepackage{fontspec}
\usepackage{unicode-math}
\newcommand{\diagup}{\char"27CB}
\setmainfont[Path=fonts/,
	UprightFont=*-Regular,
	BoldFont=*-Bold,
	ItalicFont=*-Italic,
	BoldItalicFont=*-BoldItalic,
	]{STIXTwoText}
\setsansfont[Path=fonts/Source_Sans_3/static/]{SourceSans3-Regular}[Ligatures={Common,TeX}]
\setmathfont[Path=fonts/]{STIXTwoMath-Regular}
\setmathfont[Path=fonts/,range=\star]{TeXGyrePagellaMath}
\setmonofont[Path=fonts/,Scale=MatchLowercase]{FiraCode-Regular}

\usepackage{xcolor}
\definecolor{codebg}{gray}{0.95}
\definecolor{codeframe}{gray}{0.8}

\usepackage[backend=biber]{biblatex}
\addbibresource{refs.bib}

\usepackage{enumerate}
\usepackage{amsthm}
\usepackage{faktor}
\usepackage{hyperref}
\usepackage{relsize}
\usepackage{quiver}

\usepackage{cleveref}
% https://tex.stackexchange.com/a/81645
\crefformat{section}{\S#2#1#3}
\crefmultiformat{section}{\S\S#2#1#3}{ and~#2#1#3}{, #2#1#3}{, and~#2#1#3}

\usepackage[shortlabels]{enumitem}
\setlist[enumerate,1]{label={(\roman*)}}
\setlist[enumerate,2]{label={(\alph*)}}

\hypersetup{
	colorlinks=true,
	linkcolor=red!50!black,
	citecolor=green!50!black,
	urlcolor=magenta!70!black,
}

\setoperatorfont\symsf

\newcommand{\ttype}{\operatorname{type}}
\newcommand{\mquote}[1]{\ensuremath{\text{`}#1\text{'}}}
\newcommand{\symmdiff}{\mathrel{\raisebox{1pt}{\( \mathsmaller\triangle \)}}}
\newcommand{\dom}{\operatorname{dom}}
\newcommand{\ran}{\operatorname{ran}}
\newcommand{\im}{\operatorname{im}}
\newcommand{\pr}{\operatorname{pr}}
\newcommand{\LS}{\operatorname{LS}}
\newcommand{\NL}{\operatorname{NL}}
\newcommand{\supp}{\operatorname{supp}}
\newcommand{\typed}{\operatorname{typed}}
\newcommand{\pcomp}{\mathbin{\vysmblksquare}}

\theoremstyle{definition}
\newtheorem{theorem}{Theorem}[section]
\newtheorem*{theorem*}{Theorem}
\newtheorem{definition}[theorem]{Definition}
\newtheorem{lemma}[theorem]{Lemma}
\newtheorem{proposition}[theorem]{Proposition}
\newtheorem{corollary}[theorem]{Corollary}

\theoremstyle{remark}
\newtheorem{remark}[theorem]{Remark}
\newtheorem{remarks}[theorem]{Remarks}
\newtheorem{example}[theorem]{Example}
\newtheorem{examples}[theorem]{Examples}
\newtheorem{detail}[theorem]{Implementation detail}
\newtheorem{details}[theorem]{Implementation details}
\newtheorem*{note}{Note}

\title{New Foundations is consistent:\\an exposition and formal verification}
\author{Sky Wilshaw}
\date{\today}

\newcommand{\inlinecode}[1]{\fcolorbox{codeframe}{codebg}{\small\!\texttt{#1}\!}}
\newcommand{\lean}[1]{\inlinecode{#1}}
\newcommand{\llabel}[1]{\label{#1}\inlinecode{#1}}
\newcommand{\abs}[1]{\left|#1\right|}

\newcommand{\Base}{\mathsf{Base}}
\newcommand{\Str}{\mathsf{Str}}
\newcommand{\Tree}{\mathsf{Tree}}

\begin{document}

\maketitle

\begin{abstract}
	We give a self-contained account of a version of Holmes' proof \cite{con-nf} that Quine's set theory \emph{New Foundations} \cite{quine-nf} is consistent relative to the metatheory ZFC.
	We have formalised this proof in the Lean interactive theorem prover \cite{leanprover-community-con-nf}, and this paper is a `deformalisation' of that work.
	We discuss the challenges of formalising new and untested mathematics in an interactive theorem prover, and how the process of completing the formalisation has influenced our presentation of the proof.
\end{abstract}

\tableofcontents

\section{Overview}
\label{s:overview}

In \cref{s:lean}, we will briefly discuss Lean \cite{lean}, the interactive theorem prover in which our result is formalised.
We will also explain why our formalisation in \cite{leanprover-community-con-nf} can be trusted as evidence that Holmes' proof in \cite{con-nf} is correct, without needing to understand the underlying details of the proof.
We have made frequent use of the community-made repository \texttt{mathlib} \cite{mathlib2020}, which encodes standard mathematical definitions and theorems in Lean; without this, we would have needed to write our own libraries for (for example) abstract algebra and cardinal and ordinal arithmetic.

Lean is based on a version of the \emph{calculus of constructions}, which is a dependent type theory.
In order to authentically present the formalised proof, the mathematics of this paper will take place in this type theory, or some suitable variant of it.
We will rarely make note of this choice, and readers are not expected to be familiar with such type theories.
However, this will be relevant for some discussion sections, as some parts of the proof were made significantly harder by the fact that we are working in a type theory.

In \cref{s:theories}, we will establish the mathematical context for the proof we will present.
In particular, our proof will not directly show the consistency of \( \mathsf{NF} \); instead, we will construct a model of a related theory known as \emph{tangled type theory}, or \( \mathsf{TTT} \).
This is the result which has been formally verified: there is a structure that satisfies a particular axiomatisation of \( \mathsf{TTT} \) which we will discuss in \cref{s:theories}.
The expected conclusion that \( \mathsf{NF} \) is consistent then follows from the fact that \( \mathsf{NF} \) and \( \mathsf{TTT} \) are equiconsistent \cite{holmes-ttt}.

We will now outline our general strategy for the construction of a model of tangled type theory.
As we will outline in \cref{ss:theories:ttt}, \( \mathsf{TTT} \) is a many-sorted theory with types indexed by a limit ordinal \( \lambda \).
In order to impose symmetry conditions on our structure, we will add an additional level of objects below type zero.
These will not be a part of the final model we construct.
This base type will be comprised of objects called \emph{atoms} (although they are not atoms in the traditional model-theoretic sense).
Alongside the construction of the types of our model, we will also construct a group of permutations of each type, called the \emph{allowable permutations}.
Such permutations will preserve the structure of the model in a strong sense; for instance, they preserve membership.

The construction of a given type can only be done under certain hypotheses on the construction of lower types.
The most restrictive condition that we will need to satisfy is a bound on the size of each type.
In order to do this, we will need to show that there are a lot of allowable permutations.
The main technical lemma establishing this, called the \emph{freedom of action theorem}, roughly states that a partial function that locally behaves like an allowable permutation can be extended to an allowable permutation.
Much of this paper will be allocated to proving the freedom of action theorem and its various corollaries, and it will be outlined in more detail when we are in a position to prove it.

We can then finish the main induction to build the entire model out of the types we have constructed.
This step, while invisible to a human reader in set theory, takes substantial effort to formally establish in a dependent type theory.
% TODO: Add a section about it and \cref it here
It then remains to show that this is a model of \( \mathsf{TTT} \) as desired, or more precisely, a model of a particular finite axiomatisation.

[Finish the introduction\dots]

% TODO: Use outline.tex from the previous version

\section{The Lean interactive theorem prover}
\label{s:lean}
\subsection{Lean and its type theory}

Lean \cite{lean} is a functional programming language and interactive theorem prover.
As indicated in \cref{s:overview}, its underlying logic is a dependent type theory based on the calculus of constructions.
Carneiro proved in \cite{leantt} that Lean's type theory is consistent relative to
\[ \mathsf{ZFC} + \{ \text{there are } n \text{ inaccessible cardinals} \mid n < \omega \} \]
These inaccessible cardinals are needed to support Lean's hierarchy of type universes.
Higher universes are commonly used whenever they are convenient, for example in definitions of cardinals and ordinals.
However, these uses are not strictly necessary for our purposes, and the entire proof can be carried out in plain \( \mathsf{ZFC} \), as shown by \cite{holmes2023nf} and this paper.

Proofs in Lean may be written in its \emph{tactic mode}, which tracks hypotheses and goals, and enables the use of \emph{tactics} to update these hypotheses and goals according to logical rules.
There are a large variety of tactics to perform different tasks, such as simplification (\texttt{simp}), rewriting of subexpressions (\texttt{rw}), structural induction (\texttt{induction}), and so on.
These tactics output a \emph{proof term}, which is a term in Lean's underlying type theory.
The type of this term corresponds to the proposition that we intend to prove under the Curry--Howard correspondence.
The proof term is then passed to Lean's \emph{kernel}, which contains a type-checking algorithm.
If the proof term generated by a tactic has the correct type, the kernel accepts the proof.

\subsection{Trusting Lean}

% Lean's kernel is the only code that must be `trusted' in order to guarantee that a proof accepted by Lean as a whole is correct, since any unsound tactics would output a proof term that is not type-correct.

Lean is a large project, but one need only trust its kernel to ensure that accepted proofs are correct.
If a tactic were to output an incorrect proof term, then the kernel would have the opportunity to find this mistake before the proof were to be accepted.

It is important to note that the kernel has no way of knowing whether a formal definition written in Lean matches the familiar mathematical definition.
Any definitions used in a theorem statement must be manually checked by a human reader; all that Lean guarantees is that the conclusion is correct as written in its own type theory.
For example, if verifying a formalised proof of Fermat's last theorem, one should manually check the definitions of natural numbers, addition, exponentiation, and so on, but need not check (for example) definitions and results about elliptic curves.

All of the proofs in this paper (except in \cref{s:theories}, upon which no other results depend) are verified by Lean.
To help with the verification step, our main result can be found in the \texttt{Result.lean} file (\href{https://github.com/leanprover-community/con-nf/blob/main/ConNF/Model/Result.lean}{source}, \href{https://leanprover-community.github.io/con-nf/doc/ConNF/Model/Result.html}{documentation}).
Each result is tagged with a hyperlink (such as \lean{ConNF.ext}) to the documentation generated from the corresponding Lean code.


\section{The theories at issue}
\label{s:theories}

In 1937, Quine introduced \emph{New Foundations} (NF) \cite{quine-nf}, a set theory with a very small collection of axioms.
To give a proper exposition of the theory that we intend to prove consistent, we will first make a digression to introduce the related theory TST, as explained by Holmes in \cite{holmes2023nf}.
We will then describe the theory TTT, which we will use to prove our theorem.

\subsection{The simple theory of types}

The \emph{simple theory of types} (known as \emph{théorie simple des types} or TST) is a first order set theory with several sorts, indexed by the nonnegative integers.
Each sort, called a \emph{type}, is comprised of \emph{sets} of that type; each variable \( x \) has a nonnegative integer \( \ttype(\mquote x) \) which denotes the type it belongs to.
For convenience, we may write \( x^n \) to denote a variable \( x \) with type \( n \).

The primitive predicates of this theory are equality and membership.
An equality \( \mquote{x = y} \) is a well-formed formula precisely when \( \ttype(\mquote{x}) = \ttype(\mquote{y}) \), and similarly a membership formula \( \mquote{x \in y} \) is well-formed precisely when \( \ttype(\mquote{x}) + 1 = \ttype(\mquote{y}) \).

The axioms of this theory are extensionality
\[ \forall x^{n + 1}.\, \forall y^{n + 1}.\, (\forall z^n.\, z^n \in x^{n+1} \leftrightarrow z^n \in y^{n+1}) \to x^{n+1} = y^{n+1} \]
and comprehension
\[ \exists x^{n + 1}.\, \forall y^n.\, (y^n \in x^{n+1} \leftrightarrow \varphi(y^n)) \]
where \( \varphi \) is any well-formed formula, possibly with parameters.

\begin{remarks}
	\begin{enumerate}
		\item These are both axiom schemes, quantifying over all type levels \( n \), and (in the latter case) over all well-formed formulae \( \varphi \).
		\item The inhabitants of type 0, called \emph{individuals}, cannot be examined using these axioms.
		\item By comprehension, there is a set at each type that contains all sets of the previous type.
		Russell-style paradoxes are avoided as formulae of the form \( x^n \in x^n \) are ill-formed.
		% \item A theory has \emph{atoms}, or \emph{urelements}, if there are objects that have no members yet are not equal.
		% The axiom of extensionality prohibits atoms from existing in any positive type, since any two such objects would be equal.
		% Note that there is a different empty set in each type, but they are not atoms; the formula \( \varnothing^n \neq \varnothing^m \) is ill-formed for \( n \neq m \), so they cannot be called `not equal'.
	\end{enumerate}
\end{remarks}
% Replacing the extensionality axiom with
% \[ \forall x^{n + 1}.\, \forall y^{n + 1}.\, \forall w^n.\, w \in x \to ((\forall z^n.\, z^n \in x^{n+1} \leftrightarrow z^n \in y^{n+1}) \to x^{n+1} = y^{n+1}) \]
% yields a theory in which atoms in positive types are permitted.
% We will name the resulting theory TSTU, or TST with urelements.

\subsection{New Foundations}

New Foundations is a one-sorted first-order theory based on TST.
Its primitive propositions are equality and membership.
There are no well-formedness constraints on these primitive propositions.

Its axioms are precisely the axioms of TST with all type annotations erased.
That is, it has an axiom of extensionality
\[ \forall x.\, \forall y.\, (\forall z.\, z \in x \leftrightarrow z \in y) \to x = y \]
and an axiom scheme of comprehension
\[ \exists x.\, \forall y.\, (y \in x \leftrightarrow \varphi(y)) \]
the latter of which is defined for those formulae \( \varphi \) that can be obtained by erasing the type annotations of a well-formed formula of TST.
Such formulae are called \emph{stratified}.
To avoid the explicit dependence on TST, we can equivalently characterise the stratified formulae as follows.
A formula \( \varphi \) is said to be stratified when there is a function \( \sigma \) from the set of variables to the nonnegative integers, in such a way that for each subformula \( \mquote{x = y} \) of \( \varphi \) we have \( \sigma(\mquote{x}) = \sigma(\mquote{y}) \), and for each subformula \( \mquote{x \in y} \) we have \( \sigma(\mquote{x}) + 1 = \sigma(\mquote{y}) \).

\begin{remarks}
	\begin{enumerate}
		\item It is important to emphasise that while the axioms come from a many-sorted theory, NF is not one; it well-formed to ask if any set is a member of, or equal to, any other.
		\item Russell's paradox is avoided because the set \( \{ x \mid x \notin x \} \) cannot be formed; indeed, \( x \notin x \) is an unstratified formula.
		Note, however, that the set \( \{ x \mid x = x \} \) is well-formed, and so we have a universe set.
		% (TODO: add reference for Burali--Forti)
		\item The infinite set of stratified comprehension axioms can be described with a finite set; this is a result of Hailperin \cite{hailperin-finite-axiomatisation}.
		\item Specker showed in \cite{specker-choice-nf} that NF disproves the Axiom of Choice.
	\end{enumerate}
\end{remarks}

While our main result is that New Foundations is consistent, we attack the problem by means of an indirection through a third theory.

\subsection{Tangled type theory}
\label{sec:ttt}

Introduced by Holmes in \cite{holmes-ttt}, \emph{tangled type theory} (TTT) is a multi-sorted first order theory based on TST.
This theory is parametrised by a limit ordinal \( \lambda \), the elements of which will index the sorts.
As in TST, each variable \( x \) has a type that it belongs to, denoted \( \ttype(\mquote x) \).
However, in TTT, this is not a positive integer, but an element of \( \lambda \).

The primitive predicates of this theory are equality and membership.
An equality \( \mquote{x = y} \) is a well-formed formula when \( \ttype(\mquote{x}) = \ttype(\mquote{y}) \).
A membership formula \( \mquote{x \in y} \) is well-formed when \( \ttype(\mquote{x}) < \ttype(\mquote{y}) \).

The axioms of TTT are obtained by taking the axioms of TST and replacing all type indices in a consistent way with elements of \( \lambda \).
More precisely, for any order-embedding \( s : \omega \to \lambda \), we can convert a well-formed formula \( \varphi \) of TST into a well-formed formula \( \varphi^s \) of TTT by replacing a type variable \( \alpha \) with \( s(\alpha) \).

\begin{remarks}
	\begin{enumerate}
		\item Membership across types in TTT behaves in some quite bizarre ways.
		Let \( \alpha \in \lambda \), and let \( x \) be a set of type \( \alpha \).
		For any \( \beta < \alpha \), the extensionality axiom implies that \( x \) is uniquely determined by its type-\( \beta \) elements.
		However, it is simultaneously determined by its type-\( \gamma \) elements for any \( \gamma < \alpha \).
		In this way, one extension of a set controls all of the other extensions.
		\item The comprehension axiom allows a set to be built which has a specified extension in a single type.
		The elements not of this type may be considered `controlled junk'.
	\end{enumerate}
\end{remarks}

We now present the following striking theorem.

\begin{theorem}[Holmes]
	NF is consistent if and only if TTT is consistent.
\end{theorem}

The proof is not long, but is outside the scope of this paper; it requires more model theory than the rest of this paper expects a reader to be familiar with, and relies on additional results such as those proven by Specker in \cite{typical-ambiguity}.

Thus, our task of proving NF consistent is reduced to the task of proving TTT consistent.
We will do this by exhibiting an explicit model (albeit one that requires a great deal of Choice to construct).
As TTT has types indexed by a limit ordinal, and sets can only contain sets of lower type, we can construct a model by recursion over \( \lambda \).
This was not an option with NF directly, as the universe set \( \{ x \mid x = x \} \) would necessarily be constructed before many of its elements.


\section{The supertype structure}
\label{s:structure}
\subsection{Atoms, litters, and near-litters}

As described in \cref{ss:outline:atoms}, we have an additional level of objects below type zero.
To index the levels of the model, together with this new level, we make the following definition.
\begin{definition}
    A \emph{type index} is an element of \( \lambda \) or a distinguished symbol \( \bot \).
    We impose an order on type indices by setting \( \bot < \alpha \) for all \( \alpha \in \lambda \).
    The set of type indices is denoted \( \lambda^\bot \).
\end{definition}
Elements of \( \lambda \) may be called \emph{proper type indices}.

Our base type is a set of \emph{atoms}, organised into \emph{litters}.
\begin{definition}
    A \emph{litter} is a triple \( L = (\nu, \beta, \gamma) \) where \( \nu \in \mu \), \( \beta \) is a type index, and \( \gamma \neq \beta \) is a proper type index.
\end{definition}
This somewhat arcane definition will be used to great effect later when defining the fuzz map.
A litter \( L = (\nu, \beta, \gamma) \) encodes data coming from type \( \beta \) and going into type \( \gamma \).
Note that \( \beta \) may be \( \bot \), but \( \gamma \) may not; this corresponds to the fact that we never construct data in type \( \bot \) from data at higher levels.
The first component \( \nu \) is an index allowing us to have \( \mu \) distinct litters with the same source and target types.
\begin{remark}
    There are precisely \( \mu \) litters.
\end{remark}
\begin{definition}
    An \emph{atom} is a pair \( a = (L, i) \) where \( L \) is a litter and \( i \in \kappa \).
    The \emph{associated litter} of an atom is its first projection \( \pr_1(a) \), written \( a^\circ \) for brevity.
    The \emph{litter set} \( \LS(L) \) of a given litter \( L \) is the set of atoms whose associated litter is \( L \); that is, \( \LS(L) = \{ (L, i) \mid i \in \kappa \} \).
    The litter sets partition the set of atoms into \( \mu \) sets of \( \kappa \) atoms, and there are \( \mu \) atoms in total.
\end{definition}
\begin{remark}
    Many of our constructions rely on having only a small set of constraints.
    If our constraints take the form of atoms, the smallness assumption guarantees that most of the atoms in a given litter are unconstrained.
    Motivated by smallness concerns, we make the following definition.
\end{remark}
\begin{definition}
    A \emph{near-litter} is a pair \( N = (L, s) \) where \( L \) is a litter and \( s \) is a set of atoms with small symmetric difference to the litter set of \( L \).
    We say that the \emph{associated litter} of \( N \) is \( N^\circ = \pr_1(N) \), or that \( N \) is \emph{near} \( L \).
    For brevity, we will frequently identify a near-litter with its underlying set.
\end{definition}
\begin{remarks}\mbox{\negthinspace}
    \label{rk:mk_near_litter}
    \begin{enumerate}
        \item A set of atoms can be near at most one litter.
        \item The litter set of any litter \( L \) can be made into a near-litter: \( \NL(L) = (L, \LS(L)) \).
        \item Each (second projection of a) near-litter has size exactly \( \kappa \), and there are \( \mu \) near-litters in total; the latter follows from the fact that the cofinality of \( \mu \) is at least \( \kappa \).
    \end{enumerate}
\end{remarks}
We can now define the allowable permutations of type \( \bot \), although we will give them a different name for now; they will be precisely those permutations of atoms that respect the structure of near-litters.
\begin{definition}
    A \emph{near-litter permutation} \( \pi \) is a permutation of atoms such that for every near-litter \( N = (L, s) \), there is some (necessarily unique) litter \( L' \) such that \( (L', \pi '' s) \) is a near-litter.
\end{definition}
A near-litter permutation \( \pi \) induces a permutation of near-litters, which we will also call \( \pi \).
Moreover, a near-litter permutation \( \pi \) induces a permutation of litters (which we will again call \( \pi \)), defined by mapping \( L \) to the associated litter of \( \pi '' \LS(L) \).
Thus, a near-litter permutation is simultaneously a permutation of atoms, litters, and near-litters.

\subsection{Higher type structure}

A type-\( \alpha \) object has elements of any type \( \beta < \alpha \), which have elements of any type \( \gamma < \beta \), and so on; we must eventually reach \( \bot \) in a finite number of steps by well-foundedness.
We will now make a definition to deal with sequences of type indices obtained in this way.
\begin{definition}
    A \emph{path of type indices} \( \alpha \rightsquigarrow \varepsilon \) is a finite sequence of type indices
    \[ \alpha > \beta > \gamma > \dots > \varepsilon \]
    If \( \alpha \) is a type index, an \emph{\( \alpha \)-extended type index} is a path \( \alpha \rightsquigarrow \bot \).
    If \( A \) is a path \( \alpha \rightsquigarrow \beta \) and \( B \) is a path \( \beta \rightsquigarrow \gamma \), their composition \( A \pcomp B \) is a path \( \alpha \rightsquigarrow \gamma \) obtained by concatenation of the sequences but removing the duplicated index \( \beta \).
\end{definition}
\begin{remark}
    \label{rk:mk_extended_index}
    For any \( \alpha \in \lambda^\bot \), the set of \( \alpha \)-extended type indices has size at most \( \lambda \).
\end{remark}
In our model, the iterated extensions of objects of type \( \alpha \) are indexed by the \( \alpha \)-extended type indices.
We would like to apply operations to an object along each path \( \alpha \rightsquigarrow \bot \); this motivates the definition of \emph{structural permutations}.
\begin{definition}
    \label{def:struct_perm}
    An \emph{\( \alpha \)-structural permutation} \( \pi \) assigns a near-litter permutation to each \( \alpha \)-extended index.
    If \( A \) is an \( \alpha \)-extended index, the assigned near-litter permutation is called the \emph{derivative} of \( \pi \) along \( A \), and is denoted \( \pi_A \).
    We can extend this notion to arbitrary paths \( \alpha \rightsquigarrow \beta \); in this case, the derivative \( \pi_A \) is a \( \beta \)-structural permutation.
    The group of \( \alpha \)-structural permutations is denoted \( S_\alpha \).
\end{definition}
\begin{remark}
    Near-litter permutations may be identified with \( \bot \)-structural permutations.
\end{remark}

At proper type indices \( \alpha \), we will define the set of \( \alpha \)-allowable permutations to be a certain subgroup of the group of \( \alpha \)-structural permutations.
These permutations will be chosen in such a way that gives an action on the set of model elements at level \( \alpha \).
Not every structural permutation will have such an action.

\subsection{Addresses and supports}

We are interested in objects that can be characterised by a small amount of data; this data will take the form of \emph{addresses}.
\begin{definition}
    Let \( \alpha \) be a type index.
    An \emph{\( \alpha \)-address} is a pair \( (A, x) \) where \( A \) is an \( \alpha \)-extended type index and \( x \) is either an atom or a near-litter.
\end{definition}
An \( \alpha \)-address encodes a set of atoms (possibly a singleton), and a path to get there from a type-\( \alpha \) object by descending through iterated extensions.
\begin{note}
    In \cite{holmes2023nf}, addresses are called \emph{support conditions}, and the elements are written in the reverse order.
\end{note}
\begin{remark}
    \label{rk:mk_address}
    The set of \( \alpha \)-addresses has cardinality \( \mu \).
    This follows from \cref{rk:mk_near_litter,rk:mk_extended_index}.
\end{remark}

The group of \( \alpha \)-structural permutations acts on the set of \( \alpha \)-addresses by
\[ \pi(A, x) = (A, \pi_A(x)) \]
We would like to say that a \emph{support} is a small set of \( \alpha \)-addresses; for technical reasons that will be required for \cref{thm:foa_behaviour}, it will actually be better to make the following definition instead.
\begin{definition}
    An \( \alpha \)-support is a function \( S \) whose domain is a small ordinal and whose values are \( \alpha \)-addresses, such that for all near-litters \( N_1, N_2 \) in the range of \( S \),
    \begin{enumerate}
        \item if \( N_1^\circ = N_2^\circ \), then the atoms in their symmetric difference \( N_1 \symmdiff N_2 \) are in the range of \( S \); and
        \item if \( N_1^\circ \neq N_2^\circ \), then the atoms in their intersection \( N_1 \cap N_2 \) are in the range of \( S \).
    \end{enumerate}
    The \emph{underlying set} of a support is its range.
\end{definition}
\begin{remarks}\mbox{\negthinspace}
    \begin{enumerate}
        \item An address may occur multiple times in the range of a support, but all we usually care about is the presence or absence of an address.
        The fact that these addresses have indices in a given support will be useful for some constructions later.
        \item For any small set of addresses \( s \), there is a support \( S \) whose range is \( s \) together with the small set of atom addresses required to satisfy the symmetric difference and intersection constraints.
    \end{enumerate}
\end{remarks}
As with many parts of our construction, we will need to count precisely how many supports there are.
We will first need to establish the following lemma.
\begin{lemma}
    \label{lem:konig_converse}
    Let \( \mu \) be a strong limit cardinal.
    Then for any \( \kappa > 0 \) strictly smaller than the cofinality of \( \mu \), we have
    \[ \mu^\kappa = \mu \]
\end{lemma}
Recall that \emph{König's theorem} states that for an infinite cardinal \( \mu \), we have \( \mu^{\operatorname{cf}(\kappa)} > \mu \); this lemma is a partial converse to this theorem in the case where \( \mu \) is a strong limit.
\begin{proof}
    Let \( A \) be a set of size \( \kappa \), and let \( B \) be a set of size \( \mu \).
    We assume that \( B \) has a well-ordering, perhaps obtained from the order on \( \mu \).
    Given any function \( f : A \to B \), we define a relation \( \prec_f \) on \( A \) by the inverse image of the ordering on \( B \) under \( f \):
    \[ a \prec_f b \iff f(a) < f(b) \]
    One can show by induction on \( B \) that any for two functions \( f, g : A \to B \), if \( \im f = \im g \) and \( \prec_f = \prec_g \), then \( f = g \).
    The amount of pairs \( (\im f, \prec_f) \) is bounded by \( \mu \).
    Indeed, the amount of possible images of a function \( f : A \to B \) is bounded by \( \mu \) as \( \kappa \) is less than the cofinality of \( \mu \) and \( \mu \) is a strong limit, and the amount of relations \( \prec_f \) is bounded by \( (2^\kappa)^\kappa \), which is bounded by \( \mu \) as it is a strong limit.
    Thus, \( \mu^\kappa \leq \mu \), and the converse is clear.
\end{proof}
We can now prove that there are exactly \( \mu \) supports for any type index \( \alpha \).
\begin{proposition}
    \label{prop:mk_support}
    Let \( \alpha \) be a type index.
    The set of \( \alpha \)-supports has cardinality \( \mu \).
\end{proposition}
\begin{proof}
    A support is a function from a small ordinal to the set of \( \alpha \)-addresses, which has cardinality \( \mu \) by \cref{rk:mk_address}.
    Thus, the amount of supports is bounded by
    \[ \sum_{i < \kappa} \mu^i \]
    By \cref{lem:konig_converse}, each summand is precisely \( \mu \), and so the sum is precisely \( \kappa \cdot \mu = \mu \).
\end{proof}
The \( \alpha \)-structural permutations act on \( \alpha \)-supports pointwise:
\[ \pi(S)(i) = \pi(S(i)) \]
We will need to concatenate supports together.
The naïve concatenation of supports \( S_1, S_2 \) given by
\[ S(i) = \begin{cases}
    S_1(i) & \text{if } i \in \dom S_1 \\
    S_2(i - \dom S_1) & \text{otherwise}
\end{cases} \]
need not satisfy the extra atom conditions, so \( S \) may not be a support.
There is not a unique way to expand this concatenation into a support, so we will instead make the following definition.
\begin{definition}
    Let \( f \) be a function from a small ordinal whose values are \( \alpha \)-support conditions.
    We say that a support \( S \) is a \emph{completion} of \( f \) if \( S|_{\dom f} = f \) and every address in the extension \( S|_{\dom S \setminus \dom f} \) is an atom address \( (A, x) \) that is forced to be in \( \ran S \) by the definition of a support.
    Every such function has a completion, as the amount of atom addresses in question is small.
    A support \( S \) is a \emph{sum} of supports \( S_1, S_2 \) if it is a completion of the concatenation of \( S_1 \) and \( S_2 \).
    Every pair of supports has a sum.
\end{definition}


\printbibliography

\end{document}
