In this section, we describe the way in which we construct each type in the model, under the assumption that all previous levels were constructed properly.

\subsection{Hypotheses}

We explicitly describe the hypotheses required for the inductive step to succeed.
The data constructed at each level is called \emph{tangle data}; for uniformity we will also define tangle data at level \( \bot \).

\begin{definition}
    Let \( \alpha \) be a type index.
    Then \emph{tangle data} at level \( \alpha \) consists of
    \begin{enumerate}
        \item A set \( \tau_\alpha \), called the set of \emph{\( \alpha \)-tangles}.
        Tangles encapsulate our model elements.
        \item A group \( A_\alpha \), called the set of \emph{\( \alpha \)-allowable permutations}, which acts on \( \tau_\alpha \).
        Allowable permutations will be denoted using the symbol \( \rho \).
        \item A group homomorphism \( A_\alpha \to S_\alpha \), where \( S_\alpha \) is the group of \( \alpha \)-structural permutations as defined in \cref{def:struct_perm}.
        We will notationally suppress this homomorphism, and treat \( A_\alpha \) as a subgroup of \( S_\alpha \), so allowable permutations act on anything that structural permutations do.
        \item A function \( \supp \) that assigns an \( \alpha \)-support to each \( \alpha \)-tangle.
        This support must actually support the input tangle: for any \( \alpha \)-allowable permutation \( \rho \), if \( \rho \) fixes \( \supp t \) pointwise, it fixes \( t \).
        \[ (\forall i.\, \rho((\supp t)(i)) = (\supp t)(i)) \to \rho(t) = t \]
        We require that \( \supp \) commutes with \( \alpha \)-allowable permutations \( \rho \) in the sense that
        \[ \supp(\rho(t)) = \rho(\supp t) \]
    \end{enumerate}
\end{definition}

The tangle data at level \( \bot \) is constructed by taking the tangles to be the atoms, the allowable permutations to be the near-litter permutations, and the support function to be given by
\[ \dom(\supp a) = 1;\quad (\supp a)(0) = (\varnothing, a) \]
where \( \varnothing \) is the empty path \( \bot \rightsquigarrow \bot \).

At higher levels \( \alpha \in \lambda \), the tangles will be pairs \( (x, S) \) where \( x \) is a model element of type \( \alpha \) and \( S \) is a support that supports \( x \) under the action of \( \alpha \)-allowable permutations.
There are \( \mu \) tangles that encapsulate any given model element.
Allowable permutations act on these pairs pointwise.
\[ \rho((x, S)) = (\rho(x), \rho(S)) \]
It is clear to see that the second projection \( \pr_2 \) satisfies the requirements of \( \supp \) defined in (iv) above are satisfied.

\begin{definition}
    Let \( \alpha \) be a type index with tangle data.
    Then a \emph{position function} at level \( \alpha \) is an injection \( \iota_\alpha : \tau_\alpha \to \mu \).
    For convenience, we will simply write \( \iota \) for all position functions.
\end{definition}

At level \( \bot \), the position function is chosen arbitrarily; one exists as there are exactly \( \mu \) atoms.
At higher levels \( \alpha \in \lambda \), the existence of a position function depends on the fact that \( \tau_\alpha \) has size at most \( \mu \).
We will typically assume that all type levels previously constructed have a position function.

Let \( \alpha \in \lambda \), and let \( N \) be a near-litter.
We will construct our model in such a way that there is an element \( x \) of type \( \alpha \) with a \( \bot \)-extension that is exactly (the atoms inside) \( N \).
This will imply, in particular, that each type must have size at least \( \mu \) by \cref{rk:mk_near_litter}.
Together with the existence of a position function, the size of each type can be seen to be exactly \( \mu \).
We cannot express the notion of a \( \bot \)-extension directly with the language available to us, but we do not need it in order to capture the information relevant for this section.

\begin{definition}
    Let \( \alpha \) be a proper type index with tangle data.
    We say that \( \alpha \) has \emph{typed near-litters} if there is an injection \( \typed_\alpha \) from the set of near-litters to \( \tau_\alpha \), commuting with allowable permutations in the following sense.
    \[ \rho(\typed_\alpha N) = \typed_\alpha (\rho_{(\alpha > \bot)}(N)) \]
\end{definition}

\subsection{The fuzz map}

We will now perform a construction that underpins the rest of the model, and helps us enforce extensionality.
In tangled type theory, a given model element has extensions at each level below it.
In our model, each object has a `preferred' extension, and must find a way to compute the other extensions from that information.
In order to do this, we need to be able to convert arbitrary model elements into `junk' at other levels, which can then be interpreted as a `non-preferred' extension.

The \emph{fuzz maps} perform this task.
They are parametrised by type indices \( \beta \neq \gamma \) where \( \gamma \neq \bot \), representing the source type level and the target type level.
For each such pair, the fuzz map is an injection from \( \beta \)-tangles to litters.
An arbitrary litter can only be the image of a fuzz map defined at a single pair of type levels.

Treating the output of the fuzz map as a typed near-litter, its position is always greater than the position of the input to the function.
A similar property holds for atoms.
This ensures a well-foundedness condition that we can use in many places later.

\begin{definition}
    \label{def:fuzz}
    Let \( \beta \) be a type index, and let \( \gamma \neq \beta \) be a proper type index.
    Suppose that \( \beta, \gamma \) have tangle data and a position function, and that \( \gamma \) has typed near-litters.
    A \emph{fuzz map} \( f_{\beta,\gamma} \) is an injection from \( \tau_\beta \) to the set of litters, such that
    \begin{enumerate}
        \item For any \( t \in \tau_\beta \), \( f_{\beta,\gamma}(t) = (\nu, \beta, \gamma) \) for some \( \nu \in \mu \), so all of the fuzz maps have pairwise disjoint ranges.
        \item For any \( t \in \tau_\beta \) and any near-litter \( N \) near \( f_{\beta,\gamma}(N) \),
        \[ \iota_\beta(t) < \iota_\gamma(\typed_\gamma N) \]
        \item For any \( t \in \tau_\beta \) and atom \( a \) in the litter set corresponding to \( f_{\beta,\gamma}(t) \),
        \[ \iota_\beta(t) < \iota_\bot(a) \]
    \end{enumerate}
\end{definition}

We will now show that such a function exists, and will thenceforth refer to it as \emph{the} fuzz map.

\begin{proposition}
    Let \( \beta \) be a type index, and let \( \gamma \neq \beta \) be a proper type index.
    Suppose that \( \beta, \gamma \) have tangle data and a position function, and that \( \gamma \) has typed near-litters.
    Then a fuzz map \( f_{\beta,\gamma} \) exists.
\end{proposition}
\begin{proof}
    We construct the map by recursion along \( \tau_\beta \), with the order induced by \( \iota_\beta \).
    Suppose we have already constructed \( f_{\beta,\gamma} \) for \( s \in \tau_\beta \) with \( \iota(s) < \iota(t) \).
    We must choose an index \( \nu \in \mu \) subject to three constraints:
    % TODO: check phrasing
    \begin{enumerate}
        \item \( \nu \) was not picked for any previous \( s \);
        \item \( \iota(t) \) is greater than \( \iota_\gamma(\typed_\gamma N) \) for any near-litter near \( (\nu, \beta, \gamma) \);
        \item \( \iota(t) \) is greater than \( \iota_\bot(a) \) for any atom \( a \) with \( a^\circ = (\nu, \beta, \gamma) \).
    \end{enumerate}
    Each constraint denies us less than \( \mu \) choices, and so there is always an available index \( \nu \) to choose.
\end{proof}

\subsection{Codes and clouds}

From now, we will assume that \( \alpha \in \lambda \) is a fixed proper type index at which we intend to construct tangle data.
In this section, all other (proper) type indices will be assumed to be strictly less than \( \alpha \).
We assume that we have tangle data, position functions, and typed near-litters for all type indices below \( \alpha \).

We will now begin the process of stitching together the fuzz maps to form the function that deduces an alternative extension from a preferred extension.

\begin{definition}
    Let \( \beta < \alpha \) be a type index, and let \( \gamma < \alpha \) be a proper type index not equal to \( \beta \).
    The \emph{cloud map} \( c_{\beta,\gamma} : \mathcal P(\tau_\beta) \to \mathcal P(\tau_\gamma) \) is given by
    \[ c_{\beta,\gamma}(s) = {\typed_\gamma} '' \bigcup_{t \in s} \{ N \mid N^\circ = f_{\beta,\gamma}(t) \} \]
\end{definition}

For each \( t \in s \), the map \( c_{\beta,\gamma} \) produces the `cloud' of near-litters near to \( f_{\beta,\gamma}(t) \), and turns them into tangles through the typed near-litter map.
This will be used to construct the alternative extension map.

\begin{note}
    In \cite{holmes2023nf}, the cloud map is called \( A_{\beta,\gamma} \), but in this paper the notation \( A_{\beta,\gamma} \) is reserved for the alternative extension map.
\end{note}

\begin{remark}
    The cloud map is injective, and further, if \( c_{\beta,\gamma}(s) = c_{\beta',\gamma}(s') \) and \( s \) is nonempty, then \( \beta = \beta' \).
    In particular, a nonempty set of \( \gamma \)-tangles has at most one inverse image under any cloud map \( c_{\beta,\gamma} \).
    We will show that there are only finitely many iterated images under an inverse cloud map, and we will use this to define preferred extensions.
    To prove this, we will make the following definition that encapsulates the notion of a particular choice of \( \beta < \alpha \) and an extension at that level.
\end{remark}

\begin{definition}
    Let \( \alpha \in \lambda \).
    An \emph{\( \alpha \)-code} is a pair \( (\beta, s) \) where \( \beta < \alpha \) is a type index, and \( s \subseteq \tau_\beta \).
\end{definition}

\begin{definition}
    Let \( \beta < \alpha \) be a proper type index.
    The \emph{code cloud map} \( C_\beta \) is a map from \( \alpha \)-codes to \( \alpha \)-codes, given by
    \[ C_\beta((\gamma, s)) = \begin{cases}
        (\beta, s) & \text{if } \gamma = \beta \\
        (\beta, c_{\gamma,\beta}(s)) & \text{if } \gamma \neq \beta
    \end{cases} \]
    That is, \( C_\beta \) applies the relevant cloud map to obtain a code describing a \( \beta \)-extension, or does nothing if the first projection of the code is already \( \beta \).
\end{definition}

Observe that \( C_\beta \) is injective on codes with first projection not equal to \( \beta \).
We will show that there are only finitely many iterated images under the inverse of \( C_\beta \).

\begin{definition}
    Define a relation \( \rightcurvedarrow \) on \( \alpha \)-codes by letting \( x \rightcurvedarrow C_\beta(x) \) whenever \( \pr_1(x) \neq \beta \).
\end{definition}
\begin{remark}
    A code \( x \) has at most one predecessor under \( \rightcurvedarrow \).
    If \( x \rightcurvedarrow y \), then \( x \) is empty (more precisely, its second projection is empty) if and only if \( y \) is.
    Moreover, all empty codes are related to each other under \( \rightcurvedarrow \).
\end{remark}

\begin{lemma}
    \label{lem:code_wf}
    The relation \( \rightcurvedarrow \) is well-founded on nonempty codes.
\end{lemma}
\begin{proof}
    For each nonempty \( \alpha \)-code \( x \), we define an ordinal \( \iota(x) \in \mu \) given by the smallest position of any tangle in \( \pr_2(x) \).
    \[ \iota((\beta, s)) = \min_{t \in s} \iota_\beta(t) \]
    We show that if \( x \rightcurvedarrow y \), then \( \iota(x) < \iota(y) \).
    Let \( x = (\beta, s) \) and \( y = (\gamma, c_{\beta,\gamma}(s)) \).
    Suppose \( t \in \tau_\gamma \) is the tangle with smallest position in \( c_{\beta,\gamma}(s) \).
    Then \( t = \typed_\gamma N \) where \( N = f_{\beta,\gamma}(t') \) for some \( t' \in s \).
    It suffices to show \( \iota_\beta(t') < \iota_\gamma(t) \), but this holds by the construction of the fuzz map in \cref{def:fuzz}.
\end{proof}

An object in our model will correspond to a collection of codes, representing its extensions.
A given object at level \( \alpha \) has exactly one extension at every proper type index \( \beta < \alpha \), and may optionally have a \( \bot \)-extension.

By \cref{lem:code_wf}, the set of nonempty \( \alpha \)-codes forms a tree.
Each code has a finite height; a code with no \( \rightcurvedarrow \)-predecessor is of height zero, and the height of any other code can be computed recursively.
We will partition the tree of nonempty \( \alpha \)-codes into smaller trees of height 1, and use those as equivalence classes of codes representing a single model element.

\begin{definition}
    We define an equivalence relation \( \equiv \) on nonempty \( \alpha \)-codes, generated by the assertion that \( x \equiv y \) whenever \( x \rightcurvedarrow y \) and \( x \) has even height.
\end{definition}

\begin{remarks}
    \begin{enumerate}
        \item If two codes \( x, y \) are equivalent, then either \( x = y \), or \( x \) has even height and \( x \rightcurvedarrow y \) (or vice versa), or both \( x, y \) have odd height and there is some \( z \) with \( z \rightcurvedarrow x, y \).
        \item Each equivalence class contains precisely one even code.
        This will be the `preferred' extension of the corresponding model element, as outlined in \cref{ss:outline:construction}.
        \item Each equivalence class contains exactly one code for each proper type index \( \beta < \alpha \).
        If the equivalence class contains a code at level \( \bot \), it must be the even code, because such codes can never be produced by the code cloud map.
    \end{enumerate}
\end{remarks}

\subsection{Model elements}

We now collate equivalence classes of codes to create our model elements.

\begin{definition}
    An \emph{\( \alpha \)-preobject} \( x \) assigns to each proper type index \( \beta < \alpha \) a set of \( \beta \)-tangles \( x_\beta \), in such a way that either
    \begin{enumerate}
        \item there is a set \( s \) of atoms such that \( x_\beta = c_{\bot,\beta}(s) \) for all \( \beta < \alpha \); or
        \item there is a proper type index \( \beta < \alpha \) such that \( (\beta, x_\beta) \) is an even code and \( x_\gamma = c_{\beta,\gamma}(x_\beta) \) for all \( \gamma \neq \beta \).
    \end{enumerate}
    Thus nonempty preobjects correspond to equivalence classes of nonempty codes, and there is precisely one empty \( \alpha \)-preobject which corresponds to all of the empty codes.
    An even representative code can be easily extracted from any preobject.
\end{definition}

These preobjects satisfy the required form of extensionality in tangled type theory.

\begin{theorem}
    Let \( x, y \) be \( \alpha \)-preobjects, and let \( \beta < \alpha \) be a proper type index such that \( x_\beta = y_\beta \).
    Then \( x = y \).
\end{theorem}
Note that applying metatheoretic extensionality to \( x_\beta \) and \( y_\beta \) strengthens this assertion into the proper form.
\begin{proof}
    First, note that if \( x \) and \( y \) have the same representative (even) code, then they are equal.
    This follows from the fact that the extensions of a preobject can be calculated by appling the code cloud map to the representative code.
    We have three cases.
    \begin{enumerate}
        \item \emph{The extensions of \( x \) and \( y \) can both be calculated from a set of atoms.}
        Let \( s_x, s_y \) be the set of atoms for \( x, y \) respectively.
        Then \( c_{\bot,\beta}(s_x) = x_\beta = y_\beta = c_{\bot,\beta}(s_y) \), hence \( s_x = s_y \), so \( x \) and \( y \) have the same representative code.
        \item \emph{The extensions of \( x \) can be calculated from a set of atoms \( s \), and the extensions of \( y \) can be calculated from its \( \gamma \)-extension.}
        We must show that \( (\bot, s) \equiv (\gamma, y_\gamma) \).
        Suppose \( \beta = \gamma \), so \( c_{\bot,\beta}(s) = x_\beta = y_\beta \).
        In this case, \( (\bot, s) \equiv (\beta, x_\beta) \) by assumption, giving the result.
        Instead, suppose \( \beta \neq \gamma \).
        In this case,
        \[ (\gamma, y_\gamma) \equiv (\beta, c_{\gamma,\beta}(y_\gamma)) = (\beta, y_\beta) = (\beta, x_\beta) \equiv (\bot, s_x) \]
        The case where \( x \) and \( y \) are swapped holds by symmetry.
        \item \emph{The extensions of \( x \) can be calculated from its \( \gamma \)-extension, and the extensions of \( y \) can be calculated from its \( \delta \)-extension.}
        We must show \( (\gamma, x_\gamma) \equiv (\delta, y_\delta) \).
        We have
        \[ (\gamma, x_\gamma) \equiv (\beta, x_\beta) = (\beta, y_\beta) \equiv (\delta, y_\delta) \]
    \end{enumerate}
\end{proof}

We want to define the \( \alpha \)-objects to be those \( \alpha \)-preobjects with a small support under the action of \( \alpha \)-allowable permutations, so we must first define this group.
The allowable permutations will be built up from allowable permutations at lower levels.
These will become the derivatives along the one-step paths \( \alpha > \beta \).
These permutations are constrained so as to commute with the fuzz map.

\begin{definition}
    An \emph{\( \alpha \)-allowable permutation} \( \rho \) assigns to each type index \( \beta < \alpha \) a \( \beta \)-allowable permutation \( \rho_{(\alpha > \beta)} \), in such a way that
    \begin{equation}
        (\rho_{(\alpha > \gamma)})_{(\gamma > \bot)}(f_{\beta,\gamma}(t)) = f_{\beta,\gamma}(\rho_{(\alpha > \beta)})
        \tag{\( \ast \)}
    \end{equation}
    for all type indices \( \beta \), all proper type indices \( \gamma \neq \beta \), and all \( \beta \)-tangles \( t \).
    The homomorphism to the group of \( \alpha \)-structural permutations is given by mapping each of the one-step derivatives \( \rho_{(\alpha > \beta)} \) to structural permutations and collating the results.
\end{definition}
\begin{remark}
    \begin{enumerate}
        \item Note that the appearance of \( \beta \)-allowable permutations in this definition is not circular; these are assumed to exist in the tangle data assigned to level \( \beta \).
        We will later stipulate that the allowable permutations obtained through tangle data at lower levels already satisfy this condition.
        \item A small calculation reveals that condition (\( \ast \)) enforces, for any allowable \( \rho \), that for any set \( s \subseteq \tau_\beta \) and \( \gamma \neq \beta \) proper, \[ {\rho_{(\alpha > \gamma)}} '' c_{\beta,\gamma}(s) = c_{\beta,\gamma}({\rho_{(\alpha > \beta)}} '' s) \]
        Hence, for any code \( x \), we have \( \rho(C_\beta(x)) = C_\beta(\rho(x)) \), and so allowable permutations preserve code equivalence.
        This suggests that they are a sensible choice for permutations that respect the structure of \( \alpha \)-preobjects.
    \end{enumerate}
\end{remark}

We are now in a position to define our model elements.

\begin{definition}
    An \emph{\( \alpha \)-object} is an \( \alpha \)-preobject that admits an \( \alpha \)-support that supports it under the action of \( \alpha \)-allowable permutations.
    That is, \( x \) is an object if there is some support \( S \) such that \( \rho(S) = S \) implies \( \rho(x) = x \).
    An \emph{\( \alpha \)-tangle} is a pair \( (x, S) \) where \( x \) is an \( \alpha \)-object and \( S \) is an \( \alpha \)-support that supports \( x \) under the action of allowable permutations.
\end{definition}

We have thus defined tangle data at level \( \alpha \).
We can also immediately define typed near-litters at level \( \alpha \), by considering the preobject \( x \) given by the even code \( (\bot, N) \).
This evidently has a small support \( S \) given by
\[ \dom S = 1;\quad S(0) = ((\alpha > \bot), N) \]
and thus this is an \( \alpha \)-object.
The map \( \typed_\alpha \) thus sends \( N \) to the tangle \( (x, S) \).

\subsection{Comments}

We have constructed tangle data and typed near-litters at level \( \alpha \), under the assumption that they exist (along with position functions) at all lower type levels.

The use of a group of permutations is known as a \emph{Fraenkel--Mostowski construction}.
Permutations of atoms in a model of a set theory with atoms give rise to permutations of the entire structure.
Typically, such constructions involve a normal filter of subgroups of such permutations; this role is played by our allowable permutations.

Recall that a tangle is a model elements bundled together with a support.
The main benefit that tangles provide is that the support function \( \supp \) commutes with allowable permutations.
In other Fraenkel--Mostowski constructions, objects typically have a \emph{minimal support}, and these commute with allowable permutations; by bundling supports, we retain this nice behaviour without requiring our objects to have minimal supports (and in fact they do not).
Note that tangles themselves do have minimal supports, namely, the support given by \( \supp \).
Picturing tangles instead of model elements as the primitive notion, we lose extensionality but gain some some symmetry that we will use to great effect later.
The idea to bundle tangles in this way was not part of the original proof, and is a new contribution by this author.

We will later impose a restriction on objects that if \( (x, S) \) is an element in some extension, then \( (x, T) \) is also an element for any other support \( T \).
% TODO: Can this ever violate the small support condition?
We can then collapse the entire model by taking the first projection of every tangle, reducing it to a model of tangled type theory satisfying extensionality as desired.

The condition that \( \supp(\rho(t)) = \rho(\supp t) \) was not used in this section.
In particular, this entire construction succeeds by replacing the lower-type tangles in the tangle data with model elements themselves.
This method was used in an earlier version of the proof before shifting to the new `bundled' approach.
