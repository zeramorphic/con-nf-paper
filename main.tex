\documentclass{article}
\setcounter{tocdepth}{2}

\usepackage[a4paper]{geometry}
\usepackage{parskip}
\usepackage{fontspec}
\usepackage{unicode-math}
\newcommand{\diagup}{\char"27CB}
\setmainfont[Path=fonts/,
	UprightFont=*-Regular,
	BoldFont=*-Bold,
	ItalicFont=*-Italic,
	BoldItalicFont=*-BoldItalic,
	]{STIXTwoText}
\setmathfont[Path=fonts/]{STIXTwoMath-Regular}
\setmathfont[Path=fonts/,range=\star]{TeXGyrePagellaMath}
\setmonofont[Path=fonts/,Scale=MatchLowercase]{FiraCode-Regular}

\usepackage[backend=biber]{biblatex}
\addbibresource{refs.bib}

\usepackage{enumerate}
\usepackage{amsthm}
\usepackage{faktor}
\usepackage{hyperref}
\usepackage{relsize}
\usepackage{quiver}

\usepackage{cleveref}
% https://tex.stackexchange.com/a/81645
\crefformat{section}{\S#2#1#3}
\crefmultiformat{section}{\S\S#2#1#3}{ and~#2#1#3}{, #2#1#3}{, and~#2#1#3}

\usepackage[shortlabels]{enumitem}
\setlist[enumerate,1]{label={(\roman*)}}
\setlist[enumerate,2]{label={(\alph*)}}

\hypersetup{
	colorlinks=true,
	linkcolor=red!50!black,
	citecolor=green!50!black,
	urlcolor=magenta!70!black,
}

\setoperatorfont\symsf

\newcommand{\ttype}{\texttt{type}}
\newcommand{\mquote}[1]{\ensuremath{\text{`}#1\text{'}}}
\newcommand{\symmdiff}{\mathrel{\raisebox{1pt}{\( \mathsmaller\triangle \)}}}
\newcommand{\dom}{\operatorname{dom}}
\newcommand{\ran}{\operatorname{ran}}
\newcommand{\im}{\operatorname{im}}
\newcommand{\pr}{\operatorname{pr}}
\newcommand{\LS}{\operatorname{LS}}
\newcommand{\NL}{\operatorname{NL}}
\newcommand{\supp}{\operatorname{supp}}
\newcommand{\typed}{\operatorname{typed}}
\newcommand{\pcomp}{\mathbin{\vysmblksquare}}

\theoremstyle{definition}
\newtheorem{theorem}{Theorem}[section]
\newtheorem*{theorem*}{Theorem}
\newtheorem{definition}[theorem]{Definition}
\newtheorem{lemma}[theorem]{Lemma}
\newtheorem{proposition}[theorem]{Proposition}
\newtheorem{corollary}[theorem]{Corollary}

\theoremstyle{remark}
\newtheorem{remark}[theorem]{Remark}
\newtheorem{remarks}[theorem]{Remarks}
\newtheorem*{note}{Note}

\title{New Foundations is consistent}
\author{Sky Wilshaw}
\date{January 2024}

\begin{document}

\maketitle

\begin{abstract}
	We give a self-contained account of a version of Holmes' proof \cite{holmes2023nf} that Quine's set theory \emph{New Foundations} \cite{quine-nf} is consistent relative to the metatheory ZFC.
	This is a `deformalisation' of the formal proof written in Lean at \cite{leanprover-community-con-nf}.
\end{abstract}

\tableofcontents

\section{Overview}

In \cref{sec:theories}, we outline the context for the proof we will present.
The mathematical background expected in subsequent sections will be limited to basic familiarity with cardinals and ordinals.
We will then give an outline of the proof in \cref{sec:outline}.
In \cref{sec:structure}, we introduce some basic preliminaries, and explicitly describe the structure within which our model will reside.
The objects of our model will be constructed in \cref{sec:construction}.
We then show the main technical theorem, the \emph{freedom of action theorem}, in \cref{sec:foa}.

All proofs given in \cref{sec:structure,sec:construction,sec:foa} are verified by Lean \cite{lean}, which is a functional programing language and interactive theorem prover.
Most of our proofs are written in Lean's \emph{tactic mode}, in which Lean tracks a number of hypotheses and goals, and we instruct it via tactics to update these hypotheses and goals according to logical rules.
There are a large variety of tactics to perform different tasks, such as simplification (\texttt{simp}), rewriting of subexpressions (\texttt{rw}), case splitting (\texttt{cases} and \texttt{by\_cases}), and so on.
These tactics output a \emph{proof term}, which is then passed to Lean's \emph{kernel}.
The kernel is a small program that verifies proofs by performing a type-checking algorithm.
Lean's kernel is the only code that must be `trusted' in order to guarantee that a proof accepted by Lean as a whole is correct, since any unsound tactics would output a proof term that is not type-correct.

We made frequent use of the community-made repository \texttt{mathlib} \cite{mathlib2020}, which encodes standard mathematical definitions and theorems in Lean; without this, we would have needed to write our own libraries for abstract algebra and cardinal and ordinal arithmetic.

It is important to note that although theorems can be verified by Lean's kernel, it has no way of knowing whether a formal definition written in Lean matches the familiar mathematical definition; any definitions used in a theorem statement must be manually checked.
For example, if verifying a proof of Fermat's last theorem, it would suffice to manually check the definitions of natural numbers, addition, exponentiation, and natural equalities and inequalities.

Carneiro proved in \cite{leantt} that Lean's type theory is consistent relative to
\[ \text{ZFC} + \{ \text{there are } n \text{ inaccessible cardinals} \mid n < \omega \} \]
These inaccessible cardinals are needed to support Lean's hierarchy of type universes.
Higher universes are used by \texttt{mathlib} whenever they are convenient, for example in its definitions of cardinals and ordinals.
However, these uses are not strictly necessary for our purposes, and the entire proof can be carried out in plain ZFC, as shown by \cite{holmes2023nf} and this paper.

\emph{%
	Note: At the present time, the formal proof \cite{leanprover-community-con-nf} is incomplete, and this paper reflects the unfinished state of that proof.
	We aim to keep this paper proof in line with the formal proof, although as the project is ongoing, some variance is to be expected.
	The current version of the paper is available at \url{https://zeramorphic.github.io/con-nf-paper/main.pdf}.
}

\section{The theories at issue}
\label{sec:theories}
In 1937, Quine introduced \emph{New Foundations} (\( \mathsf{NF} \)) \cite{quine-nf}, a set theory with a very small collection of axioms.
To give a proper exposition of the theory that we intend to prove consistent, we will first make a digression to introduce the related theory \( \mathsf{TST} \), as explained by Holmes in \cite{holmes2023nf}.
We will then describe the theory \( \mathsf{TTT} \), which we will use to prove our theorem.

\subsection{The simple theory of types}

The \emph{simple theory of types} (known as \emph{théorie simple des types} or \( \mathsf{TST} \)) is a first order set theory with several sorts, indexed by the nonnegative integers.
Each sort, called a \emph{type}, is comprised of \emph{sets} of that type; each variable \( x \) has a nonnegative integer \( \ttype(\mquote x) \) which denotes the type it belongs to.
For convenience, we may write \( x^n \) to denote a variable \( x \) with type \( n \).

The primitive predicates of this theory are equality and membership.
An equality \( \mquote{x = y} \) is a well-formed formula precisely when \( \ttype(\mquote{x}) = \ttype(\mquote{y}) \), and similarly a membership formula \( \mquote{x \in y} \) is well-formed precisely when \( \ttype(\mquote{x}) + 1 = \ttype(\mquote{y}) \).

The axioms of this theory are extensionality
\[ \forall x^{n + 1}.\, \forall y^{n + 1}.\, (\forall z^n.\, z^n \in x^{n+1} \leftrightarrow z^n \in y^{n+1}) \to x^{n+1} = y^{n+1} \]
and comprehension
\[ \exists x^{n + 1}.\, \forall y^n.\, (y^n \in x^{n+1} \leftrightarrow \varphi(y^n)) \]
where \( \varphi \) is any well-formed formula, possibly with parameters.

\begin{remarks}\mbox{\negthinspace}
	\begin{enumerate}
		\item These are both axiom schemes, instantiated for all type levels \( n \), and (in the latter case) for all well-formed formulae \( \varphi \).
		\item The inhabitants of type 0, called \emph{individuals}, cannot be examined using these axioms.
		\item By comprehension, there is a set at each nonzero type that contains all sets of the previous type.
		Russell-style paradoxes are avoided as formulae of the form \( x^n \in x^n \) are ill-formed.
		% \item A theory has \emph{atoms}, or \emph{urelements}, if there are objects that have no members yet are not equal.
		% The axiom of extensionality prohibits atoms from existing in any positive type, since any two such objects would be equal.
		% Note that there is a different empty set in each type, but they are not atoms; the formula \( \varnothing^n \neq \varnothing^m \) is ill-formed for \( n \neq m \), so they cannot be called `not equal'.
	\end{enumerate}
\end{remarks}
% Replacing the extensionality axiom with
% \[ \forall x^{n + 1}.\, \forall y^{n + 1}.\, \forall w^n.\, w \in x \to ((\forall z^n.\, z^n \in x^{n+1} \leftrightarrow z^n \in y^{n+1}) \to x^{n+1} = y^{n+1}) \]
% yields a theory in which atoms in positive types are permitted.
% We will name the resulting theory \( \mathsf{TST} \)U, or \( \mathsf{TST} \) with urelements.

\subsection{New Foundations}

New Foundations is a one-sorted first-order theory based on \( \mathsf{TST} \).
Its primitive propositions are equality and membership.
There are no well-formedness constraints on these primitive propositions.

Its axioms are precisely the axioms of \( \mathsf{TST} \) with all type annotations erased.
That is, it has an axiom of extensionality
\[ \forall x.\, \forall y.\, (\forall z.\, z \in x \leftrightarrow z \in y) \to x = y \]
and an axiom scheme of comprehension
\[ \exists x.\, \forall y.\, (y \in x \leftrightarrow \varphi(y)) \]
the latter of which is defined for those formulae \( \varphi \) that can be obtained by erasing the type annotations of a well-formed formula of \( \mathsf{TST} \).
Such formulae are called \emph{stratified}.
To avoid the explicit dependence on \( \mathsf{TST} \), we can equivalently characterise the stratified formulae as follows.
A formula \( \varphi \) is said to be stratified when there is a function \( \sigma \) from the set of variables to the nonnegative integers, in such a way that for each subformula \( \mquote{x = y} \) of \( \varphi \) we have \( \sigma(\mquote{x}) = \sigma(\mquote{y}) \), and for each subformula \( \mquote{x \in y} \) we have \( \sigma(\mquote{x}) + 1 = \sigma(\mquote{y}) \).

\begin{remarks}\mbox{\negthinspace}
	\begin{enumerate}
		\item It is important to emphasise that while the axioms come from a many-sorted theory, \( \mathsf{NF} \) is not one; it is well-formed to ask if any set is a member of, or equal to, any other.
		\item Russell's paradox is avoided because the set \( \{ x \mid x \notin x \} \) cannot be formed; indeed, \( x \notin x \) is an unstratified formula.
		Note, however, that the set \( \{ x \mid x = x \} \) is well-formed, and so we have a universe set.
		% (TODO: add reference for Burali--Forti and Cantor)
		% \item The infinite set of stratified comprehension axioms can be captured by a finite subset; this is a result of Hailperin \cite{hailperin-finite-axiomatisation}.
		\item Specker showed in \cite{specker-choice-nf} that \( \mathsf{NF} \) disproves the Axiom of Choice.
	\end{enumerate}
\end{remarks}

While our main result is that New Foundations is consistent, we attack the problem by means of an indirection through a third theory.

\subsection{Tangled type theory}
\label{ss:theories:ttt}

Introduced by Holmes in \cite{holmes-ttt}, \emph{tangled type theory} (\( \mathsf{TTT} \)) is a multi-sorted first order theory based on \( \mathsf{TST} \).
This theory is parametrised by a limit ordinal \( \lambda \), the elements of which will index the sorts; \( \omega \) works, but we prefer generality.
As in \( \mathsf{TST} \), each variable \( x \) has a type that it belongs to, denoted \( \ttype(\mquote x) \).
However, in \( \mathsf{TTT} \), this is not a positive integer, but an element of \( \lambda \).

The primitive predicates of this theory are equality and membership.
An equality \( \mquote{x = y} \) is a well-formed formula when \( \ttype(\mquote{x}) = \ttype(\mquote{y}) \).
A membership formula \( \mquote{x \in y} \) is well-formed when \( \ttype(\mquote{x}) < \ttype(\mquote{y}) \).

The axioms of \( \mathsf{TTT} \) are obtained by taking the axioms of \( \mathsf{TST} \) and replacing all type indices in a consistent way with elements of \( \lambda \).
More precisely, for any order-embedding \( s : \omega \to \lambda \), we can convert a well-formed formula \( \varphi \) of \( \mathsf{TST} \) into a well-formed formula \( \varphi^s \) of \( \mathsf{TTT} \) by replacing each type variable \( \alpha \) with \( s(\alpha) \).

\begin{remarks}\mbox{\negthinspace}
	\begin{enumerate}
		\item Membership across types in \( \mathsf{TTT} \) behaves in some quite bizarre ways.
		Let \( \alpha \in \lambda \), and let \( x \) be a set of type \( \alpha \).
		For any \( \beta < \alpha \), the extensionality axiom implies that \( x \) is uniquely determined by its type-\( \beta \) elements.
		However, it is simultaneously determined by its type-\( \gamma \) elements for any \( \gamma < \alpha \).
		In this way, one extension of a set controls all of the other extensions.
		\item The comprehension axiom allows a set to be built which has a specified extension in a single type.
		The elements not of this type may be considered `controlled junk'.
	\end{enumerate}
\end{remarks}

We now present the following striking theorem.

\begin{theorem}[Holmes]
	\( \mathsf{NF} \) is consistent if and only if \( \mathsf{TTT} \) is consistent. \cite{holmes-ttt}
\end{theorem}

We will actually prove something slightly stronger.

\begin{theorem}
    \label{thm:nf_ttt}
    Let \( T \) be a theory in the language of \( \mathsf{TST} \).
    Let \( T_{\mathsf{NF}} \) be the theory in the language of \( \mathsf{NF} \) given by erasing the type annotations of \( T \).
    Let \( T_{\mathsf{TTT}} \) be the theory in the language of \( \mathsf{TTT} \) given by instantiating the sentences of \( T \) at all possible combinations of type levels.
    Then \( T_{\mathsf{NF}} \) is consistent if and only if \( T_{\mathsf{TTT}} \) is consistent.
\end{theorem}
\begin{proof}
    Suppose that \( T_{\mathsf{NF}} \) has a model \( M \).
    Let \( N \) be the structure in the language of \( \mathsf{TTT} \) where each type \( \alpha \) is interpreted as \( M \), and where the membership relation is given by that on \( M \).
    It is easy to see by induction that all sentences in \( T_{\mathsf{TTT}} \) hold in \( N \), as required.

    Now suppose that \( T_{\mathsf{TTT}} \) has some model \( M \).
    This proof that \( T_{\mathsf{NF}} \) is consistent proceeds in two stages.
    In the first stage, we show that \( T + \mathsf{Amb} \) is consistent, where \( \mathsf{Amb} \) is the \emph{ambiguity scheme}
    \[ \mathsf{Amb} \equiv \{ \varphi \leftrightarrow \varphi^+ \mid \varphi \text{ is a sentence in the language of } \mathsf{TST} \} \]
    % TODO: Write about the (-)^+ operation
    This result is due to Holmes in \cite{holmes-ttt}.
    We will then use this to show that \( T_{\mathsf{NF}} \) is consistent, using a result due to Specker in \cite{typical-ambiguity}.

    Suppose that \( T + \mathsf{Amb} \) is not consistent.
    By compactness, there is some finite set of sentences \( \Sigma \) in the language of \( \mathsf{TST} \) such that \( T + \mathsf{Amb}_\Sigma \) is inconsistent, where
    \[ \mathsf{Amb}_\Sigma \equiv \{ \varphi \leftrightarrow \varphi^+ \mid \varphi \in \Sigma \} \]
    Suppose that \( \Sigma \) uses only type indices \( 0, \dots, n - 1 \).
    Let \( [\lambda]^n \) be the collection of \( n \)-element subsets of \( \lambda \), and define a function \( \sigma : [\lambda]^n \to \mathcal P(\Sigma) \) as follows.
    If
    \[ A = \{\alpha_0, \dots, \alpha_{n-1}\} \text{ with } \alpha_0 < \dots < \alpha_{n-1} \]
    then \( \varphi \in \sigma(A) \) if and only if the interpretation of \( \varphi \) in \( M \) at levels \( \alpha_0, \dots, \alpha_{n-1} \) is true.
    % TODO: Write about interpreting TST formulas in a TTT structure
    This defines a partition of \( [\lambda]^n \) into finitely many subsets.
    By Ramsey's theorem, there is an infinite homogeneous set \( H \subseteq \lambda \) for this partition, that is, if \( A, B \in [H]^n \), then \( \sigma(A) = \sigma(B) \).
    Let \( \alpha_0, \alpha_1, \dots \) be an increasing sequence in \( H \), and define a structure \( N \) in the language of \( \mathsf{TST} \) by interpreting type \( i \) as \( M_{\alpha_i} \).
    Then, \( N \) models \( T + \mathsf{Amb}_\Sigma \) as required.

    Now, we show that the consistency of \( T + \mathsf{Amb} \) implies that of \( T_{\mathsf{NF}} \).
    This relies on a lemma of Specker in \cite{typical-ambiguity}.
    An \emph{endomorphism} of a one-sorted language is an operation \( (-)^\ast \) on the function and relation symbols, mapping them to terms (respectively formulas) with the same free variables.
    This extends in a natural way to formulas in the language.

    We can reformalise \( T \) into a theory \( T' \) over a one-sorted language by adding a unary relation symbol \( T_n \) for each type index \( n \), and recursively replacing each instance of \( \exists x^n.\, \varphi \) with \( \exists x.\, T_n(x) \wedge \varphi \).
    This language has an endomorphism \( (-)^+ \) which maps \( T_n \) to \( T_{n+1} \).

    Specker's lemma can be phrased in the following way.
    \begin{lemma}
        Let \( U \) be a complete theory in a one-sorted language \( L \) with endomorphism \( (-)^\ast \).
        Then if
        \[ U + \{ \varphi \leftrightarrow \varphi^\star \mid \varphi \text{ is an \( L \)-sentence} \} \]
        is consistent, then there is a model \( M \) of \( U \) that admits a function \( f : M \to M \) such that for every relation symbol \( R \) of \( L \),
        \[ M \vDash R(x_1, \dots, x_m) \text{ if and only if } M \vDash R(f(x_1), \dots, f(x_m)) \]
    \end{lemma}
    In our case, \( T + \mathsf{Amb} \) is consistent, so the corresponding one-sorted theory as required for the lemma is consistent (and has a complete extension).
    This requires choosing an interpretation of the membership relation for pairs of type indices that do not differ by one, but this does not interfere with anything that we need (for instance, the relation can always be interpreted as false).
    This yields a model of \( T' \) with a type-raising function \( f \).
    This naturally gives rise to a model of \( T \) in the language of \( \mathsf{TST} \) in which all type levels are isomorphic.
    Therefore, the carrier set of each type level of this model provides a model of \( T_{\mathsf{NF}} \) as required.
\end{proof}

Thus, our task of proving \( \mathsf{NF} \) consistent is reduced to the task of proving \( \mathsf{TTT} \) consistent.
We will do this by exhibiting an explicit model (albeit one that requires a great deal of Choice to construct).
As \( \mathsf{TTT} \) has types indexed by a limit ordinal, and sets can only contain sets of lower type, we can construct a model by recursion over \( \lambda \).
In particular, a model of \( \mathsf{TTT} \) is a well-founded structure.
This was not an option with \( \mathsf{NF} \) directly, as the universe set \( \{ x \mid x = x \} \) would necessarily be constructed before many of its elements.

\subsection{Finitely axiomatising tangled type theory}

Hailperin showed in \cite{hailperin-finite-axiomatisation} that the comprehension scheme of \( \mathsf{NF} \) is equivalent to a finite conjunction of its instances.
These axioms are all stratified (as is extensionality), so \( \mathsf{NF} \) is equivalent to a theory of the form \( T_{\mathsf{NF}} \) where \( T \) is a particular finite theory in the language of \( \mathsf{TST} \).
Then, by \cref{thm:nf_ttt}, the consistency of \( \mathsf{NF} \) can be established by witnessing a model of \( T_{\mathsf{TTT}} \).
The same theorem shows that any model of \( T_{\mathsf{TTT}} \) is a model of \( \mathsf{TTT} \), by executing Hailperin's proof in the language of \( \mathsf{NF} \) and transporting the result back to the language of \( \mathsf{TTT} \).

We will exhibit one such theory \( T \) here, with a list of eleven axioms.
Our choice of axioms for the comprehension scheme are inspired by those used in the Metamath implementation of Hailperin's algorithm in \cite{metamath-nf}.
In the following table, the notation \( \langle a, b \rangle \) denotes the Kuratowski pair \( \{ \{ a \}, \{ a, b \} \} \).
The first column is Hailperin's name for the axiom.
\begin{center}
    \begin{tabular}{lll}
        \( - \) & extensionality & \( \forall x^1.\, \forall y^1.\, (\forall z^0.\, z \in x \leftrightarrow z \in y) \to x = y \) \\
        P1 & anti-intersection & \( \forall x^1 y^1.\, \exists z^1.\, \forall w^0.\, w \in z \leftrightarrow \neg(w \in x \wedge w \in y) \) \\
        P2 & singleton image & \( \forall x^3.\, \exists y^4.\, \forall z^0 w^0.\, \langle \{ z \}, \{ w \} \rangle \in y \leftrightarrow \langle z, w \rangle \in x \) \\
        \( - \) & singleton & \( \forall x^0.\, \exists y^1.\, \forall z^0.\, z \in y \leftrightarrow z = x \) \\
        P3 & insertion two & \( \forall x^3.\, \exists y^5.\, \forall z^0 w^0 t^0.\, \langle \{ \{ z \} \}, \langle w, t \rangle \rangle \in y \leftrightarrow \langle z, t \rangle \in x \) \\
        P4 & insertion three & \( \forall x^3.\, \exists y^5.\, \forall z^0 w^0 t^0.\, \langle \{ \{ z \} \}, \langle w, t \rangle \rangle \in y \leftrightarrow \langle z, w \rangle \in x \) \\
        P5 & cross product & \( \forall x^1.\, \exists y^3.\, \forall z^2.\, z \in y \leftrightarrow \exists w^0 t^0.\, z = \langle w, t \rangle \wedge t \in x \) \\
        P6 & type lowering & \( \forall x^4.\, \exists y^1.\, \forall z^0.\, z \in y \leftrightarrow \forall w^1.\, \langle w, \{ z \} \rangle \in x \) \\
        P7 & converse & \( \forall x^2.\, \exists y^2.\, \forall z^0 w^0.\, \langle z, w \rangle \in y \leftrightarrow \langle w, z \rangle \in x \) \\
        P8 & cardinal one & \( \exists x^2.\, \forall y^1.\, y \in x \leftrightarrow \exists z^0.\, \forall w.\, w \in y \leftrightarrow w = z \) \\
        P9 & subset & \( \exists x^3.\, \forall y^1 z^1.\, \langle y, z \rangle \in x \leftrightarrow \forall w^0.\, w \in y \to w \in z \)
    \end{tabular}
\end{center}


\section{Outline}
\label{sec:outline}
To construct a model of tangled type theory, we build each type individually, and then prove that the resulting structure satisfies the required axioms.
The process for building each type is complicated, and depends on some knowledge about the construction of the previous types.
In the following subsections, we outline the construction the types, as well as the precise facts we need to carry through the inductive hypothesis at each stage.

\subsection{Model parameters}
\label{ss:outline:params}

As described in \cref{ss:theories:ttt}, the types of a given model of tangled type theory are indexed by a limit ordinal \( \lambda \).
Our model will also have two more cardinal parameters, denoted \( \kappa \) and \( \mu \), satisfying \( \lambda < \kappa < \mu \).

Sets smaller than size \( \kappa \) will be called \emph{small}.
We require that \( \kappa \) is a regular cardinal; this ensures that small-indexed unions of small sets are small.
Note that \( \aleph_0 \) is small.

Each type in our model will have size \( \mu \).
We require \( \mu \) to be a strong limit cardinal; power sets of sets smaller than \( \mu \) must also be smaller than \( \mu \).
We stipulate that the cofinality of \( \mu \) is at least \( \kappa \).
This assumption will become important whenever we consider objects indexed by small ordinals.

We remark that these constraints are satisfiable; \( \lambda = \aleph_0, \kappa = \aleph_1, \mu = \beth_{\omega_1} \) suffice.

\subsection{Atoms and permutations}
\label{ss:outline:atoms}

To aid our construction, we will add an additional level of objects below type zero.
These will not be a part of the final model we construct.
This base type will be comprised of objects called \emph{atoms} (although they are not atoms in the traditional model-theoretic sense).

Alongside the construction of the types of our model, we will also construct a collection of permutations of each type, called the \emph{allowable permutations}.
Such permutations will preserve the structure of the model in a strong sense; for instance, they preserve membership.

\subsection{Construction of each type}
\label{ss:outline:construction}

Objects in our model are defined by their elements at all lower type indices.
However, not all collections of extensions may become model elements; for example, they may fail to satisfy extensionality at all levels simultaneously.
We impose two restrictions on what kind of extensions an object may have.

The first restriction is that one of the extensions of a given object must be `preferred', and every other extension must be easily derivable from that particular extension.
This will help us to establish extensionality, as model elements will be the same if and only if their preferred extensions are the same.
The system to compute other extensions uses a construction called the \emph{fuzz} map.
This map turns information about one extension into `ordered junk' in another extension, in such a way that the model cannot learn anything useful about the non-preferred extensions.
Our allowable permutations will be defined as a set of permutations that respect the fuzz map.

The second restriction is that the object must have a small \emph{support} comprised of \emph{addresses}.
That is, the behaviour of the object under the action of allowable permutations must be fully characterisable by the behaviour of a small set of addresses under allowable permutations.
This will ensure that the objects of our model are not too complex.
Because the cofinality of \( \mu \) is at least \( \kappa \), there are only \( \mu \) small sets of elements taken from a collection of size \( \mu \); this observation will play a key role in establishing the sizes of our types.

\subsection{Constraining the size of each type}
\label{ss:outline:size}

The construction of a given type can only be done under the assumption that each smaller type was of size exactly \( \mu \).
This means that we need to prove that each type has size \( \mu \) in the inductive step.
In order to do this, we will need to show that there are a lot of allowable permutations.
The main theorem establishing this, called the \emph{freedom of action theorem}, roughly states that under certain assumptions, a permutation defined on a small set of addresses can be extended to an allowable permutation.
The majority of this paper will be allocated to proving the freedom of action theorem, and it will be outlined in more detail when we are in a position to prove it.
One this is established, we can prove that the size of each type is precisely \( \mu \) by carefully counting the possible ways to describe a model element.

\subsection{Finishing the induction}
\label{ss:outline:finishing}

We can then finish the inductive step and build the entire model.
It remains to show that this is a model of TTT as desired.
This part of the proof is quite direct, and also uses the freedom of action theorem.


\section{The underlying structure}
\label{sec:structure}
\subsection{Atoms, litters, and near-litters}

As described in \cref{ss:outline:atoms}, we have an additional level of objects below type zero.
To index the levels of the model, together with this new level, we make the following definition.
\begin{definition}
    A \emph{type index} is an element of \( \lambda \) or a distinguished symbol \( \bot \).
    We impose an order on type indices by setting \( \bot < \alpha \) for all \( \alpha \in \lambda \).
    The set of type indices is denoted \( \lambda^\bot \).
\end{definition}
Elements of \( \lambda \) may be called \emph{proper type indices}.

Our base type is a set of \emph{atoms}, organised into \emph{litters}.
\begin{definition}
    A \emph{litter} is a triple \( L = (\nu, \beta, \gamma) \) where \( \nu \in \mu \), \( \beta \) is a type index, and \( \gamma \neq \beta \) is a proper type index.
\end{definition}
This somewhat arcane definition will be used to great effect later when defining the fuzz map.
A litter \( L = (\nu, \beta, \gamma) \) encodes data coming from type \( \beta \) and going into type \( \gamma \).
Note that \( \beta \) may be \( \bot \), but \( \gamma \) may not; this corresponds to the fact that we never construct data in type \( \bot \) from data at higher levels.
The first component \( \nu \) is an index allowing us to have \( \mu \) distinct litters with the same source and target types.
\begin{remark}
    There are precisely \( \mu \) litters.
\end{remark}
\begin{definition}
    An \emph{atom} is a pair \( a = (L, i) \) where \( L \) is a litter and \( i \in \kappa \).
    The \emph{associated litter} of an atom is its first projection \( \pr_1(a) \), written \( a^\circ \) for brevity.
    The \emph{litter set} \( \LS(L) \) of a given litter \( L \) is the set of atoms whose associated litter is \( L \); that is, \( \LS(L) = \{ (L, i) \mid i \in \kappa \} \).
    The litter sets partition the set of atoms into \( \mu \) sets of \( \kappa \) atoms, and there are \( \mu \) atoms in total.
\end{definition}
\begin{remark}
    Many of our constructions rely on having only a small set of constraints.
    If our constraints take the form of atoms, the smallness assumption guarantees that most of the atoms in a given litter are unconstrained.
    Motivated by smallness concerns, we make the following definition.
\end{remark}
\begin{definition}
    A \emph{near-litter} is a pair \( N = (L, s) \) where \( L \) is a litter and \( s \) is a set of atoms with small symmetric difference to the litter set of \( L \).
    We say that the \emph{associated litter} of \( N \) is \( N^\circ = \pr_1(N) \), or that \( N \) is \emph{near} \( L \).
\end{definition}
\begin{remarks}\mbox{\negthinspace}
    \label{rk:mk_near_litter}
    \begin{enumerate}
        \item A set of atoms can be near at most one litter.
        For brevity, we will frequently identify a near-litter with its underlying set.
        \item The litter set of any litter \( L \) can be made into a near-litter: \( \NL(L) = (L, \LS(L)) \).
        \item Each near-litter has size exactly \( \kappa \), and there are \( \mu \) near-litters in total; the latter follows from the fact that the cofinality of \( \mu \) is at least \( \kappa \).
    \end{enumerate}
\end{remarks}
We can now define the allowable permutations of type \( \bot \), although we will give them a different name for now; they will be precisely those permutations of atoms that respect the structure of near-litters.
\begin{definition}
    A \emph{near-litter permutation} \( \pi \) is a permutation of atoms that sends near-litters to near-litters.
\end{definition}
\begin{remarks}\mbox{\negthinspace}
    \begin{enumerate}
        \item A near-litter permutation \( \pi \) induces a permutation of litters, which we will also call \( \pi \).
        This is defined by mapping \( L \) to the associated litter of \( \pi '' \LS(L) \), where the double apostrophe denotes pointwise function application (\( f '' s \) denotes the set \( \{ f(x) \mid x \in s \} \)).
        Thus, a near-litter permutation is simultaneously a permutation of atoms, litters, and near-litters.
        \item The set of near-litter permutations forms a group under composition.
    \end{enumerate}
\end{remarks}

\subsection{Higher type structure}

A type-\( \alpha \) object has elements of any type \( \beta < \alpha \), which have elements of any type \( \gamma < \beta \), and so on; we must eventually reach \( \bot \) in a finite number of steps by well-foundedness.
We will now make a definition to deal with sequences of type indices obtained in this way.
\begin{definition}
    A \emph{path of type indices} \( \alpha \rightsquigarrow \varepsilon \) is a finite sequence of type indices
    \[ \alpha > \beta > \gamma > \dots > \varepsilon \]
    If \( \alpha \) is a type index, an \emph{\( \alpha \)-extended type index} is a path \( \alpha \rightsquigarrow \bot \).
    If \( A \) is a path \( \alpha \rightsquigarrow \beta \) and \( B \) is a path \( \beta \rightsquigarrow \gamma \), their composition \( A \gg B \) is a path \( \alpha \rightsquigarrow \gamma \) obtained by concatenation of the sequences but removing the duplicated index \( \beta \).
\end{definition}
\begin{remark}
    \label{rk:mk_extended_index}
    For any \( \alpha \in \lambda^\bot \), the set of \( \alpha \)-extended type indices has size at most \( \lambda \).
\end{remark}
In our model, the iterated extensions of objects of type \( \alpha \) are indexed by the \( \alpha \)-extended type indices.
We can apply operations to an object along each path \( \alpha \rightsquigarrow \bot \).
One important notion defined in this way will be \emph{structural permutations}.
\begin{definition}
    \label{def:struct_perm}
    An \emph{\( \alpha \)-structural permutation} \( \pi \) assigns a near-litter permutation to each \( \alpha \)-extended index.
    If \( A \) is an \( \alpha \)-extended index, the assigned near-litter permutation is called the \emph{derivative} of \( \pi \) along \( A \), and is denoted \( \pi_A \).
    We can extend this notion to arbitrary paths \( \alpha \rightsquigarrow \beta \); in this case, the derivative \( \pi_A \) is a \( \beta \)-structural permutation.
    The group of \( \alpha \)-structural permutations is denoted \( S_\alpha \).
\end{definition}
\begin{remark}
    Near-litter permutations may be identified with \( \bot \)-structural permutations.
\end{remark}
To apply an \( \alpha \)-structural permutation \( \pi \) to some model element \( x \) of type \( \alpha \), we first consider its elements of some type \( \beta < \alpha \).
By recursion, we apply \( \pi_{(\alpha > \beta)} \) to each such element, where \( \pi_{(\alpha > \beta)} \) is the derivative of \( \pi \) along the one-step path from \( \alpha \) to \( \beta \).
These images are then assembled to form a new object of type \( \alpha \); note however that this new object need not in general satisfy the constraints outlined in \cref{ss:outline:construction}, so structural permutations need not map elements of the model to other elements of the model.

\subsection{Addresses and supports}

We are interested in objects that can be characterised by a small amount of data; this data will take the form of \emph{addresses}.
\begin{definition}
    Let \( \alpha \) be a type index.
    An \emph{\( \alpha \)-address} is a pair \( (A, x) \) where \( A \) is an \( \alpha \)-extended type index and \( x \) is either an atom or a near-litter.
\end{definition}
An \( \alpha \)-address encodes a set of atoms, and a path to get there from a type-\( \alpha \) object by descending through iterated extensions.
\begin{note}
    In \cite{holmes2023nf}, addresses are called \emph{support conditions}.
\end{note}
\begin{remark}
    \label{rk:mk_address}
    The set of \( \alpha \)-addresses has cardinality \( \mu \).
    This follows from \cref{rk:mk_near_litter,rk:mk_extended_index}.
\end{remark}

The group of \( \alpha \)-structural permutations acts on the set of \( \alpha \)-addresses by
\[ \pi(A, x) = (A, \pi_A(x)) \]
We would like to say that a \emph{support} is a small set of \( \alpha \)-addresses; for technical reasons that will be discussed later it will actually be better to make the following definition instead.
\begin{definition}
    An \( \alpha \)-support is a function \( S \) whose domain is a small ordinal and whose values are \( \alpha \)-addresses, such that whenever \( N_1, N_2 \) are near-litters in the range of \( S \), the atoms in their symmetric difference \( N_1 \symmdiff N_2 \) are also in the range of \( S \).
    The \emph{underlying set} of a support is its range.
\end{definition}
\begin{remarks}\mbox{\negthinspace}
    \begin{enumerate}
        \item An address may occur multiple times in the range of a support, but all we usually care about is the presence or absence of an address.
        The fact that these addresses have indices in a given support will be useful for some constructions later.
        The symmetric difference condition is also a technical condition that we will use later, it avoids a problem where initial conditions for a construction could be overspecified.
        \item For any small set of addresses \( s \), there is a support \( S \) whose range is \( s \) together with the small set of atom addresses required to satisfy the symmetric difference constraint.
    \end{enumerate}
\end{remarks}
As with many parts of our construction, we will need to count precisely how many supports there are.
We will first need to establish the following lemma.
\begin{lemma}
    \label{lem:konig_converse}
    Let \( \mu \) be a strong limit cardinal.
    Then for any \( \kappa > 0 \) strictly smaller than the cofinality of \( \mu \), we have
    \[ \mu^\kappa = \mu \]
\end{lemma}
Recall that \emph{König's theorem} states that for an infinite cardinal \( \mu \), we have \( \mu^{\operatorname{cf}(\kappa)} > \mu \); this lemma is a partial converse to this theorem in the case where \( \mu \) is a strong limit.
\begin{proof}
    Let \( A \) be a set of size \( \kappa \), and let \( B \) be a set of size \( \mu \).
    We assume that \( B \) has a well-ordering, perhaps obtained from the order on \( \mu \).
    Given any function \( f : A \to B \), we define a relation \( \prec_f \) on \( A \) by the inverse image of the ordering on \( B \) under \( f \):
    \[ a \prec_f b \iff f(a) < f(b) \]
    One can show by induction on \( B \) that any for two functions \( f, g : A \to B \), if \( \im f = \im g \) and \( \prec_f = \prec_g \), then \( f = g \).
    The amount of pairs \( (\im f, \prec_f) \) is bounded by \( \mu \).
    Indeed, the amount of possible images of a function \( f : A \to B \) is bounded by \( \mu \) as \( \kappa \) is less than the cofinality of \( \mu \) and \( \mu \) is a strong limit, and the amount of relations \( \prec_f \) is bounded by \( (2^\kappa)^\kappa \), which is bounded by \( \mu \) as it is a strong limit.
    Thus, \( \mu^\kappa \leq \mu \), and the converse is clear.
\end{proof}
We can now prove that there are exactly \( \mu \) supports for any type index \( \alpha \).
\begin{proposition}
    \label{prop:mk_support}
    Let \( \alpha \) be a type index.
    The set of \( \alpha \)-supports has cardinality \( \mu \).
\end{proposition}
\begin{proof}
    A support is a function from a small ordinal to the set of \( \alpha \)-addresses, which has cardinality \( \mu \) by \cref{rk:mk_address}.
    Thus, the amount of supports is bounded by
    \[ \sum_{i < \kappa} \mu^i \]
    By \cref{lem:konig_converse}, each summand is precisely \( \mu \), and so the sum is precisely \( \kappa \cdot \mu = \mu \).
\end{proof}
The \( \alpha \)-structural permutations act on \( \alpha \)-supports pointwise:
\[ \pi(S)(i) = \pi(S(i)) \]
We will need to concatenate supports together.
The naïve concatenation of supports \( S_1, S_2 \) given by
\[ S(i) = \begin{cases}
    S_1(i) & \text{if } i \in \dom S_1 \\
    S_2(i - \dom S_1) & \text{otherwise}
\end{cases} \]
need not satisfy the symmetric difference condition, so \( S \) may not be a support.
There is not a unique way to expand this concatenation into a support, so we will instead make the following definition.
\begin{definition}
    Let \( f \) be a function from a small ordinal whose values are \( \alpha \)-support conditions.
    We say that a support \( S \) is a \emph{completion} of \( f \) if \( S|_{\dom f} = f \) and every address in the extension \( S|_{\dom S \setminus \dom f} \) is an atom address \( (A, x) \) where \( x \in N_1 \symmdiff N_2 \) and \( (A, N_1), (A, N_2) \in \ran f \).
    Every such function has a completion, as the amount of atom addresses in question is small.
    A support \( S \) is a \emph{sum} of supports \( S_1, S_2 \) if it is a completion of the concatenation of \( S_1 \) and \( S_2 \).
    Every pair of supports has a sum.
\end{definition}


\section{Constructing the types}
\label{sec:construction}
In this section, we describe the way in which we construct each type in the model, under the assumption that all previous levels were constructed properly.

\subsection{Hypotheses}

We explicitly describe the hypotheses required for the inductive step to succeed.
The data constructed at each level is called \emph{tangle data}; for uniformity we will also define tangle data at level \( \bot \).

\begin{definition}
    Let \( \alpha \) be a type index.
    Then \emph{tangle data} at level \( \alpha \) consists of
    \begin{enumerate}
        \item A set \( \tau_\alpha \), called the set of \emph{\( \alpha \)-tangles}.
        Tangles encapsulate our model elements.
        \item A group \( A_\alpha \), called the set of \emph{\( \alpha \)-allowable permutations}, which acts on \( \tau_\alpha \).
        Allowable permutations will be denoted using the symbol \( \rho \).
        \item A group homomorphism \( A_\alpha \to S_\alpha \), where \( S_\alpha \) is the group of \( \alpha \)-structural permutations as defined in \cref{def:struct_perm}.
        We will notationally suppress this homomorphism, and treat \( A_\alpha \) as a subgroup of \( S_\alpha \), so allowable permutations act on anything that structural permutations do.
        \item A function \( \supp \) that assigns an \( \alpha \)-support to each \( \alpha \)-tangle.
        This support must actually support the input tangle: for any \( \alpha \)-allowable permutation \( \rho \), if \( \rho \) fixes \( \supp t \) pointwise, it fixes \( t \).
        \[ (\forall i.\, \rho((\supp t)(i)) = (\supp t)(i)) \to \rho(t) = t \]
        We require that \( \supp \) commutes with \( \alpha \)-allowable permutations \( \rho \) in the sense that
        \[ \supp(\rho(t)) = \rho(\supp t) \]
    \end{enumerate}
\end{definition}

The tangle data at level \( \bot \) is constructed by taking the tangles to be the atoms, the allowable permutations to be the near-litter permutations, and the support function to be given by
\[ \dom(\supp a) = 1;\quad (\supp a)(0) = (\varnothing, a) \]
where \( \varnothing \) is the empty path \( \bot \rightsquigarrow \bot \).

At higher levels \( \alpha \in \lambda \), the tangles will be pairs \( (x, S) \) where \( x \) is a model element of type \( \alpha \) and \( S \) is a support that supports \( x \) under the action of \( \alpha \)-allowable permutations.
There are \( \mu \) tangles that encapsulate any given model element.
Allowable permutations act on these pairs pointwise.
\[ \rho((x, S)) = (\rho(x), \rho(S)) \]
It is clear to see that the second projection \( \pr_2 \) satisfies the requirements of \( \supp \) defined in (iv) above are satisfied.

\begin{definition}
    Let \( \alpha \) be a type index with tangle data.
    Then a \emph{position function} at level \( \alpha \) is an injection \( \iota_\alpha : \tau_\alpha \to \mu \).
    For convenience, we will simply write \( \iota \) for all position functions.
\end{definition}

At level \( \bot \), the position function is chosen arbitrarily; one exists as there are exactly \( \mu \) atoms.
At higher levels \( \alpha \in \lambda \), the existence of a position function depends on the fact that \( \tau_\alpha \) has size at most \( \mu \).
We will typically assume that all type levels previously constructed have a position function.

Let \( \alpha \in \lambda \), and let \( N \) be a near-litter.
We will construct our model in such a way that there is an element \( x \) of type \( \alpha \) with a \( \bot \)-extension that is exactly (the atoms inside) \( N \).
This will imply, in particular, that each type must have size at least \( \mu \) by \cref{rk:mk_near_litter}.
Together with the existence of a position function, the size of each type can be seen to be exactly \( \mu \).
We cannot express the notion of a \( \bot \)-extension directly with the language available to us, but we do not need it in order to capture the information relevant for this section.

\begin{definition}
    Let \( \alpha \) be a proper type index with tangle data.
    We say that \( \alpha \) has \emph{typed near-litters} if there is an injection \( \typed_\alpha \) from the set of near-litters to \( \tau_\alpha \), commuting with allowable permutations in the following sense.
    \[ \rho(\typed_\alpha N) = \typed_\alpha (\rho_{(\alpha > \bot)}(N)) \]
\end{definition}

\subsection{The fuzz map}

We will now perform a construction that underpins the rest of the model, and helps us enforce extensionality.
In tangled type theory, a given model element has extensions at each level below it.
In our model, each object has a `preferred' extension, and must find a way to compute the other extensions from that information.
In order to do this, we need to be able to convert arbitrary model elements into `junk' at other levels, which can then be interpreted as a `non-preferred' extension.

The \emph{fuzz maps} perform this task.
They are parametrised by type indices \( \beta \neq \gamma \) where \( \gamma \neq \bot \), representing the source type level and the target type level.
For each such pair, the fuzz map is an injection from \( \beta \)-tangles to litters.
An arbitrary litter can only be the image of a fuzz map defined at a single pair of type levels.

Treating the output of the fuzz map as a typed near-litter, its position is always greater than the position of the input to the function.
A similar property holds for atoms.
This ensures a well-foundedness condition that we can use in many places later.

\begin{definition}
    \label{def:fuzz}
    Let \( \beta \) be a type index, and let \( \gamma \neq \beta \) be a proper type index.
    Suppose that \( \beta, \gamma \) have tangle data and a position function, and that \( \gamma \) has typed near-litters.
    A \emph{fuzz map} \( f_{\beta,\gamma} \) is an injection from \( \tau_\beta \) to the set of litters, such that
    \begin{enumerate}
        \item For any \( t \in \tau_\beta \), \( f_{\beta,\gamma}(t) = (\nu, \beta, \gamma) \) for some \( \nu \in \mu \), so all of the fuzz maps have pairwise disjoint ranges.
        \item For any \( t \in \tau_\beta \) and any near-litter \( N \) near \( f_{\beta,\gamma}(N) \),
        \[ \iota_\beta(t) < \iota_\gamma(\typed_\gamma N) \]
        \item For any \( t \in \tau_\beta \) and atom \( a \) in the litter set corresponding to \( f_{\beta,\gamma}(t) \),
        \[ \iota_\beta(t) < \iota_\bot(a) \]
    \end{enumerate}
\end{definition}

We will now show that such a function exists, and will thenceforth refer to it as \emph{the} fuzz map.

\begin{proposition}
    Let \( \beta \) be a type index, and let \( \gamma \neq \beta \) be a proper type index.
    Suppose that \( \beta, \gamma \) have tangle data and a position function, and that \( \gamma \) has typed near-litters.
    Then a fuzz map \( f_{\beta,\gamma} \) exists.
\end{proposition}
\begin{proof}
    We construct the map by recursion along \( \tau_\beta \), with the order induced by \( \iota_\beta \).
    Suppose we have already constructed \( f_{\beta,\gamma} \) for \( s \in \tau_\beta \) with \( \iota(s) < \iota(t) \).
    We must choose an index \( \nu \in \mu \) subject to three constraints:
    % TODO: check phrasing
    \begin{enumerate}
        \item \( \nu \) was not picked for any previous \( s \);
        \item \( \iota(t) \) is greater than \( \iota_\gamma(\typed_\gamma N) \) for any near-litter near \( (\nu, \beta, \gamma) \);
        \item \( \iota(t) \) is greater than \( \iota_\bot(a) \) for any atom \( a \) with \( a^\circ = (\nu, \beta, \gamma) \).
    \end{enumerate}
    Each constraint denies us less than \( \mu \) choices, and so there is always an available index \( \nu \) to choose.
\end{proof}

\subsection{Codes and clouds}

From now, we will assume that \( \alpha \in \lambda \) is a fixed proper type index at which we intend to construct tangle data.
In this section, all other (proper) type indices will be assumed to be strictly less than \( \alpha \).
We assume that we have tangle data, position functions, and typed near-litters for all type indices below \( \alpha \).

We will now begin the process of stitching together the fuzz maps to form the function that deduces an alternative extension from a preferred extension.

\begin{definition}
    Let \( \beta < \alpha \) be a type index, and let \( \gamma < \alpha \) be a proper type index not equal to \( \beta \).
    The \emph{cloud map} \( c_{\beta,\gamma} : \mathcal P(\tau_\beta) \to \mathcal P(\tau_\gamma) \) is given by
    \[ c_{\beta,\gamma}(s) = {\typed_\gamma} '' \bigcup_{t \in s} \{ N \mid N^\circ = f_{\beta,\gamma}(t) \} \]
\end{definition}

For each \( t \in s \), the map \( c_{\beta,\gamma} \) produces the `cloud' of near-litters near to \( f_{\beta,\gamma}(t) \), and turns them into tangles through the typed near-litter map.
This will be used to construct the alternative extension map.

\begin{note}
    In \cite{holmes2023nf}, the cloud map is called \( A_{\beta,\gamma} \), but in this paper the notation \( A_{\beta,\gamma} \) is reserved for the alternative extension map.
\end{note}

\begin{remark}
    The cloud map is injective, and further, if \( c_{\beta,\gamma}(s) = c_{\beta',\gamma}(s') \) and \( s \) is nonempty, then \( \beta = \beta' \).
    In particular, a nonempty set of \( \gamma \)-tangles has at most one inverse image under any cloud map \( c_{\beta,\gamma} \).
    We will show that there are only finitely many iterated images under an inverse cloud map, and we will use this to define preferred extensions.
    To prove this, we will make the following definition that encapsulates the notion of a particular choice of \( \beta < \alpha \) and an extension at that level.
\end{remark}

\begin{definition}
    Let \( \alpha \in \lambda \).
    An \emph{\( \alpha \)-code} is a pair \( (\beta, s) \) where \( \beta < \alpha \) is a type index, and \( s \subseteq \tau_\beta \).
\end{definition}

\begin{definition}
    Let \( \beta < \alpha \) be a proper type index.
    The \emph{code cloud map} \( C_\beta \) is a map from \( \alpha \)-codes to \( \alpha \)-codes, given by
    \[ C_\beta((\gamma, s)) = \begin{cases}
        (\beta, s) & \text{if } \gamma = \beta \\
        (\beta, c_{\gamma,\beta}(s)) & \text{if } \gamma \neq \beta
    \end{cases} \]
    That is, \( C_\beta \) applies the relevant cloud map to obtain a code describing a \( \beta \)-extension, or does nothing if the first projection of the code is already \( \beta \).
\end{definition}

Observe that \( C_\beta \) is injective on codes with first projection not equal to \( \beta \).
We will show that there are only finitely many iterated images under the inverse of \( C_\beta \).

\begin{definition}
    Define a relation \( \rightcurvedarrow \) on \( \alpha \)-codes by letting \( x \rightcurvedarrow C_\beta(x) \) whenever \( \pr_1(x) \neq \beta \).
\end{definition}
\begin{remark}
    A code \( x \) has at most one predecessor under \( \rightcurvedarrow \).
    If \( x \rightcurvedarrow y \), then \( x \) is empty (more precisely, its second projection is empty) if and only if \( y \) is.
    Moreover, all empty codes are related to each other under \( \rightcurvedarrow \).
\end{remark}

\begin{lemma}
    \label{lem:code_wf}
    The relation \( \rightcurvedarrow \) is well-founded on nonempty codes.
\end{lemma}
\begin{proof}
    For each nonempty \( \alpha \)-code \( x \), we define an ordinal \( \iota(x) \in \mu \) given by the smallest position of any tangle in \( \pr_2(x) \).
    \[ \iota((\beta, s)) = \min_{t \in s} \iota_\beta(t) \]
    We show that if \( x \rightcurvedarrow y \), then \( \iota(x) < \iota(y) \).
    Let \( x = (\beta, s) \) and \( y = (\gamma, c_{\beta,\gamma}(s)) \).
    Suppose \( t \in \tau_\gamma \) is the tangle with smallest position in \( c_{\beta,\gamma}(s) \).
    Then \( t = \typed_\gamma N \) where \( N = f_{\beta,\gamma}(t') \) for some \( t' \in s \).
    It suffices to show \( \iota_\beta(t') < \iota_\gamma(t) \), but this holds by the construction of the fuzz map in \cref{def:fuzz}.
\end{proof}

An object in our model will correspond to a collection of codes, representing its extensions.
A given object at level \( \alpha \) has exactly one extension at every proper type index \( \beta < \alpha \), and may optionally have a \( \bot \)-extension.

By \cref{lem:code_wf}, the set of nonempty \( \alpha \)-codes forms a tree.
Each code has a finite height; a code with no \( \rightcurvedarrow \)-predecessor is of height zero, and the height of any other code can be computed recursively.
We will partition the tree of nonempty \( \alpha \)-codes into smaller trees of height 1, and use those as equivalence classes of codes representing a single model element.

\begin{definition}
    We define an equivalence relation \( \equiv \) on nonempty \( \alpha \)-codes, generated by the assertion that \( x \equiv y \) whenever \( x \rightcurvedarrow y \) and \( x \) has even height.
\end{definition}

\begin{remarks}\mbox{\negthinspace}
    \begin{enumerate}
        \item If two codes \( x, y \) are equivalent, then either \( x = y \), or \( x \) has even height and \( x \rightcurvedarrow y \) (or vice versa), or both \( x, y \) have odd height and there is some \( z \) with \( z \rightcurvedarrow x, y \).
        \item Each equivalence class contains precisely one even code.
        This will be the `preferred' extension of the corresponding model element, as outlined in \cref{ss:outline:construction}.
        \item Each equivalence class contains exactly one code for each proper type index \( \beta < \alpha \).
        If the equivalence class contains a code at level \( \bot \), it must be the even code, because such codes can never be produced by the code cloud map.
    \end{enumerate}
\end{remarks}

\subsection{Model elements}

We now collate equivalence classes of codes to create our model elements.

\begin{definition}
    An \emph{\( \alpha \)-preobject} \( x \) assigns to each proper type index \( \beta < \alpha \) a set of \( \beta \)-tangles \( x_\beta \), in such a way that either
    \begin{enumerate}
        \item there is a set \( s \) of atoms such that \( x_\beta = c_{\bot,\beta}(s) \) for all \( \beta < \alpha \); or
        \item there is a proper type index \( \beta < \alpha \) such that \( (\beta, x_\beta) \) is an even code and \( x_\gamma = c_{\beta,\gamma}(x_\beta) \) for all \( \gamma \neq \beta \).
    \end{enumerate}
    Thus nonempty preobjects correspond to equivalence classes of nonempty codes, and there is precisely one empty \( \alpha \)-preobject which corresponds to all of the empty codes.
    An even representative code can be easily extracted from any preobject.
\end{definition}

These preobjects satisfy the required form of extensionality in tangled type theory.

\begin{theorem}
    Let \( x, y \) be \( \alpha \)-preobjects, and let \( \beta < \alpha \) be a proper type index such that \( x_\beta = y_\beta \).
    Then \( x = y \).
\end{theorem}
Note that applying metatheoretic extensionality to \( x_\beta \) and \( y_\beta \) strengthens this assertion into the proper form.
\begin{proof}
    First, note that if \( x \) and \( y \) have the same representative (even) code, then they are equal.
    This follows from the fact that the extensions of a preobject can be calculated by appling the code cloud map to the representative code.
    We have three cases.
    \begin{enumerate}
        \item \emph{The extensions of \( x \) and \( y \) can both be calculated from a set of atoms.}
        Let \( s_x, s_y \) be the set of atoms for \( x, y \) respectively.
        Then \( c_{\bot,\beta}(s_x) = x_\beta = y_\beta = c_{\bot,\beta}(s_y) \), hence \( s_x = s_y \), so \( x \) and \( y \) have the same representative code.
        \item \emph{The extensions of \( x \) can be calculated from a set of atoms \( s \), and the extensions of \( y \) can be calculated from its \( \gamma \)-extension.}
        We must show that \( (\bot, s) \equiv (\gamma, y_\gamma) \).
        Suppose \( \beta = \gamma \), so \( c_{\bot,\beta}(s) = x_\beta = y_\beta \).
        In this case, \( (\bot, s) \equiv (\beta, x_\beta) \) by assumption, giving the result.
        Instead, suppose \( \beta \neq \gamma \).
        In this case,
        \[ (\gamma, y_\gamma) \equiv (\beta, c_{\gamma,\beta}(y_\gamma)) = (\beta, y_\beta) = (\beta, x_\beta) \equiv (\bot, s_x) \]
        The case where \( x \) and \( y \) are swapped holds by symmetry.
        \item \emph{The extensions of \( x \) can be calculated from its \( \gamma \)-extension, and the extensions of \( y \) can be calculated from its \( \delta \)-extension.}
        We must show \( (\gamma, x_\gamma) \equiv (\delta, y_\delta) \).
        We have
        \[ (\gamma, x_\gamma) \equiv (\beta, x_\beta) = (\beta, y_\beta) \equiv (\delta, y_\delta) \]
    \end{enumerate}
\end{proof}

We want to define the \( \alpha \)-objects to be those \( \alpha \)-preobjects with a small support under the action of \( \alpha \)-allowable permutations, so we must first define this group.
The allowable permutations will be built up from allowable permutations at lower levels.
These will become the derivatives along the one-step paths \( \alpha > \beta \).
These permutations are constrained so as to commute with the fuzz map.

\begin{definition}
    \label{def:allowable}
    An \emph{\( \alpha \)-allowable permutation} \( \rho \) assigns to each type index \( \beta < \alpha \) a \( \beta \)-allowable permutation \( \rho_{(\alpha > \beta)} \), in such a way that
    \begin{equation}
        (\rho_{(\alpha > \gamma)})_{(\gamma > \bot)}(f_{\beta,\gamma}(t)) = f_{\beta,\gamma}(\rho_{(\alpha > \beta)}(t))
        \tag{\( \ast \)}
    \end{equation}
    for all type indices \( \beta \), all proper type indices \( \gamma \neq \beta \), and all \( \beta \)-tangles \( t \).
    The homomorphism to the group of \( \alpha \)-structural permutations is given by mapping each of the one-step derivatives \( \rho_{(\alpha > \beta)} \) to structural permutations and collating the results.
\end{definition}
\begin{remark}
    \begin{enumerate}
        \item Note that the appearance of \( \beta \)-allowable permutations in this definition is not circular; these are assumed to exist in the tangle data assigned to level \( \beta \).
        We will later stipulate that the allowable permutations obtained through tangle data at lower levels already satisfy this condition.
        \item A small calculation reveals that condition (\( \ast \)) enforces, for any allowable \( \rho \), that for any set \( s \subseteq \tau_\beta \) and \( \gamma \neq \beta \) proper, \[ {\rho_{(\alpha > \gamma)}} '' c_{\beta,\gamma}(s) = c_{\beta,\gamma}({\rho_{(\alpha > \beta)}} '' s) \]
        Hence, for any code \( x \), we have \( \rho(C_\beta(x)) = C_\beta(\rho(x)) \), and so allowable permutations preserve code equivalence.
        This suggests that they are a sensible choice for permutations that respect the structure of \( \alpha \)-preobjects.
    \end{enumerate}
\end{remark}

We are now in a position to define our model elements.

\begin{definition}
    An \emph{\( \alpha \)-object} is an \( \alpha \)-preobject that admits an \( \alpha \)-support that supports it under the action of \( \alpha \)-allowable permutations.
    That is, \( x \) is an object if there is some support \( S \) such that \( \rho(S) = S \) implies \( \rho(x) = x \).
    An \emph{\( \alpha \)-tangle} is a pair \( (x, S) \) where \( x \) is an \( \alpha \)-object and \( S \) is an \( \alpha \)-support that supports \( x \) under the action of allowable permutations.
\end{definition}

We have thus defined tangle data at level \( \alpha \).
We can also immediately define typed near-litters at level \( \alpha \), by considering the preobject \( x \) given by the even code \( (\bot, N) \).
This evidently has a small support \( S \) given by
\[ \dom S = 1;\quad S(0) = ((\alpha > \bot), N) \]
and thus this is an \( \alpha \)-object.
The map \( \typed_\alpha \) thus sends \( N \) to the tangle \( (x, S) \).

\subsection{Comments}

We have constructed tangle data and typed near-litters at level \( \alpha \), under the assumption that they exist (along with position functions) at all lower type levels.

The use of a group of permutations is known as a \emph{Fraenkel--Mostowski construction}.
Permutations of atoms in a model of a set theory with atoms give rise to permutations of the entire structure.
Typically, such constructions involve a normal filter of subgroups of such permutations; this role is played by our allowable permutations.

Recall that a tangle is a model elements bundled together with a support.
The main benefit that tangles provide is that the support function \( \supp \) commutes with allowable permutations.
In other Fraenkel--Mostowski constructions, objects typically have a \emph{minimal support}, and these commute with allowable permutations; by bundling supports, we retain this nice behaviour without requiring our objects to have minimal supports (and in fact they do not).
Note that tangles themselves do have minimal supports, namely, the support given by \( \supp \).
Picturing tangles instead of model elements as the primitive notion, we lose extensionality but gain some some symmetry that we will use to great effect later.
The idea to bundle tangles in this way was not part of the original proof, and is a new contribution by this author.

We will later impose a restriction on objects that if \( (x, S) \) is an element in some extension, then \( (x, T) \) is also an element for any other support \( T \).
% TODO: Can this ever violate the small support condition?
We can then collapse the entire model by taking the first projection of every tangle, reducing it to a model of tangled type theory satisfying extensionality as desired.

The condition that \( \supp(\rho(t)) = \rho(\supp t) \) was not used in this section.
In particular, this entire construction succeeds by replacing the lower-type tangles in the tangle data with model elements themselves.
This method was used in an earlier version of the proof before shifting to the new `bundled' approach.


\section{Freedom of action}
\label{sec:foa}
\subsection{Outline}
\label{ss:foa:outline}

In this section, we will prove the freedom of action theorem.
Two versions are stated in \cref{thm:foa,thm:foa_behaviour}.
This will allow us to prove that the collection of model elements we have constructed has size exactly \( \mu \), among other things.
The idea of the theorem is that under suitable hypotheses, a partially defined map of \( \alpha \)-addresses can be extended into an \( \alpha \)-allowable permutation.

There are many different things one could mean by a `partially defined map', and we will need several of these for various purposes.
The two most important forms of partial maps of addresses are \emph{approximations} and \emph{actions}, the former of which will be used in the statement of the theorem.
We will prove some technical results linking approximations and actions.
The version of freedom of action stated in \cref{thm:foa_behaviour} uses a third notion, called \emph{behaviours}.

We will construct the allowable permutation by induction along a particular well-founded relation on addresses: we will define \( \rho((A, x)) \) given the knowledge of \( \rho((B, y)) \) for all \( (B, y) \prec (A, x) \).
In certain cases, we will need to know that the freedom of action theorem holds at all lower levels; we accomplish this by wrapping the entire proof in an induction over all proper type indices below \( \alpha \).

We will then demonstrate that the map as defined is a permutation, and fulfils the criteria to be an allowable permutation.

Our proof for this theorem will not depend on the construction given in \cref{sec:construction}, but rather on an explicit list of properties that the tangles at each level must satisfy for the theorem to hold; this is discussed in \cref{ss:foa:hypotheses}.

\subsection{Partial permutations}
\label{ss:foa:partial_perm}

In the proof of freedom of action, we will often have injective functions that we wish to turn into permutations.
In this subsection, we describe a sufficient condition for such a permutation to exist.

\begin{definition}
    A \emph{partial permutation} on a set \( X \) is a partial function \( f : X \rightdasharrow X \) that is a permutation of its domain.
\end{definition}

Suppose we have an injective partial function \( f : X \rightdasharrow X \).
We will construct a pair of functions \( g \) and \( h \) that agree with \( f \) and its inverse respectively on \( s = \dom f \), in such a way that forms a partial permutation of \( X \), where the extra elements of the domain of this partial permutation lie in a given set \( t \subseteq X \setminus (s \cup f''s) \).
To do this, we require the hypothesis
\[ |s \symmdiff f''s| \cdot \aleph_0 \leq |t| \]
Consider the diagram
\[\begin{tikzcd}[column sep=small]
	\cdots & {L_2} & {L_1} & {L_0} & {s \setminus f''s} & \cdots & {f''s \setminus s} & {R_0} & {R_1} & {R_2} & \cdots
	\arrow[from=1-1, to=1-2]
	\arrow[from=1-2, to=1-3]
	\arrow[from=1-3, to=1-4]
	\arrow[from=1-4, to=1-5]
	\arrow["f", from=1-5, to=1-6]
	\arrow["f", from=1-6, to=1-7]
	\arrow[from=1-7, to=1-8]
	\arrow[from=1-8, to=1-9]
	\arrow[from=1-9, to=1-10]
	\arrow[from=1-10, to=1-11]
\end{tikzcd}\]
To fill in the orbits of \( f \), we construct a sequence of disjoint subsets of \( X \) called \( L_n \) and \( R_n \), where for each \( i : \mathbb N \), the cardinality of \( L_i \) is the same as \( s \setminus f''s \), and the cardinality of \( R_i \) is that of \( f''s \setminus s \).
There are natural bijections along this diagram, mapping \( L_{n + 1} \) to \( L_n \) and \( R_n \) to \( R_{n + 1} \).
There are also bijections \( f '' s \setminus s \to R_0 \) and \( L_0 \to s \setminus f '' s \).
This completes the diagram, giving a partial permutation extending \( f \) defined on
\[ s \cup f '' s \cup \left( \bigcup_{i \in \mathbb N} L_i \right) \cup \left( \bigcup_{i \in \mathbb N} R_i \right) \]

\subsection{Hypotheses}
\label{ss:foa:hypotheses}

Let \( \alpha \) be a proper type index.
The complete list of hypotheses we will use when proving the freedom of action theorem at all levels \( \beta \leq \alpha \) are as follows.
\begin{enumerate}
    \item We assume that we have tangle data and typed near-litters at all proper type levels \( \beta \leq \alpha \).
    The data for level \( \alpha \) will later be provided by running the construction in \cref{sec:construction}.
    We also assume position functions at all proper type levels \( \beta < \alpha \).
    \item We assume that the groups of allowable permutations \( A_\beta \) can be related by the use of a derivative map.
    Each path \( A : \beta \rightsquigarrow \gamma \) gives rise to a group homomorphism \( A_\beta \to A_\gamma \), and its action on an allowable \( \rho \) is denoted \( \rho_A \).
    We assume that the following diagram commutes.
    \[\begin{tikzcd}
        {A_\beta} & {S_\beta} \\
        {A_\gamma} & {S_\gamma}
        \arrow[from=1-1, to=1-2]
        \arrow["{\pi \mapsto \pi_A}", from=1-2, to=2-2]
        \arrow["{\rho \mapsto \rho_A}"', from=1-1, to=2-1]
        \arrow[from=2-1, to=2-2]
    \end{tikzcd}\]
    Alternatively, viewing allowable permutations as subgroups of structural permutations, this assumption states that any derivative of an allowable permutation is also allowable.
    \item Let \( \beta < \alpha \) be a proper type index and let \( t \in \tau_\beta \).
    Let \( \gamma < \alpha \) be a type index, let \( \delta < \alpha \) be a proper type index distinct from \( \gamma \), and let \( s \in \tau_\gamma \).
    Suppose that \( (A, x) \in \ran \supp t \), where \( A \) is a \( \beta \)-extended type index and \( x \) is an atom or a near-litter.
    Suppose further that either \( x = a \) is an atom with \( a^\circ = f_{\gamma,\delta}(s) \), or that \( x = N \) is a near-litter that has nonempty intersection with \( \LS(f_{\gamma,\delta}(s)) \).
    Then \( \iota_\gamma(s) < \iota_\beta(t) \).
    \item Let \( \beta \leq \alpha \) be a type index, let \( \gamma < \alpha \) be a type index, and let \( \delta < \alpha \) be a proper type index distinct from \( \gamma \).
    Let \( \rho \) be a \( \beta \)-allowable permutation, and let \( t \) be a \( \gamma \)-tangle.
    Then
    \[ \rho_{(\beta > \delta > \bot)}(f_{\gamma,\delta}(t)) = f_{\gamma,\delta}(\rho_{(\beta > \gamma)}(t)) \]
    This is a restatement of equation (\( \ast \)) from \cref{def:allowable}, and is sometimes called the \emph{coherence condition} for allowable permutations.
    \item Conversely, let \( \beta \leq \alpha \) be a type index, and let \( \rho_{(\beta > \gamma)} \) be a \( \gamma \)-allowable permutation for each type index \( \gamma < \beta \).
    Suppose that the \( \rho_{(\beta > \gamma)} \) satisfy the one-step fuzz constraints
    \[ (\rho_{(\gamma >\delta)})_{(\delta > \bot)}(f_{\gamma,\delta}(t)) = f_{\gamma,\delta}(\rho_{(\beta > \gamma)}(t)) \]
    Then we can assemble the \( \rho_{(\beta > \gamma)} \) into a \( \beta \)-allowable permutation; more precisely, there exists an allowable permutation \( \rho \) with one-step derivatives given by the \( \rho_{(\beta > \gamma)} \).
\end{enumerate}

\begin{remarks}\mbox{\negthinspace}
    \begin{enumerate}
        \item Apart from (iii), these conditions are obviously satisfied by our construction in \cref{sec:construction}.
        Note that this remaining condition only constrains type indices strictly below \( \alpha \), so we may carry this knowledge through the main inductive step.
        Once we know that a given type is of size \( \mu \), a position function can be constructed in such a way to make (iii) hold, using a similar method to the construction of the fuzz map in \cref{def:fuzz}.
        \item Hypothesis (v) will be used exactly once, at the end of freedom of action.
        This will allow us to build the allowable permutation asserted in the theorem.
    \end{enumerate}
\end{remarks}

\subsection{Dependency of addresses}

As described in \cref{ss:foa:outline}, we will define a well-founded relation on the set of \( \alpha \)-addresses that will allow us to define the action of an allowable permutation recursively.

\begin{definition}
    \label{def:dependency}
    Let \( \beta \leq \alpha \) be a proper type index.
    We define the \emph{dependency} relation \( \prec \) on \( \beta \)-support conditions by the following constructors.
    \begin{enumerate}
        \item Let \( A \) be a \( \beta \)-extended index, and let \( a \) be an atom.
        Then
        \[ (A, \NL(a^\circ)) \prec (A, a) \]
        \item Let \( A \) be a \( \beta \)-extended index, and let \( N \) be a near-litter with \( N \neq \NL(N^\circ) \).
        Then
        \[ (A, \NL(N^\circ)) \prec (A, N) \]
        \item Let \( A \) be a \( \beta \)-extended index, let \( N \) be a near-litter, and let \( a \) be an atom in the symmetric difference \( N \symmdiff \LS(N^\circ) \).
        Then
        \[ (A, a) \prec (A, N) \]
        \item Let \( \gamma \leq \alpha \) be a proper type index, and let \( \delta, \varepsilon < \gamma \) be distinct proper type indices.
        Let \( A \) be a path \( \beta \rightsquigarrow \gamma \) and let \( t \in \tau_\delta \).
        Then for all \( c \in \ran \supp t \),
        \[ (A \pcomp (\gamma > \delta) \pcomp \pr_1(c), \pr_2(c)) \prec (A \pcomp (\gamma > \varepsilon > \bot), \NL(f_{\delta,\varepsilon}(t))) \]
        \item Let \( \gamma \leq \alpha \) be a proper type index, and let \( \varepsilon < \gamma \) be a proper type index.
        Let \( A \) be a path \( \beta \rightsquigarrow \gamma \) and let \( a \) be an atom.
        Then
        \[ (A \pcomp (\gamma > \bot), a) \prec (A \pcomp (\gamma > \varepsilon > \bot), \NL(f_{\bot,\varepsilon}(a))) \]
    \end{enumerate}
    If \( x \prec y \), we say that \( y \) \emph{depends on} \( x \).
\end{definition}

\begin{remarks}\mbox{\negthinspace}
    \begin{enumerate}
        \item Let \( x \) be an address and \( \rho \) be an allowable permutation we wish to study, and suppose we know the values of \( \rho(y) \) for \( y \prec x \).
        We can then deduce some useful information about the possible values of \( \rho(x) \), and in some cases determine it exactly (for instance, if \( x \) is a near-litter \( N \neq \NL(N^\circ) \)).
        \item Each address depends on only a small set of other addresses.
        This holds as supports have small ranges.
        \item A litter is said to be \emph{inflexible} with respect to a particular \( \beta \)-extended index if it is the image of a fuzz map as given by constructors (iv) or (v) above.
        A litter is \emph{flexible} with respect to an extended index if it is not inflexible.
        The addresses formed from flexible litters are minimal elements under \( \prec \).
        It is technically possible for an address formed from an inflexible litter to be minimal under \( \prec \) if its support is empty, but aside from this pathology, this is a full description of the minimal elements.
    \end{enumerate}
\end{remarks}

\begin{proposition}
    The dependency relation is well-founded.
\end{proposition}
\begin{proof}
    If \( X \) is a set endowed with a relation \( \prec \), we say that \( x \in X \) is \emph{\( \prec \)-accessible} if all of its \( \prec \)-predecessors are \( \prec \)-accessible.
    A relation on \( X \) is well-founded precisely when every \( x \in X \) is accessible under that relation.
    We make the following observations about accessibility of addresses under the dependency relation, both of which follow directly from the relevant definitions.
    \begin{enumerate}
        \item Let \( A \) be a \( \beta \)-extended type index, and let \( a \) be an atom.
        Then \( (A, a) \) is accessible if \( (A, \NL(a^\circ)) \) is accessible.
        \item Let \( A \) be a \( \beta \)-extended type index, and let \( N \) be an near-litter.
        Then \( (A, N) \) is accessible if \( (A, \NL(a^\circ)) \) is accessible for all \( a \in N \).
    \end{enumerate}
    We have thus reduced accessibility of arbitrary addresses to accessibility of litter addresses.
    Motivated by this, we will define an auxiliary relation \( R \) on litters and prove that it is well-founded.
    We define the following constructors for \( R \).
    \begin{enumerate}
        \item Let \( \delta, \varepsilon < \alpha \) be distinct proper type indices, and let \( t \in \tau_\delta \).
        Then if \( B \) is an extended index and \( a \) is an atom such that either \( (B, a) \in \ran \supp t \) or \( (B, N) \in \ran \supp t \) and \( a \in N \), then
        \[ a^\circ \mathrel{R} f_{\delta,\varepsilon}(t) \]
        \item Let \( \varepsilon < \alpha \) be a proper type index, and let \( a \) be an atom.
        Then
        \[ a^\circ \mathrel{R} f_{\bot,\varepsilon}(a) \]
    \end{enumerate}
    We will say that a litter \( L \) has position \( \nu \in \mu \) if either
    \begin{enumerate}
        \item \( \delta, \varepsilon < \alpha \) are distinct proper type indices, \( t \in \tau_\delta \), and
        \[ L = f_{\delta,\varepsilon}(t); \quad \nu = \iota(t) \]
        or
        \item \( \varepsilon < \alpha \), \( a \) is an atom, and
        \[ L = f_{\bot,\varepsilon}(a); \quad \nu = \iota(a) \]
    \end{enumerate}
    First, note that a litter has at most one position; this follows by injectivity of the fuzz map.
    It follows by definition that if \( L_1 \mathrel{R} L_2 \), then \( L_2 \) has a position.
    Moreover, if additionally \( L_1 \) has a position, its position is less than that of \( L_2 \).
    This last fact follows from part (iii) of the definition of the fuzz map in \cref{def:fuzz}, as well as hypothesis (iii) in \cref{ss:foa:hypotheses}.
    In particular, as the order on \( \mu \) is well-founded, the relation \( R \) must also be well-founded.

    We show by induction on \( R \) that for every litter \( L \) and \( \beta \)-extended index \( A \), the address \( (A, \NL(L)) \) is accessible; the result then follows from the observations above, since they concern only litter addresses.
    Suppose \( c \prec (A, \NL(L)) \); we must show \( c \) is accessible.
    Only cases (iv) and (v) of \cref{def:dependency} can possibly occur.
    Suppose \( c = (B, a) \) and \( L = f_{\delta,\varepsilon}(t) \).
    Then by observation (i), it suffices to show \( (B, \NL(a^\circ)) \) is accessible, but this is given by the inductive hypothesis.
    If \( c = (B, N) \) and \( L = f_{\delta,\varepsilon}(t) \), then the result similarly holds by applying observation (ii) to the inductive hypothesis.
    Finally, if \( c = (B, a) \) and \( L = f_{\bot,\varepsilon}(a) \), then again the result holds by observation (i) and the inductive hypothesis.
\end{proof}

\begin{remark}
    If \( x \prec y \) are \( \gamma \)-addresses and \( B \) is a path \( \beta \rightsquigarrow \gamma \), then
    \[ (B \pcomp \pr_1(x), \pr_2(x)) \prec (B \pcomp \pr_1(y), \pr_2(y)) \]
\end{remark}

We will define a partial order on addresses by defining \( < \) to be the transitive closure of \( \prec \).
Evidently, the relation \( < \) is also well-founded.

\subsection{Approximations and the theorem statement}

We can now define one notion of a partially defined map of addresses.

\begin{definition}
    A \emph{near-litter approximation} is a pair \( \varphi = (\varphi^A, \varphi^L) \) where \( \varphi^A \) is a partial permutation of atoms, \( \varphi^L \) is a partial permutation of litters, and the domain of \( \varphi^A \) contains only a small number of atoms associated to a given litter \( L \).
\end{definition}

For brevity, we will often suppress the superscripts on approximations.

\begin{definition}
    An atom \( a \) is said to be \emph{exceptional} in a near-litter permutation \( \pi \) if \[ (\pi(a))^\circ \neq \pi(a^\circ) \quad\text{or}\quad (\pi^{-1}(a))^\circ \neq \pi^{-1}(a^\circ) \]
\end{definition}
\begin{definition}
    A near-litter approximation \( \varphi \) is said to \emph{approximate} a near-litter permutation \( \pi \) when they agree at all atoms and litters at which \( \varphi \) is defined:
    \[ \forall a \in \dom \varphi^A.\, \pi(a) = \varphi(a);\quad \forall L \in \dom \varphi^L.\, \pi(L) = \varphi(L) \]
    and additionally that all exceptional atoms in \( \pi \) are in \( \dom \varphi^A \).
\end{definition}

\begin{remark}
    In \cite{holmes2023nf}, Holmes makes a distinction between merely `approximating' a permutation and `exactly approximating' it.
    We make no use of the weaker notion, so we will refer to `exactly approximating' as simply `approximating'.
\end{remark}

\begin{remark}
    If \( \varphi \) approximates \( \pi \), we can determine the exact image of \( \pi(\NL(L)) \) for any litter \( L \in \dom \varphi^L \) from data in \( \varphi \).
    It can be shown directly that every near-litter approximation approximates some near-litter permutation, although we will not need this fact.
\end{remark}

In the same way that structural permutations were built up from near-litter permutations in \cref{def:struct_perm}, we will build up structural approximations from near-litter approximations.

\begin{definition}
    Let \( \beta \) be a type index.
    A \emph{\( \beta \)-structural approximation} \( \varphi \) assigns a near-litter approximation to each \( \beta \)-extended index.
    For a given \( \beta \)-extended index \( A \), the assigned near-litter permutation is denoted \( \varphi_A \).
    A \( \beta \)-structural approximation \( \varphi \) is said to \emph{(exactly) approximate} a \( \beta \)-structural permutation \( \pi \) if \( \varphi_A \) (exactly) approximates \( \pi_A \) for all \( \beta \)-extended indices \( A \).
    We say that \( \beta \) is \emph{free} if for all \( \beta \)-extended indices \( A \), the only litters that appear in \( \dom (\varphi_A)^L \) are \( A \)-flexible.
\end{definition}

We can now state one form of the freedom of action theorem.
\Cref{thm:foa_behaviour} is a slightly weaker statement of the same result, but is often more useful in practice.

\begin{theorem}[freedom of action]
    \label{thm:foa}
    Let \( \beta \leq \alpha \) be a proper type index.
    Then any free \( \beta \)-structural approximation \( \varphi \) approximates some \( \beta \)-allowable permutation \( \rho \).
\end{theorem}

\begin{remarks}\mbox{\negthinspace}
    \begin{enumerate}
        \item This theorem extends the fact that any near-litter approximation approximates some near-litter permutation; in fact, this statement is precisely freedom of action for \( \beta = \bot \).
        \item If \( L \) is \( A \)-inflexible, then the value of \( \rho_A(L) \) depends on a the value of a certain tangle under \( \rho \), so we may not choose it freely.
        The freedom of action theorem states that this is the only restriction to the images of litters under allowable permutations.
        \item The theorem is phrased in terms of approximations, although this is somewhat arbitrary.
        The purpose of the theorem is to construct allowable permutations, but the data we need to encapsulate in an allowable permutation may come in various different forms.
        \item As described in \cref{ss:outline:size}, we will later use the freedom of action theorem to show that the set of \( \alpha \)-objects has cardinality exactly \( \mu \).
    \end{enumerate}
\end{remarks}

\subsection{Actions}

We will now introduce another form of data that can be used as the input to the freedom of action theorem.

\begin{definition}
    A \emph{near-litter action} is a pair \( \psi = (\psi^A, \psi^L) \) where \( \psi^A \) is a partial function from the set of atoms to itself, \( \psi^L \) is a partial function from the set of litters to the set of near-litters, and both \( \psi^A \) and \( \psi^L \) have small domain.
\end{definition}

As with approximations, we will suppress the superscripts when possible.

\begin{remarks}\mbox{\negthinspace}
    \begin{enumerate}
        \item The function \( \psi^L \) encapsulates the mapping \( L \mapsto \pi(\NL(L)) \), whereas for near-litter approximations \( \varphi \), the function \( \varphi^L \) encapsulates \( L \mapsto \pi(L) \).
        In this way, near-litter actions keep track of more data about where a litter is sent under a permutation, since we know precisely where its set of atoms is sent.
        \item The appropriate notion of approximating a permutation here is that
        \[ \forall a \in \dom \psi^A.\, \pi(a) = \psi(a);\quad \forall L \in \dom \psi^L.\, \pi(\NL(L)) = \psi(L) \]
        We need no distinction between approximating and exactly approximating since near-litter actions already capture precise images of litters.
        % TODO: I think this should be in Lean!
        % We can use this idea to define the data format we need for support shenanigans.
        Not every near-litter action approximates some near-litter permutation, because we may have a situation where \( a \in \dom \psi^A, a^\circ \in \dom \psi^L \) but \( \psi(a) \notin \psi(a^\circ) \).
        This pathology, as well as injectivity, turns out to be the only way in which near-litter actions can fail to approximate some near-litter permutation.
        We therefore make the following definition.
    \end{enumerate}
\end{remarks}

\begin{definition}
    A near-litter action \( \psi \) is said to be \emph{lawful} if
    \begin{enumerate}
        \item if \( a_1, a_2 \) are atoms and \( \psi(a_1) = \psi(a_2) \) then \( a_1 = a_2 \);
        \item if \( L_1, L_2 \) are litters and \( \psi(L_1) \cap \psi(L_2) \neq \varnothing \) then \( L_1 = L_2 \);
        \item if \( a \in \dom \psi^A \) and \( L \in \dom \psi^L \), then \( a^\circ = L \) if and only if \( \psi(a) \in \psi(L) \).
    \end{enumerate}
\end{definition}

We now describe a procedure for converting lawful near-litter actions into free near-litter approximations.
This process has two steps: expanding the near-litter action so that it is \emph{precise}, and then completing it into an approximation.

\begin{definition}
    % TODO: this might be wrong, check defs
    \label{def:precise}
    A near-litter action \( \psi \) is said to be \emph{precise} if for all \( L \in \dom \psi^L \),
    \begin{enumerate}
        \item \( \psi(L) \symmdiff \LS(\psi(L)^\circ) \subseteq \ran \psi^A \);
        \item \( \ran \psi^A \cap \LS(L) \subseteq \dom \psi^A \);
        \item \( \dom \psi^A \cap \psi(L) \subseteq \ran \psi^A \).
    \end{enumerate}
\end{definition}

These are precisely the requirements that our completion process needs in order to compute complete orbits for all atoms and litters.

\begin{lemma}
    \label{lem:foa:precise}
    Any lawful near-litter action can be extended to a precise near-litter action with the same map on litters.
\end{lemma}
This proof is explicit, but unenlightening.
\begin{proof}
    Let \( \psi \) be a near-litter action.
    Call a litter \emph{banned} for \( \psi \) if it contains any atom in the domain or range of \( \psi^A \), is in \( \dom \psi^L \), or shares an atom with any near-litter in \( \ran \psi^L \).
    Observe that there are only a small amount of banned litters for a given \( \psi \).
    We will use the unbanned litters to store the extra atoms we need to insert into \( \psi \).

    We will first extend \( \psi \) so that it satisfies condition (i) in \cref{def:precise}.
    Suppose that \( a \in \LS(\psi(L)^\circ) \setminus \psi(L) \); we need to allocate a preimage for \( a \) under \( \psi \).
    Choose an atom \( a' \) in an unbanned litter, and add to \( \psi^A \) that \( \psi(a') = a \).
    The addition of \( a' \) to \( \psi \) does not create more problems with satisfying \cref{def:precise}, since \( a' \) is part of an unbanned litter.
    There are only a small amount of atoms in such circumstances, so they can all be added to the same unbanned litter for convenience.

    Now suppose that \( a \in \psi(L) \setminus \LS(\psi(L)^\circ) \).
    We need to allocate a preimage for \( a \) under \( \psi \) that lies in \( L \).
    Since there are only a small amount of atoms in question, and only a small amount of atoms in \( L \) already appear in \( \dom \psi^A \), we can fit all necessary preimages inside \( L \).
    Again, this does not create any more problematic atoms that we must deal with.
    Performing this process for every \( L \in \dom \psi^L \), we obtain a new near-litter action \( \psi' \) which satisfies condition (i) of \cref{def:precise} and has the same litter map as \( \psi \).
    One can easily check that \( \psi' \) is lawful.

    We now extend \( \psi' \) so that it satisfies conditions (ii) and (iii), while not disturbing condition (i).
    To do this, we need to fill in certain images and preimages of atoms.
    First, use the construction in \cref{ss:foa:partial_perm} to produce a partial permutation of litters \( \pi \) extending \( L \mapsto \psi'(L)^\circ \).
    We require that all extra litters used to make \( \dom \pi \) are unbanned; this can be arranged since we have a choice of \( \mu \) unbanned litters to use.
    We then extend \( \pi \) to be defined on all banned litters by setting \( \pi(L) = L \) in this case; the resulting map is clearly still a partial permutation.
    This map will be used to control the litters inside which the new images and preimages of atoms under \( \psi' \) will be allocated.
    For now, we will describe only the allocation of forward images; backward images work entirely symmetrically.

    Inside each litter, we reserve enough unused atoms to correspond with all possible pairs \( (n, a) \) where \( a \in \ran \psi^A \setminus \dom \psi^A \).
    We then define that an atom \( a \in \ran \psi^A \setminus \dom \psi^A \) is mapped to the atom corresponding to \( (0, a) \) in the litter \( \pi(a^\circ) \).
    If \( a' \) is an atom corresponding to \( (n, a) \) in the litter \( \pi^{n+1}(a^\circ) \), then it is mapped to the atom corresponding to \( (n+1, a) \) in the litter \( \pi^{n+2}(a^\circ) \).
    This behaviour ensures that we our extra atoms do not violate condition (i) of \cref{def:precise}.
    Note that most atoms corresponding to one of the pairs \( (n, a) \) will not have an image defined.
    This completes the description of the forward orbits; after filling the backward orbits analogously, the resulting near-litter action \( \psi'' \) can be seen to be precise and lawful.
\end{proof}

We now describe the method for turning a precise action into a free approximation.

\begin{lemma}
    \label{lem:foa:nl_action_to_approx}
    Let \( \psi \) be a precise lawful near-litter action and let \( A \) be a \( \beta \)-extended index.
    Suppose that \( \psi \) maps \( A \)-flexible litters to near-litters that are near \( A \)-flexible litters.
    Then there is a near-litter approximation \( \varphi \) whose litter map contains only \( A \)-flexible litters, such that if \( \varphi \) approximates \( \pi \), we have
    \begin{enumerate}
        \item if \( a \in \dom \psi^A \), then \( \psi(a) = \pi(a) \);
        \item if \( L \in \dom \psi^L \) is \( A \)-flexible, then \( \psi(L)^\circ = \pi(L) \); and
        \item if \( L \in \dom \psi^L \) and \( \psi(L)^\circ = \pi(L) \), then \( \psi(L) = \pi(\NL(L)) \).
    \end{enumerate}
\end{lemma}
This lemma provides exactly what is needed to use the freedom of action theorem with actions instead of approximations directly.
We will shortly restate this lemma in a more useful form.
\begin{proof}
    First, consider the restriction of \( \psi^L \) to the \( A \)-flexible litters.
    This map can be completed to a partial permutation \( \varphi^L \) as in \cref{ss:foa:partial_perm}, where any added litters are flexible and unbanned; there is a supply of \( \mu \) such litters, so this can always be arranged.
    We similarly complete \( \psi^A \) into a partial permutation \( \varphi^A \); the added atoms are chosen to belong to another unbanned litter.
    This yields a near-litter approximation \( \varphi = (\varphi^A, \varphi^L) \).
    We claim that this satisfies the conclusion of the lemma.
    It is clear that (i) and (ii) hold.
    Part (iii) can be shown by checking all possible cases of where an atom can come from and be sent; this requires the full strength of the hypothesis that \( \psi \) was precise.
\end{proof}

This concludes the main work of this subsection.
We now briefly describe how near-litter actions are composed to form structural actions, and combine \cref{lem:foa:precise,lem:foa:nl_action_to_approx} to make an easy way to apply freedom of action to structural actions.

\begin{definition}
    Let \( \beta \) be a type index.
    A \emph{\( \beta \)-structural action} \( \psi \) assigns a near-litter action to each \( \beta \)-extended index.
    For a given \( \beta \)-extended index \( A \), the assigned near-litter action is denoted \( \psi_A \).
    We say that \( \beta \) is \emph{lawful} if for all \( \beta \)-extended indices \( A \), the near-litter action \( \psi_A \) is lawful.
\end{definition}

\begin{proposition}
    \label{lem:foa:struct_action_to_approx}
    Let \( \psi \) be a lawful \( \beta \)-structural action.
    Suppose that for every \( \beta \)-extended index \( A \), the near-litter action \( \psi_A \) maps \( A \)-flexible litters to near-litters that are near \( A \)-flexible litters.
    Then there is a free \( \beta \)-structural approximation such that if \( \varphi \) approximates \( \pi \), we have
    \begin{enumerate}
        \item if \( a \in \dom{(\psi_A)^A} \), then \( \psi_A(a) = \pi_A(a) \);
        \item if \( L \in \dom{(\psi_A)^L} \) is \( A \)-flexible, then \( \psi_A(L)^\circ = \pi_A(L) \); and
        \item if \( L \in \dom{(\psi_A)^L} \) and \( \psi_A(L)^\circ = \pi_A(L) \), then \( \psi(L) = \pi_A(\NL(L)) \).
    \end{enumerate}
\end{proposition}
\begin{proof}
    Follows directly from \cref{lem:foa:precise,lem:foa:nl_action_to_approx}.
\end{proof}

We will sometimes convert actions to approximations, then convert approximations to allowable permutations using freedom of action.
Even though freedom of action only guarantees nice behaviour to flexible litters, we can show that any two suitably similar allowable permutations created in this way must agree on arbitrary litters.

\begin{definition}
    \label{def:exact}
    Let \( \pi_1, \pi_2 \) be \( \beta \)-structural permutations, and let \( x \) be a \( \beta \)-address.
    We say that \( \pi_1 \) and \( \pi_2 \) are \emph{exact before \( x \)} if:
    \begin{enumerate}
        \item for all \( (A, a) \leq x \), we have \( (\pi_1)_A(a) = (\pi_2)_A(a) \);
        \item for all \( (A, \NL(L)) \leq x \) where \( L \) is \( A \)-flexible, we have \( (\pi_1)_A(L) = (\pi_2)_A(L) \);
        \item for all \( (A, \NL(L)) \leq x \), if \( (\pi_1)_A(L) = (\pi_2)_A(L) \), then \( (\pi_1)_A(\NL(L)) = (\pi_2)_A(\NL(L)) \).
    \end{enumerate}
\end{definition}

\begin{lemma}
    \label{lem:exact}
    Let \( \rho_1, \rho_2 \) be \( \beta \)-allowable permutations.
    If \( \rho_1, \rho_2 \) are exact before \( x \), then \( \rho_1(x) = \rho_2(x) \).
\end{lemma}
\begin{proof}
    We prove this by induction on \( x \) along the dependency relation.
    If \( x = (A, a) \) is an atom address, the conclusion holds by part (i) of \cref{def:exact}.
    If \( x = (A, N) \) is a near-litter address with \( N \neq \NL(N^\circ) \), we are done immediately by applying the inductive hypothesis.
    If \( x = (A, \NL(L)) \) where \( L \) is a litter, we use part (iii) of \cref{def:exact} to reduce to proving \( (\rho_1)_A(L) = (\rho_2)_A(L) \).
    If \( L \) is \( A \)-flexible, the result holds by (ii), and if \( L \) is \( A \)-inflexible, the result holds by the coherence condition on allowable permutations.
\end{proof}

\subsection{Building the permutation}
\label{ss:foa:construction}

We now begin to prove the freedom of action theorem.
Fix a proper type index \( \beta \leq \alpha \) and a free \( \beta \)-structural approximation \( \varphi \).
We may assume by induction on \( \lambda \) that the freedom of action holds for all proper type indices \( \gamma < \beta \); we will use this in \cref{ss:foa:inflexible_coe}.

We will construct the functions \( \rho^\star_A \) which map atoms to atoms, litters to litters, and near-litters to near-litters, with the intent that they will be the derivatives of an allowable permutation \( \rho \).
We will proceed by recursion along the transitive closure of the dependency relation from \cref{def:dependency}; at each stage \( (A, x) \), we assume that we know the value of \( \rho^\star_B(y) \) for all \( (B, y) < (A, x) \), and compute \( \rho^\star_A(x) \).
This construction will take a different form depending on what kind of address \( (A, x) \) is.
In particular, if \( x = \NL(L) \) is a near-litter that came from a litter, we will break the derivation into two stages: we will first compute \( \rho^\star_A(L) \), and then calculate \( \rho^\star_A(\NL(L)) \) separately.

\subsubsection{Atoms}

Suppose we are at stage \( (A, a) \) where \( a \) is an atom.
If \( a \) was already assigned by \( \varphi \), we simply define
\[ \rho^\star_A(a) = \varphi_A(a) \]
Suppose \( a \notin \dom (\varphi_A)^A \).
For each extended index \( A \) and litter \( L \), the set \( L^- = L \setminus \dom (\varphi_A)^A \) is a near-litter, near the litter \( L \).
For each pair of litters \( L_1, L_2 \), we choose in advance a bijection \( \pi_{L_1,L_2} \) between \( L_1^- \) and \( L_2^- \).
We use this bijection now to define
\[ \rho^\star_A(a) = \pi_{a^\circ,\rho^\star_A(a^\circ)}(a) \]

\subsubsection{Flexible litters}
\label{ss:foa:flexible}

Suppose we are at stage \( (A, \NL(L)) \) where \( L \) is \( A \)-flexible.
If \( L \in \dom (\varphi_A)^L \), then we must define
\[ \rho^\star_A(L) = \varphi_A(L)^\circ \]
Otherwise, we will define that
\[ \rho^\star_A(L) = L \]

\subsubsection{Inflexible litters from proper levels}
\label{ss:foa:inflexible_coe}

Suppose we are at stage
\[ x = (A \pcomp (\gamma > \varepsilon > \bot), \NL(f_{\delta,\varepsilon}(t))) \]
where \( \delta, \varepsilon < \gamma \) are distinct proper type indices, \( A \) is a path \( \beta \rightsquigarrow \gamma \), and \( t \in \tau_\delta \).
We need to enforce the coherence condition that
\[ \rho_{(A \pcomp (\gamma > \varepsilon > \bot))}(f_{\delta,\varepsilon}(t)) = f_{\delta,\varepsilon}(\rho_{(A \pcomp (\gamma > \delta))}(t)) \]
Thus, we need to compute the right-hand side of this equation.
However, we do not know if \( \rho^\star \) happens to come from an allowable permutation, so there is no way to apply it to an arbitrary tangle \( t \).
To deal with this, we invoke the freedom of action theorem at level \( \delta \) to produce a new allowable permutation \( \rho' \) with the same action on \( t \) as \( \rho_{(A \pcomp (\gamma > \delta))} \) will eventually be shown to have.

For a given small set of addresses \( s \), there is a structural action \( \psi^s \) with
\[ a \in \dom (\psi^s_A)^A \iff (A, a) \in s;\quad L \in \dom (\psi^s_A)^L \iff (A, \NL(L)) \in s \]
given by \( \rho^\star \) on its domain.
In particular, we can define
\[ \psi^x = \psi^s;\quad s = \{ y \mid y < x \} \]
We will consider its derivative \( \psi^x_{(A \pcomp (\gamma > \delta))} \), which is a \( \delta \)-strutural action.
Outside of this recursion, we will be able to show that \( \psi^x_{(A \pcomp (\gamma > \delta))} \) is lawful and maps flexible litters to flexible litters as required for \cref{lem:foa:struct_action_to_approx}; for now, if it turned out not to be lawful, we would simply assign
\[ \rho^\star_{A \pcomp (\gamma > \varepsilon > \bot)}(f_{\delta,\varepsilon}(t)) = f_{\delta,\varepsilon}(t) \]
but this case will never actually occur.
Build a \( \delta \)-structural approximation from \( \psi^x_{(A \pcomp (\gamma > \delta))} \) as in \cref{lem:foa:struct_action_to_approx}, and apply the freedom of action theorem at level \( \delta < \gamma \leq \beta \) to obtain a \( \delta \)-allowable permutation \( \rho^x \).
We will later be able to show that \( \rho^x \) agrees with \( \rho^\star \) on all elements of the support of \( t \).
We set
\[ \rho^\star_{A \pcomp (\gamma > \varepsilon > \bot)}(f_{\delta,\varepsilon}(t)) = f_{\delta,\varepsilon}(\rho^x(t)) \]

\subsubsection{Inflexible litters from the base level}
\label{ss:foa:inflexible_bot}

Suppose we are at stage
\[ x = (A \pcomp (\gamma > \varepsilon > \bot), \NL(f_{\bot,\varepsilon}(a))) \]
where \( \varepsilon < \gamma \) is a proper type index, \( A \) is a path \( \beta \rightsquigarrow \gamma \), and \( a \) is an atom.
We must enforce
\[ \rho_{(A \pcomp (\gamma > \varepsilon > \bot))}(f_{\bot,\varepsilon}(a)) = f_{\bot,\varepsilon}(\rho_{(A \pcomp (\gamma > \bot))}(a)) \]
so we have no choice but to define
\[ \rho^\star_{A \pcomp (\gamma > \varepsilon > \bot)}(f_{\bot,\varepsilon}(a)) = f_{\bot,\varepsilon}(\rho^\star_{(A \pcomp (\gamma > \bot))}(a)) \]

\subsubsection{Near-litters}

Suppose we are at stage \( (A, N) \) where \( N \) is a near-litter.
Either by inductive hypothesis or by \cref{ss:foa:flexible,ss:foa:inflexible_coe,ss:foa:inflexible_bot}, we know the value of \( \rho^\star_A(N^\circ) \).
We then define
\[ \pr_2 (\rho^\star_A(N)) = \left( \rho^\star_A(N^\circ)^- \cup {\psi_A} '' \left(N \cap \dom{(\psi_A)^A}\right) \right) \symmdiff \left\{ \rho^\star_A(a) \mid a \in (N \symmdiff \LS(N^\circ)) \setminus \dom{(\psi_A)^A} \right\} \]
In the case that \( N = \NL(L) \), this simplifies to
\[ \pr_2 (\rho^\star_A(\NL(L))) = \rho^\star_A(L)^- \cup {\psi_A} '' \left(\LS(L) \cap \dom{(\psi_A)^A}\right) \]
This definition has the property that
\[ \rho^\star_A(N) = \rho^\star_A(\NL(N^\circ)) \symmdiff {\rho^\star_A} '' (N \symmdiff \LS(N^\circ)) \]
and that for any two distinct litters \( L_1, L_2 \), the images \( \rho^\star_A(\NL(L_1)) \) and \( \rho^\star_A(\NL(L_2)) \) are disjoint.

\subsection{Injectivity}

We have fully described \( \rho^\star \); we will now show that the \( \rho^\star_A \) are near-litter permutations.
The first step in this process is proving injectivity of the \( \rho^\star_A \).

In particular, for a given pair of addresses \( x, y \), we will write
\[ \psi^{x,y} = \psi^s;\quad s = \{z \mid z \leq x \text{ or } z \leq y\} \]
We will prove injectivity by induction along two addresses at once, and our inductive hypothesis will be that \( \psi^{x,y} \) is lawful.
Given this hypothesis, we can immediately show that injectivity of atoms extends to the current stage.

\begin{lemma}
    \label{lem:atom_injective}
    Suppose \( \psi^{x,y} \) is lawful.
    Let \( A \) be an extended index, and let \( a, b \) be atoms such that \( (A, a) \leq x \) or \( y \), and \( (A, b) \leq x \) or \( y \).
    Then if \( \rho^\star_A(a) = \rho^\star_A(b) \), we have \( a = b \).
\end{lemma}
\begin{proof}
    If both \( a \) and \( b \) lie in \( \dom (\psi_A)^A \), this holds by injectivity of \( \psi_A \).
    If just one lies in the domain, this holds as the image of the other atom lies in \( L^- \) for some litter.
    In the other case, \( \rho^\star_A(a^\circ) = \rho^\star_A(b^\circ) \) so \( a^\circ = b^\circ \) by assumption, so we can reverse the bijection \( \pi_{a^\circ, \rho^\star_A(a^\circ)} = \pi_{b^\circ, \rho^\star_A(b^\circ)} \) to show \( a = b \).
\end{proof}

\begin{lemma}[coherence lemma]
    \label{lem:coherence}
    Let \( A \) be a path \( \beta \rightsquigarrow \gamma \) where \( \gamma \) is a proper type index, and let \( B \) be a \( \gamma \)-extended type index.
    Let \( s \) be a small set of \( \beta \)-addresses such that \( \psi^s \) is lawful, and let \( \psi^s_A \) exactly approximate the \( \gamma \)-allowable permutation \( \rho \) (under the conversion to an approximation given in \cref{lem:foa:struct_action_to_approx}).
    Let \( x \) be an address such that \( (A \pcomp \pr_1 x, \pr_2 x) \leq y \) for some \( y \in s \).
    Then
    \[ \rho^\star_A(x) = \rho(x) \]
\end{lemma}
\begin{proof}
    We proceed by induction along the relation \( < \) for \( \gamma \)-addresses.
    The result for atom addresses is simple, and the result for near-litter addresses reduces to that for litter addresses.
    It suffices to show, for every \( \gamma \)-extended index \( B \) and litter \( L \) such that \( (A \pcomp B, \NL(L)) \leq y \) for some \( y \in s \),
    \[ \rho^\star_{A \pcomp B}(L) = \rho_B(L) \]
    The case where \( L \) is \( B \)-flexible is easy, as is the case where \( L = f_{\bot,\zeta}(a) \).
    Suppose that
    \[ L = f_{\varepsilon,\zeta}(t);\quad t \in \tau_\varepsilon;\quad B = C \pcomp (\delta > \zeta > \bot) \]
    The coherence condition gives
    \[ \rho_{C \pcomp (\delta > \zeta > \bot)}(f_{\varepsilon,\zeta}(t)) = f_{\varepsilon,\zeta}(\rho_{C \pcomp (\delta > \varepsilon)}(t)) \]
    By the inductive hypothesis, \( \psi^s \) is lawful, so the structural action \( \psi^x \) defined in \cref{ss:foa:inflexible_coe} is also lawful.
    Therefore, by construction,
    \[ \rho^\star_{A \pcomp C \pcomp (\delta > \zeta > \bot)}(f_{\varepsilon,\zeta}(t)) = f_{\varepsilon,\zeta}(\rho^x(t)) \]
    where \( \rho^x \) is an \( \varepsilon \)-allowable permutation made from \( \psi^x \).
    Thus, it suffices to show that
    \[ \rho_{C \pcomp (\delta > \varepsilon)}(t) = \rho^x(t) \]
    We show that both of these \( \varepsilon \)-allowable permutations have the same action on the support of \( t \).
    It suffices by \cref{lem:exact} to show that they are exact before all addresses in this support, and this simply follows from the fact that both \( \rho \) and \( \rho^x \) came from applying \cref{lem:foa:struct_action_to_approx} to \( \rho^\star \).
\end{proof}

\begin{corollary}
    \label{cor:foa_both}
    Suppose that \( \psi^{x,y} \) is lawful.
    Let \( A \) be a path \( \beta \rightsquigarrow \gamma \) and let \( t \in \tau_\delta \).
    Let \( L = f_{\delta,\varepsilon}(t) \) be a litter such that \( (A, \NL(L)) \leq x \).
    Then
    \[ \rho^x_{(A \pcomp (\gamma > \delta))}(t) = \rho^{x,y}_{(A \pcomp (\gamma > \delta))}(t) \]
    where \( \rho^s \) is the allowable permutation obtained from applying freedom of action to \( \psi^s \) as in \cref{ss:foa:inflexible_coe}.
\end{corollary}
This follows directly from applying the coherence lemma to each address in the support of \( t \).

\begin{lemma}
    \label{lem:litter_injective}
    Suppose \( \psi^{x,y} \) is lawful.
    Let \( A \) be an extended index, and let \( L_1, L_2 \) be litters such that \( (A, \NL(L_1)) \leq x \) or \( y \), and \( (A, \NL(L_2)) \leq x \) or \( y \).
    Then if \( \rho^\star_A(L_1) = \rho^\star_A(L_2) \), we have \( L_1 = L_2 \).
\end{lemma}
\begin{proof}
    Note that \( \rho^\star_A \) maps \( A \)-flexible litters to \( A \)-flexible litters, and the same holds for \( A \)-inflexible litters.
    Thus, \( L_1 \) is \( A \)-flexible precisely when \( L_2 \) is.
    If \( L_1 \) and \( L_2 \) are \( A \)-flexible, then the values of \( \rho^\star_A(L_1) \) and \( \rho^\star_A(L_2) \) are given by applying the same partial permutation as defined in \cref{ss:foa:flexible}, so we must have \( L_1 = L_2 \).

    If \( L_i = f_{\bot,\varepsilon}(a_i) \) and \( A = B \pcomp (\gamma > \varepsilon > \bot) \), then by \cref{ss:foa:inflexible_bot},
    \[ \rho^\star_A(L_i) = f_{\bot,\varepsilon}(\rho^\star_{(A \pcomp (\gamma > \bot))}(a_i)) \]
    Hence, by injectivity of the fuzz map and lawfulness of \( \psi^{x,y} \), we must have \( a_1 = a_2 \), giving the result.

    Now suppose \( L_i = f_{\delta,\varepsilon}(t_i) \) and \( A = B \pcomp (\gamma > \varepsilon > \bot) \).
    \[ \rho^\star_A(L_i) = f_{\delta,\varepsilon}(\rho^z_{(A \pcomp (\gamma > \delta))}(t_i)) \]
    where \( z \) is either \( x \) or \( y \).
    By \cref{cor:foa_both}, we obtain
    \[ f_{\delta,\varepsilon}(\rho^{x,y}_{(A \pcomp (\gamma > \delta))}(t_1)) = f_{\delta,\varepsilon}(\rho^{x,y}_{(A \pcomp (\gamma > \delta))}(t_2)) \]
    and thus \( t_1 = t_2 \).
\end{proof}

\begin{lemma}
    All of the \( \psi^{x,y} \) are lawful.
\end{lemma}
\begin{proof}
    We proceed by induction on the relation on pairs of addresses given by
    \begin{alignat*}{5}
        (x_1, y) &\vartriangleleft (x_2, y) && \quad\text{if }\quad & x_1 &\prec x_2 \\
        (x, y_1) &\vartriangleleft (x, y_2) && \quad\text{if }\quad & y_1 &\prec y_2 \\
        (z_1, z_2) &\vartriangleleft (x, y) && \quad\text{if }\quad & z_1, z_2 &\prec x \\
        (z_1, z_2) &\vartriangleleft (x, y) && \quad\text{if }\quad & z_1, z_2 &\prec y
    \end{alignat*}
    This is well-founded, and the inductive hypothesis is the precise form needed to show lawfulness from \cref{lem:atom_injective,lem:litter_injective}.
    Part (iii) of the definition of lawfulness requires some simple explicit checking.
\end{proof}

\begin{corollary}
    The \( \rho^\star_A \) maps are injective on atoms and litters, and atoms inside litters are mapped inside the corresponding image near-litter.
\end{corollary}
\begin{corollary}
    All of the \( \psi^s \) are lawful.
\end{corollary}

\subsection{Surjectivity}

As with injectivity, the result for atoms is easy to prove.

\begin{lemma}
    Suppose \( A \) is a \( \beta \)-extended index and \( a \) is an atom such that \( a^\circ \in \ran \rho^\star_A \).
    Then \( a \in \ran \rho^\star_A \).
\end{lemma}
\begin{proof}
    Either \( a \in \dom (\psi_A)^A \), in which case \( a \in \ran (\psi_A)^A \) as required, or \( a \in (a^\circ)^- \), in which case the result holds.
\end{proof}

We will prove the result for near-litters from the following result on litters.

\begin{lemma}
    Let \( A \) be a \( \beta \)-extended index and let \( L \) be a litter.
    Suppose that for all \( x \prec (A, \NL(L)) \), we have \( \pr_2 x \in \ran \rho^\star_{\pr_1 x} \).
    Then \( L \in \ran \rho^\star_A \).
\end{lemma}
\begin{proof}
    If \( L \) is \( A \)-flexible or \( L = f_{\bot,\varepsilon}(a) \), the result is clear.
    Suppose \( A = B \pcomp (\gamma > \varepsilon > \bot) \) and \( L = f_{\delta,\varepsilon}(t) \).
    We claim that
    \[ L = \rho^\star_A(f_{\delta,\varepsilon}((\rho^s)^{-1}(t))) \]
    where
    \[ s = {(\rho^\star)^{-1}} '' \{ d \mid d < (A, L) \} \]
    Note that \( s \) is small as \( \rho^\star \) is injective on addresses.
    Let \( \rho^x \) be the allowable permutation obtained in \cref{ss:foa:inflexible_coe}, so
    \[ \rho^\star_A\left(f_{\delta,\varepsilon}\left((\rho^s_{B \pcomp (\gamma > \delta)})^{-1}(t)\right)\right) = f_{\delta,\varepsilon}\left(\rho^x\left((\rho^s_{B \pcomp (\gamma > \delta)})^{-1}(t)\right)\right) \]
    We show that
    \[ (\rho^x)^{-1}(t) = \left(\rho^s_{B \pcomp (\gamma > \delta)}\right)^{-1}(t) \]
    from which we obtain the desired result.
    It suffices to check the action on addresses \( y \in \supp t \):
    \[ (\rho^x)^{-1}(y) = \left(\rho^s_{B \pcomp (\gamma > \delta)}\right)^{-1}(y) \]
    By assumption, \( y = \rho^\star_{B \pcomp (\gamma > \delta)}(z) \) for some \( z \); it therefore suffices to show that
    \[ \rho^x(z) = \rho^\star_{B \pcomp (\gamma > \delta)}(z) = \rho^s_{B \pcomp (\gamma > \delta)}(z) \]
    The coherence lemma (\cref{lem:coherence}) immediately gives the right-hand equality.
    It also gives the left-hand equality, but with slightly more work; we must show that
    \[ (B \pcomp (\gamma > \delta) \pcomp \pr_1 z, \pr_2 z) < \left(B \pcomp (\gamma > \varepsilon > \bot), f_{\delta,\varepsilon}\left((\rho^s_{B \pcomp (\gamma > \delta)})^{-1}(t)\right)\right) \]
    But since allowable permutations commute with supports, we can deduce \( z \in \supp (\rho^s_{B \pcomp (\gamma > \delta)})^{-1}(t) \) from the fact that
    \[ \rho^s_{B \pcomp (\gamma > \delta)}(z) = \rho^\star_{B \pcomp (\gamma > \delta)}(z) = y \in \supp t \]
\end{proof}

\subsection{Finishing the proof}

We can now prove the freedom of action theorem, which states the following.

\begin{theorem*}[freedom of action, restatement of \cref{thm:foa}]
    Let \( \beta \leq \alpha \) be a proper type index.
    Then any free \( \beta \)-structural approximation \( \varphi \) approximates some \( \beta \)-allowable permutation \( \rho \).
\end{theorem*}
\begin{proof}
    We proceed by induction on \( \beta \).
    Let \( \varphi \) be a free \( \beta \)-structural approximation, and let \( \rho^\star \) be the map as defined in \cref{ss:foa:construction}.
    For any path \( A : \beta \rightsquigarrow \gamma \), we will say that \( \rho^\star \) is \emph{allowable below \( A \)} if there is a \( \gamma \)-allowable permutation \( \rho \) such that \( \rho^\star_A \) and \( \rho \) agree.
    We first show that for each \( \beta \)-extended index \( A \), \( \rho^\star \) is allowable below \( A \).
    But a \( \bot \)-allowable permutation is precisely a near-litter permutation, and each \( \rho^\star_A \) is a permutation of atoms and preserves near-litters as required.

    Now, let \( A : \beta \rightsquigarrow \gamma \) where \( \gamma \) is a proper type index.
    We show that \( \rho \) is allowable below \( A \) given that it is allowable below \( A \pcomp (\gamma > \delta) \) for all type indices \( \delta < \gamma \).
    Hypothesis (v) in \cref{ss:foa:hypotheses} allows us to construct a suitable allowable permutation; it suffices to check the coherence condition, which holds by construction.
    Hence, by induction on \( \gamma \), \( \rho^\star \) is allowable below every path, and in particular below the empty path, giving a \( \beta \)-allowable permutation \( \rho \) that coincides with \( \rho^\star \) everywhere.
    One can check that \( \varphi \) exactly aproximates \( \rho \); this is a simple computation.
\end{proof}

\subsection{Corollaries}

While \cref{thm:foa} is comparatively simple to state, the data that we want to combine into an allowable permutation rarely takes the form of a free structural approximation.
In this subsection, we describe a necessary criterion for partial functions of atoms and near-litters to come from an allowable permutation.

\begin{definition}
    A \emph{near-litter behaviour} is a pair \( \xi = (\xi^A, \xi^N) \) where \( \xi^A \) is a partial function from atoms to atoms and \( \xi^N \) is a partial function from near-litters to near-litters, such that the domains of \( \xi^A \) and \( \xi^N \) are small.
    A \emph{\( \beta \)-structural behaviour} assigns to each \( \beta \)-extended index \( A \) a near-litter behaviour \( \xi_A \).
\end{definition}

\begin{definition}
    A near-litter behaviour \( \xi \) \emph{approximates} a near-litter permutation \( \pi \) if \( \xi(a) = \pi(a) \) and \( \xi(N) = \pi(N) \) for all \( a \in \dom \xi^A \) and \( N \in \dom \xi^N \).
    Analogously, a structural behaviour \( \xi \) \emph{approximates} a structural permutation \( \pi \) if \( \xi_A \) approximates \( \pi_A \).
\end{definition}

We will describe a sufficient condition for a structural behaviour to approximate some allowable permutation.

\begin{definition}
    \label{def:foa:beh_lawful}
    A near-litter behaviour \( \xi \) is said to be \emph{lawful} if it satisfies
    \begin{enumerate}
        \item \( \xi^A \) is injective on its domain;
        \item if \( a \in \dom \xi^A \) and \( N \in \dom \xi^N \), then \( a \in N \leftrightarrow \xi(a) \in \xi(N) \);
        \item if \( N_1, N_2 \in \dom \xi^N \) and \( N_1^\circ = N_2^\circ \), then \( N_1 \symmdiff N_2 \subseteq \dom \xi^A \) and \( \xi(N_1) \symmdiff \xi(N_2) \subseteq \ran \xi^A \);
        \item if \( N_1, N_2 \in \dom \xi^N \) and \( N_1^\circ \neq N_2^\circ \), then \( N_1 \cap N_2 \subseteq \dom \xi^A \) and \( \xi(N_1) \cap \xi(N_2) \subseteq \ran \xi^A \).
    \end{enumerate}
    We say that a structural behaviour \( \xi \) is lawful if \( \xi_A \) is lawful for all \( A \).
\end{definition}

\begin{remark}
    From this, we can deduce that for \( N_1, N_2 \in \dom \xi^N \), we have \( N_1^\circ = N_2^\circ \leftrightarrow \xi(N_1)^\circ = \xi(N_2)^\circ \).
    This follows from the fact that \( \dom \xi^A \) is small by considering the cardinality of the relevant intersections.
\end{remark}

We describe a procedure for converting near-litter behaviours into near-litter actions.
This allows us to use \cref{lem:foa:nl_action_to_approx} to turn them into approximations, and then apply freedom of action.

\begin{lemma}
    \label{lem:foa:beh_to_action}
    Any lawful near-litter behaviour \( \xi \) can be extended to one defined on \( \NL(L) \) for all litters \( L = N^\circ \) for \( N \in \dom \xi^N \).
\end{lemma}
\begin{proof}
    We first add the atoms in one of the \( N \symmdiff \LS(N^\circ) \) to \( \xi \); there are only a small amount of these, so the result will be a near-litter behaviour.
    First suppose that \( L \) is a litter and
    \[ a \in \left(\bigcap_{N \in \dom \xi^N, N^\circ = L} N \setminus \LS(L)\right) \setminus \dom \xi^A \]
    We must allocate an image for \( a \) in the set
    \[ \left(\bigcap_{N \in \dom \xi^N, N^\circ = L} \xi(N)\right) \setminus \ran \xi^A \]
    This set is not small, so this allocation can always be done.
    Note that this set depends on the value of \( L \), which is unique for a given \( a \); if it were not, \( a \) would lie in the domain of \( \xi^A \) by part (iv) of \cref{def:foa:beh_lawful}.
    Instead, suppose that
    \[ a \in \left[\bigcup_{N \in \dom \xi^N} \LS(N^\circ) \setminus \bigcup_{N' \in \dom \xi^N} N' \right] \setminus \dom \xi^A \]
    Pick a litter \( L \) such that
    \[ \left(\ran \xi^A \cup \bigcup \ran \xi^N\right) \cap \LS(L) = \varnothing \]
    This can always be done as there are \( \mu \) litters but the amount of litters banned by the left-hand side is small.
    We allocate images for all atoms of this form inside \( \LS(L) \).
    One can check that this operation is well-defined and defines images of atoms \( a \in N \symmdiff \LS(N^\circ) \) for all \( N \in \dom \xi^N \).

    We now define the action of \( \xi \) on a litter \( L \) such that \( L = N^\circ \) for some \( N \in \dom \xi^N \).
    \[ \xi(\NL(L)) = \xi(N) \symmdiff \{ \xi(a) \mid a \in N \symmdiff \LS(N^\circ) \} \]
    This definition does not depend on the choice of \( N \), because for any other \( N' \) such that \( {N'}^\circ = L \), part (iii) of \cref{def:foa:beh_lawful} fixes any problematic atoms.
    One then checks directly that this behaviour is lawful.
\end{proof}

\begin{definition}
    \label{def:foa:beh_coherent}
    A \( \beta \)-structural behaviour \( \xi \) is said to be \emph{coherent} if it satisfies the following conditions.
    \begin{enumerate}
        \item For every extended index \( A : \beta \rightsquigarrow \bot \) and near-litter \( N \in \dom (\xi_A)^N \), if \( N^\circ \) is \( A \)-flexible then \( \xi(N)^\circ \) is \( A \)-flexible.
        \item Let \( A : \beta \rightsquigarrow \gamma \) and \( \delta, \varepsilon < \gamma \) be distinct proper type indices.
        Let \( t \) be a \( \delta \)-tangle, and let \( N \) be a near-litter such that \( N^\circ = f_{\delta,\varepsilon}(t) \) and
        \[ N \in \dom \left(\xi_{A \pcomp (\gamma > \varepsilon > \bot)}\right)^N \]
        Then for all addresses \( (B, x) \in \ran \supp t \), we have
        \[ x \in \dom \xi_{A \pcomp (\gamma > \delta) \pcomp B} \]
        Moreover, in this situation, if \( \rho \) is \( \gamma \)-allowable and satisfies
        \[ \xi_{A \pcomp (\gamma > \delta) \pcomp B}(x) = \rho_{(\gamma > \delta) \pcomp B}(x) \]
        for each such address, then
        \[ \xi_{A \pcomp (\gamma > \varepsilon > \bot)}(N)^\circ = f_{\delta,\varepsilon}(\rho_{(\gamma > \delta)}(t)) \]
        \item Let \( A : \beta \rightsquigarrow \gamma \) and \( \varepsilon < \gamma \) be a proper type index.
        Then for all atoms \( a \) and near-litters \( N \) such that \( N^\circ = f_{\bot,\varepsilon}(a) \) and
        \[ N \in \dom \left(\xi_{A \pcomp (\gamma > \varepsilon > \bot)}\right)^N \]
        we have
        \[ a \in \dom \left(\xi_{A \pcomp (\gamma > \bot)}\right)^A \]
        Moreover, in this situation, if \( \rho \) is \( \gamma \)-allowable and satisfies
        \[ \xi_{A \pcomp (\gamma > \bot)}(a) = \rho_{(\gamma > \bot)}(a) \]
        then
        \[ \xi_{A \pcomp (\gamma > \varepsilon > \bot)}(N)^\circ = f_{\bot,\varepsilon}(\rho_{(\gamma > \bot)}(a)) \]
    \end{enumerate}
\end{definition}

We can now state freedom of action for behaviours.

\begin{theorem}
    \label{thm:foa_behaviour}
    Every \( \beta \)-structural behaviour \( \xi \) that is both lawful and coherent approximates some \( \beta \)-allowable permutation \( \rho \).
\end{theorem}
\begin{proof}
    Augment \( \xi \) so that it is defined on all litters by using \cref{lem:foa:beh_to_action}.
    Define a structural action \( \psi \) by
    \[ (\psi_A)^A(a) = (\xi_A)^A(a);\quad (\psi_A)^L(L) = (\xi_A)^N(\NL(L)) \]
    One easily checks that this is a lawful action.
    By \cref{lem:foa:struct_action_to_approx}, we can convert it to a structural approximation \( \varphi \); we can do this by part (i) of the fact that \( \xi \) is coherent.
    We apply freedom of action to obtain a \( \beta \)-allowable permutation \( \rho \) that \( \varphi \) approximates.
    By induction on addresses, we can show directly that \( \xi \) has the same action as \( \rho \) on all addresses in its domain.
    The hypothesis needed for this induction to work is precisely parts (ii) and (iii) of the fact that \( \xi \) is coherent, as well as part (iii) of \cref{lem:foa:struct_action_to_approx}.
\end{proof}

Thus, we have established a sufficient condition for partial functions on atoms and near-litters to arise from an allowable permutation; it suffices to check that the corresponding behaviour is lawful and coherent.


\printbibliography

\end{document}
